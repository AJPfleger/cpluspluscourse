\documentclass[compress]{beamer}
\usetheme{Warsaw}
\useoutertheme{split}

\usepackage{minted}

%%%%%%%%%%%%%%%%%%%%%%%%%%%%%%%%%%%%%%%%%%%%%%%%%%%%%%%%%%%%%%%
% Improvement on the default split theme : added line numbers %
%%%%%%%%%%%%%%%%%%%%%%%%%%%%%%%%%%%%%%%%%%%%%%%%%%%%%%%%%%%%%%%

\setbeamercolor{frametitle}{fg=white}
\setbeamercolor{frametitle right}{fg=white}

\defbeamertemplate*{footline}{mysplit theme}
{%
  \leavevmode%
  \hbox{\begin{beamercolorbox}[wd=.5\paperwidth,ht=2.5ex,dp=1.125ex,leftskip=.3cm plus1fill,rightskip=.3cm]{author in head/foot}%
    \usebeamerfont{author in head/foot}\insertshortauthor
  \end{beamercolorbox}%
  \begin{beamercolorbox}[wd=.4\paperwidth,ht=2.5ex,dp=1.125ex,leftskip=.3cm,rightskip=.3cm plus1fil]{title in head/foot}%
    \usebeamerfont{title in head/foot}\insertshorttitle
  \end{beamercolorbox}}%
    \begin{beamercolorbox}[wd=.1\paperwidth,ht=2.5ex,dp=1.125ex,leftskip=.1cm plus1fill,rightskip=.1cm]{date in head/foot}
      \usebeamerfont{date in head/foot} \insertframenumber{} / \inserttotalframenumber 
    \end{beamercolorbox}
  \vskip0pt%
}

\defbeamertemplate*{headline}{mysplit theme}
{%
  \leavevmode%
  \begin{beamercolorbox}[wd=.45\paperwidth,ht=2.5ex,dp=1.125ex]{section in head/foot}%
    \insertsectionnavigationhorizontal{.4\paperwidth}{\hskip0pt plus1filll}{}%
  \end{beamercolorbox}%
  \begin{beamercolorbox}[wd=.55\paperwidth,ht=2.5ex,dp=1.125ex]{subsection in head/foot}%
    \insertsubsectionnavigationhorizontal{.6\paperwidth}{}{\hskip0pt plus1filll}%
  \end{beamercolorbox}%
}

\newcommand{\cpp}{C$^{++}$} 

\title{Exercises HowTo}
\author[S. Ponce]{S\'ebastien Ponce  \texttt{sebastien.ponce@cern.ch}}
\institute{CERN}
\date{January 2021}
\pgfdeclareimage[height=0.5cm]{cernlogo}{CERN-logo.jpg}
\logo{\pgfuseimage{cernlogo}}

\begin{document}

\begin{frame}
  \titlepage
\end{frame}

\begin{frame}
  \frametitle{Setup requirements}
  \begin{exampleblock}{Required}
    \begin{itemize}
    \item decent \cpp editor
    \item any \cpp compiler supporting \cpp17
    \end{itemize}
  \end{exampleblock}
  \begin{alertblock}{Good to have}
    \begin{itemize}
    \item \texttt{git} for getting/managing your code
    \item \texttt{gdb} to debug your problems
    \item valgrind, kcachegrind, cppcheck for corresponding exercises
    \end{itemize}
  \end{alertblock}
  \begin{block}{\cpp and python specific needs}
    \begin{itemize}
    \item python3, libpython3-dev
    \item ctypes, matplotlib, numpy python packages
    \end{itemize}
  \end{block}
\end{frame}

\begin{frame}[fragile]
  \setlength{\textwidth}{11.2cm}
  \frametitle{How to get exercises' code}
  \begin{exampleblock}{Best way}
    \begin{verbatim}
  
git clone https://gitlab.cern.ch/sponce/cpluspluscourse
cd cpluspluscourse/code
    \end{verbatim}
  \end{exampleblock}
  \begin{alertblock}{Alternative}
    \begin{verbatim}
  
wget http://cern.ch/sponce/C++Course/code.tgz
tar xzf code.tgz
cd code
    \end{verbatim}    
  \end{alertblock}
\end{frame}

\begin{frame}
  \frametitle{Anatomy of an exercise}
  Each exercise is in a subdirectory with mainly 3 sets of files
  \begin{exampleblock}{*.hpp and *.cpp files}
    \begin{itemize}
    \item the code to understand and fix/complete
    \item you never start from scratch
    \end{itemize}
  \end{exampleblock}
  \begin{block}{Makefile / cmake}
    \begin{itemize}
    \item prepared Makefile for easy compilation
    \item ``make'' and ``make clean'' are available
    \item cmake is also supported
    \end{itemize}
  \end{block}
  \begin{alertblock}{solution subdirectory}
    \begin{itemize}
    \item the solution to the exercise (``make solution'')
    \item please do not use before trying !
    \end{itemize}
  \end{alertblock}
\end{frame}

\begin{frame}
  \frametitle{Just follow the instructions}
  \begin{itemize}
  \item Each exercise comes with a set of instructions in the course
  \item See for example exercise on polymorphism
  \item Also present in exercise subdir in markdown format
  \end{itemize}
  \begin{alertblock}{Exercise Time}
    \begin{itemize}
    \item go to code/polymorphism
    \item look at the code
    \item open test.cpp
    \item create a Pentagon, call its perimeter method
    \item create an Hexagon, call its perimeter method
    \item create an Hexagon, call its parent's perimeter method
    \item retry with virtual methods
    \end{itemize}
  \end{alertblock}
\end{frame}

\begin{frame}[fragile]
  \frametitle{How to test your setup}
  \begin{exampleblock}{Just use hello exercise}
    \begin{itemize}
    \item go to code/hello
    \item follow the README
    \end{itemize}
  \end{exampleblock}
  \begin{block}{Practically}
    \begin{minted}{shell}
      cd code/hello
    \end{minted}
    \begin{itemize}
    \item g++ and make
      \begin{minted}[gobble=6]{shell}
        make;
        export LD_LIBRARY_PATH=${LD_LIBRARY_PATH}:.
        ./hello
      \end{minted}
    \item valgrind, callgrind and X11
      \begin{minted}[gobble=6]{shell}
        valgrind --tool=callgrind ./hello; kcachegrind
      \end{minted}
    \item cppcheck
      \begin{minted}[gobble=6]{shell}
        cppcheck .
      \end{minted}
    \end{itemize}
  \end{block}
\end{frame}

\begin{frame}
  \frametitle{How to get help}
  \Large Several channels available
  \begin{itemize}
  \item Training Notebook
  \item Mattermost channel
  \item Zoom rooms with mentors
  \end{itemize}
  All links and details on the indico page of the event
\end{frame}

\end{document}
