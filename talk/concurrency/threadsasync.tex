\subsection[thr]{Threads and async}

\begin{frame}[fragile]
  \frametitlecpp[11]{Basic concurrency}
  \begin{block}{Threading}
    \begin{itemize}
    \item \cpp11 added \mintinline{cpp}{std::thread} in \mintinline{cpp}{<thread>} header
    \item takes a function as argument of its constructor
    \item must be detached or joined before the main thread terminates
    \item \cpp20: \mintinline{cpp}{std::jthread} automatically joins at destruction
    \end{itemize}
  \end{block}
  \vspace{-1\baselineskip}
  \begin{overprint}
    \onslide<1>
    \begin{exampleblock}{Example code}
      \begin{cppcode*}{gobble=2}
        void foo() {...}
        void bar() {...}
        int main() {
          std::thread t1{foo};
          std::thread t2{bar};
          for (auto t: {&t1,&t2}) t->join();
          return 0;
        }
      \end{cppcode*}
    \end{exampleblock}
    \onslide<2>
    \begin{exampleblock}{Example with jthread (\cpp20)}
      \begin{cppcode*}{gobble=2}
        void foo() {...}
        void bar() {...}
        int main() {
          std::jthread t1{foo};
          std::jthread t2{bar};
          return 0;
        }
      \end{cppcode*}
    \end{exampleblock}
  \end{overprint}
\end{frame}

\begin{frame}[fragile]
  \frametitlecpp[11]{The thread constructor}
  \begin{exampleblock}{Can take a function and its arguments}
    \begin{cppcode*}{}
      void function(int j, double j) {...};
      std::thread t1{function, 1, 2.0};
    \end{cppcode*}
  \end{exampleblock}
  \pause
  \begin{exampleblock}{Can take any function-like object}
    \begin{cppcode*}{}
      struct AdderFunctor {
        AdderFunctor(int i): m_i(i) {}
        int operator() (int j) const { return m_i+j; }
        int m_i;
      };
      std::thread t2{AdderFunctor{2}, 5};
      int a;
      std::thread t3{[](int i) { return i+2; }, a};
      std::thread t4{[a]       { return a+2; }};
    \end{cppcode*}
  \end{exampleblock}
\end{frame}

\begin{frame}[fragile]
  \frametitlecpp[11]{Basic asynchronicity}
  \begin{block}{Concept}
    \begin{itemize}
    \item separation of the specification of what should be done and the retrieval of the results
    \item ``start working on this, and let me check if it's done''
    \end{itemize}
  \end{block}
  \pause
  \begin{block}{Practically}
    \begin{itemize}
    \item \mintinline{cpp}{std::async} function launches an asynchronous task
    \item \mintinline{cpp}{std::future<T>} allows to retrieve the result
    \end{itemize}
  \end{block}
  \pause
  \begin{exampleblock}{Example code}
    \begin{cppcode*}{}
      int f() {...}
      std::future<int> res = std::async(f);
      std::cout << res.get() << "\n";
    \end{cppcode*}
  \end{exampleblock}
\end{frame}

\begin{frame}[fragile]
  \frametitlecpp[11]{Mixing the two}
  \begin{block}{Is async running concurrently ?}
    \begin{itemize}
    \item it depends!
    \item you can control this with a launch policy argument
      \begin{description}
      \item[std::launch::async] start executing immediately on a separate thread (may be a new thread, a thread pool, ...)
      \item[std::launch::deferred] causes lazy execution in current thread
      \end{description}
      \begin{itemize}
      \item execution starts when \mintinline{cpp}{get()} is called on the returned future
      \end{itemize}
    \item default is not specified, but tries to use existing concurrency!
    \end{itemize}
  \end{block}
  \pause
  \begin{exampleblock}{Usage}
    \begin{cppcode*}{}
      int f() {...}
      auto res1 = std::async(std::launch::async, f);
      auto res2 = std::async(std::launch::deferred, f);
    \end{cppcode*}
  \end{exampleblock}
\end{frame}

\begin{frame}[fragile]
  \frametitlecpp[11]{Fine-grained control on asynchronous execution}
  \begin{block}{\texttt{std::packaged\_task} template}
    \begin{itemize}
    \item creates an asynchronous version of any function-like object
      \begin{itemize}
      \item identical arguments
      \item returns a \mintinline{cpp}{std::future}
      \end{itemize}
    \item provides access to the returned future
    \item associated with threads, gives full control on execution
    \end{itemize}
  \end{block}
  \pause
  \begin{exampleblock}{Usage}
    \begin{cppcode*}{}
      int f() { return 42; }
      std::packaged_task<int()> task{f};
      auto future = task.get_future();
      task();
      std::cout << future.get() << std::endl;
    \end{cppcode*}
  \end{exampleblock}
\end{frame}
