\subsection[thr]{Threads and async}

\begin{frame}[fragile]
  \frametitlecpp[11]{Basic concurrency}
  \begin{block}{Threading}
    \begin{itemize}
    \item new object std::thread in \textless{}thread\textgreater{} header
    \item takes a function as argument of its constructor
    \item must be detached or joined before the main thread terminates
    \item \cpp20: std::jthread automatically joins at destruction
    \end{itemize}
  \end{block}
  \pause
  \begin{exampleblock}{Example code}
    \begin{cppcode*}{}
      void doSth() {...}
      void doSthElse() {...}
      int main() {
        std::thread t1(doSth);
        std::thread t2(doSthElse);
        for (auto t: {&t1,&t2}) t->join();
      }
    \end{cppcode*}
  \end{exampleblock}
\end{frame}

\begin{frame}[fragile]
  \frametitlecpp[11]{The thread constructor}
  \begin{exampleblock}{Can take a function and its arguments}
    \begin{cppcode*}{}
      void function(int j, double j) {...};
      std::thread t1(function, 1, 2.0);
    \end{cppcode*}
  \end{exampleblock}
  \pause
  \begin{exampleblock}{Can take any function-like object}
    \begin{cppcode*}{}
      struct AdderFunctor {
        AdderFunctor(int i): m_i(i) {}
        int operator() (int j) const { return i+j; };
        int m_i;
      };
      std::thread t2(AdderFunctor(2), 5);
      int a;
      std::thread t3([](int i) { return i+2; }, a);
      std::thread t4([a]       { return a+2; });
    \end{cppcode*}
  \end{exampleblock}
\end{frame}

\begin{frame}[fragile]
  \frametitlecpp[11]{Basic asynchronicity}
  \begin{block}{Concept}
    \begin{itemize}
    \item separation of the specification of what should be done and the retrieval of the results
    \item ``start working on this, and ping me when it's ready''
    \end{itemize}
  \end{block}
  \pause
  \begin{block}{Practically}
    \begin{itemize}
    \item std::async function launches an asynchronous task
    \item std::future template allows to handle the result
    \end{itemize}
  \end{block}
  \pause
  \begin{exampleblock}{Example code}
    \begin{cppcode*}{}
      int computeSth() {...}
      std::future<int> res = std::async(computeSth);
      std::cout << res->get() << std::endl;
    \end{cppcode*}
  \end{exampleblock}
\end{frame}

\begin{frame}[fragile]
  \frametitlecpp[11]{Mixing the two}
  \begin{block}{Is async running concurrent code ?}
    \begin{itemize}
    \item it depends!
    \item you can control this with a launch policy argument
      \begin{description}
      \item[std::launch::async] spawns a thread for immediate execution
      \item[std::launch::deferred] causes lazy execution in current thread
      \end{description}
      \begin{itemize}
      \item execution starts when get() is called
      \end{itemize}
    \item default is not specified!
    \end{itemize}
  \end{block}
  \pause
  \begin{exampleblock}{Usage}
    \begin{cppcode*}{}
      int computeSth() {...}
      auto res = std::async(std::launch::async,
                            computeSth);
      auto res2 = std::async(std::launch::deferred,
                             computeSth);
    \end{cppcode*}
  \end{exampleblock}
\end{frame}

\begin{frame}[fragile]
  \frametitlecpp[11]{Fine grained control on asynchronous execution}
  \begin{block}{std::packaged\_task template}
    \begin{itemize}
    \item creates an asynchronous version of any function-like object
      \begin{itemize}
      \item identical arguments
      \item returns a std::future
      \end{itemize}
    \item provides access to the returned future
    \item associated with threads, gives full control on execution
    \end{itemize}
  \end{block}
  \pause
  \begin{exampleblock}{Usage}
    \begin{cppcode*}{}
      int task() { return 42; }
      std::packaged_task<int()> pckd_task(task);
      auto future = pckd_task.get_future();
      pckd_task();
      std::cout << future.get() << std::endl;
    \end{cppcode*}
  \end{exampleblock}
\end{frame}
