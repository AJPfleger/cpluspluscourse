\section[base]{Langage basics}

\subsection[Core]{Core syntax and types}

\begin{frame}[fragile]
  \frametitlegb{Hello World}
  \begin{cppcode}
    #include <iostream>

    // This is a function
    void print(int i) {
      std::cout << "Hello, world " << i << std::endl;
    }

    int main(int argc, char** argv) {
      int n = 3;
      for (int i = 0; i < n; i++) {
        print(i);
      }
      return 0;
    }
  \end{cppcode}
\end{frame}

\begin{frame}[fragile]
  \frametitlegb{Comments}
  \begin{cppcode}
    // simple comment for integer declaration
    int i;

    /* multiline comment
     * in case we need to say more
     */
    double d;

    /**
     * Best choice : doxygen compatible comments
     * \fn bool isOdd(int i)
     * \brief checks whether i is odd
     * \param i input
     * \return true if i is odd, otherwise false
     */
    bool isOdd(int i);
  \end{cppcode}
\end{frame}

\begin{frame}[fragile]
  \frametitlegb{Basic types(1)}
  \begin{cppcode}
    bool b = true;            // boolean, true or false
    
    char c = 'a';             // 8 bits ASCII char
    char* s = "a C string";   // array of chars ended by \0
    string t = "a C++ string";// class provided by the STL

    char c = -3;              // 8 bits signed integer
    unsigned char c = 4;      // 8 bits unsigned integer

    short int s = -444;       // 16 bits signed integer
    unsigned short s = 444;   // 16 bits unsigned integer
    short s = -444;           // int is optional
  \end{cppcode}
\end{frame}
\begin{frame}[fragile]
  \frametitlegb{Basic types(2)}
  \begin{cppcode}
    int i = -123456;          // 32 bits signed integer
    unsigned int i = 1234567; // 32 bits signed integer

    long l = 0L               // 32 or 64 bits (ptr size)
    unsigned long l = 0UL;    // 32 or 64 bits (ptr size)

    long long ll = 0LL;       // 64 bits signed integer
    unsigned long long l = 0ULL; // 64 bits unsigned integer

    float f = 1.23f;          // 32 (23+7+1) bits float
    double d = 1.23E34;       // 64 (52+11+1) bits float
  \end{cppcode}
\end{frame}

\begin{frame}[fragile]
  \frametitlegb{Portable numeric types}
  \alert{One needs to include specific header}
  \begin{cppcode}
    #include <cstdint>
    
    int8_t c = -3;     // 8 bits, replaces char
    uint8_t c = 4;     // 8 bits, replaces unsigned char

    int16_t s = -444;  // 16 bits, replaced short
    uint16_t s = 444;  // 16 bits, replaced unsigned short

    int32_t s = -0674; // 32 bits, replaced int
    uint32_t s = 0674; // 32 bits, replaced unsigned int

    int64_t s = -0x1bc;// 64 bits, replaced long long
    uint64_t s = 0x1bc;// 64 bits, replaced unsigned long long
    \end{cppcode}
\end{frame}

\subsection[Ptr]{Arrays and Pointers}

\begin{frame}[fragile]
  \frametitlegb{Static arrays}
  \begin{cppcode}
    int ai[4] = {1,2,3,4};
    int ai[] = {1,2,3,4};  // identical
    
    char ac[3] = {'a','b','c'};   // char array
    char ac[4] = "abc";           // valid C string
    char ac[4] = {'a','b','c',0}; // same valid string
    
    int i = ai[2];  // i = 3
    char c = ac[8]; // at best garbage, may segfault
    int i = ai[4];  // also garbage !
  \end{cppcode}
\end{frame}

\Scontents*[store-cmd=code_arrays]{
int i = 4;
int *pi = &i;
int j = *pi + 1;

int ai[] = {1,2,3};
int *pai = ai;
int *paj = pai + 1;
int k = *paj + 1;

// not compiling
int *pak = k;

// seg fault !
int *pak = (int*)k;
int l = *pak;
}
\begin{frame}[fragile]
  \frametitlegb{Pointers}
  \begin{multicols}{2}
  \begin{overprint}[\columnwidth]
    \onslide<1> \highlightCppCode{}{code_arrays}
    \onslide<2> \highlightCppCode{1}{code_arrays}
    \onslide<3> \highlightCppCode{2}{code_arrays}
    \onslide<4> \highlightCppCode{3}{code_arrays}
    \onslide<5> \highlightCppCode{5}{code_arrays}
    \onslide<6> \highlightCppCode{6}{code_arrays}
    \onslide<7> \highlightCppCode{7}{code_arrays}
    \onslide<8> \highlightCppCode{8}{code_arrays}
    \onslide<9> \highlightCppCode{11}{code_arrays}
  \end{overprint}
  \columnbreak
    \onslide<2->{
      \begin{tikzpicture}
        \memorystack[size x=3cm,word size=1,nb blocks=11]
        \onslide<2-> {\memorypush{$i = 4$}}
        \onslide<3-> {\memorypushpointer[pi =]{1}}
        \onslide<4-> {\memorypush{$j = 5$}}
        \onslide<5-> {\memorypush{$1$}}
        \onslide<5-> {\memorypush{$2$}}
        \onslide<5-> {\memorypush{$3$}}
        \onslide<6-> {\memorypushpointer[pai =]{4}}
        \onslide<7-> {\memorypushpointer[paj =]{5}}
        \onslide<8-> {\memorypush{$k = 3$}}
        \onslide<9-> {\memorypush{$pak = 3$}}
        \onslide<9-> {\draw[\stackcolor!80,->] (stack10-1.west) -- +(-0.5cm,0)
          node [anchor=east] {??};}
      \end{tikzpicture}
    }
  \end{multicols}
\end{frame}

\begin{frame}[fragile]
  \frametitleii{nullptr}
  \begin{block}{Finally a \cpp~NULL pointer}
    \begin{itemize}
    \item if a pointer doesn't point to anything, set it to \mintinline{cpp}{nullptr}
    \item works like 0 or NULL in standard cases
    \item triggers compilation error when mapped to integer
    \end{itemize}
  \end{block}
  \pause
  \begin{exampleblock}{Example code}
    \begin{cppcode*}{}
      void* vp = nullptr;
      int* ip = nullptr;
      int i = NULL;      // OK -> bug ?
      int i = nullptr;   // ERROR
    \end{cppcode*}
  \end{exampleblock}
\end{frame}

\begin{frame}[fragile]
  \frametitlegb{Dynamic Arrays}
  \begin{cppcode}
    #include <cstdlib>
    #include <cstring>

    int *bad;          // pointer to random address
    int *ai = nullptr; // better. Can be tested

    // allocate array of 10 ints (not initialized)
    ai = (int*) malloc(10*sizeof(int));
    // and set them to 0
    memset(ai, 0, 10*sizeof(int));

    // Both in one go
    ai = (int*) calloc(10, sizeof(int));
    
    // liberate memory
    free(ai);
  \end{cppcode}
\end{frame}


\subsection[Compound]{Compound data types}

\begin{frame}[fragile]
  \frametitlegb{struct}
  \begin{mdframed}[style=simplebox]
    \center ``members'' grouped together under one name
  \end{mdframed}
  \begin{multicols}{2}
    \begin{cppcode*}{gobble=2}
      struct Individual {
        unsigned char age;
        float weight;
      };

      Individual student;
      student.age = 25;
      student.weight = 78.5;

      Individual teacher = {
        45, 67
      };
    \end{cppcode*}
    \columnbreak
    \begin{cppcode*}{gobble=2,firstnumber=14}
      Individual *studentPtr =
        &student;
      studentPtr->age = 25;
    \end{cppcode*}
    \pause
    \vfill
    \hspace{-1.5cm}
    \begin{tikzpicture}
      \memorystack[nb blocks=5]
      \onslide<3-> {
        \memorypush{25,?,?,?}
        \memorypushwidevalue{78.5}
        \memorystruct{1}{2}{\scriptsize student}
      }
      \onslide<4-> {
        \memorypush{45,?,?,?}
        \memorypushwidevalue{67.0}
        \memorystruct{3}{4}{\scriptsize teacher}
      }
    \end{tikzpicture}
    \vfill \null
  \end{multicols}
\end{frame}

\begin{frame}[fragile]
  \frametitlegb{union}
  \begin{mdframed}[style=simplebox]
    \center ``members'' packed together at same memory location
  \end{mdframed}
  \begin{multicols}{2}
    \begin{cppcode*}{gobble=2}
      union Duration {
        int seconds;
        short hours;
        char days;
      };

      Duration d1, d2, d3;
      d1.seconds = 259200;
      d2.hours = 72;
      d3.days = 3;
      d1.days = 3; // d1.seconds overwritten
      int a = d1.seconds; // d1.seconds is garbage
    \end{cppcode*}
    \pause
    \columnbreak
    \null \vfill
    \begin{tikzpicture}
      \clip (0,0) rectangle (6cm, 3cm);
      \memorystack[word size=4,nb blocks=4]
      \visible<3-5>{\memorypushwidevalue{d1 259200}}
      \onslide<4->{\memorypushhalfvalue{d2 72}}
      \memorygoto{2}
      \onslide<4->{\memorypush{,,?,?}}
      \onslide<5->{\memorypush{d3 3,?,?,?}}
      \memorygoto{1}
      \onslide<6->{\memorypush{d1 3,?,?,?}}
    \end{tikzpicture}
    \vfill \null
  \end{multicols}
\end{frame}

\begin{frame}[fragile]
  \frametitlegb{Enums}
  \begin{multicols}{2}
    \begin{cppcode*}{gobble=2}
      enum VehicleType {
        BIKE,  // 0
        CAR,   // 1
        BUS,   // 2
      };
      VehicleType t = CAR;
    \end{cppcode*}
    \columnbreak
    \begin{cppcode*}{gobble=2}
      enum VehicleType {
        BIKE = 3,
        CAR = 5,
        BUS = 7,
      };
      VehicleType t2 = BUS;
    \end{cppcode*}
  \end{multicols}
\end{frame}

\begin{frame}[fragile]
  \frametitleii{Scoped enumeration, aka enum class}
  \begin{block}{Same syntax as enum, with scope}
    \begin{cppcode*}{}
      enum class VehicleType { Bus, Car };
      VehicleType t = VehicleType::Car;
    \end{cppcode*}
  \end{block}
  \pause
  \begin{exampleblock}{Only advantages}
    \begin{itemize}
    \item scoping avoids name clashes
    \item strong typing, no automatic conversion to int
    \end{itemize}
    \small
    \begin{cppcode*}{}
      enum VType { Bus, Car }; enum Color { Red, Blue };
      VType t = Bus;
      if (t == Red) {  // We do enter !  }
      int a = 5 * Car; // Ok, a = 5
      
      enum class VT { Bus, Car }; enum class Col { Red, Blue };
      VT t = VT::Bus;
      if (t == Col::Red) { // Compiler error }
      int a = t * 5;       // Compiler error
    \end{cppcode*}
  \end{exampleblock}
\end{frame}

\begin{frame}[fragile]
  \frametitlegb{More sensible example}
  \begin{multicols}{2}
    \begin{cppcode*}{gobble=2}
      enum class ShapeType {
        Circle,
        Rectangle
      };
      
      struct Rectangle {
        float width;
        float height;
      };
    \end{cppcode*}
    \columnbreak
    \pause
    \begin{cppcode*}{gobble=2,firstnumber=10}
      struct Shape {
        ShapeType type;
        union { 
          float radius;
          Rectangle rect;
        };
      };
    \end{cppcode*}
  \end{multicols}
  \pause
  \begin{multicols}{2}
    \begin{cppcode*}{gobble=2,firstnumber=17}
      Shape s;
      s.type =
        ShapeType::Circle;
      s.radius = 3.4;
      
    \end{cppcode*}
    \columnbreak
    \begin{cppcode*}{gobble=2,firstnumber=20}
      Shape t;
      t.type =
        Shapetype::Rectangle;
      t.rect.width = 3;
      t.rect.height = 4;
    \end{cppcode*}
  \end{multicols}
\end{frame}

\begin{frame}[fragile]
  \frametitle{typedef and using \hfill \cpp98 / \cpp11}
  Ways to create type aliases
  \begin{alertblock}{\cpp98}
    \begin{cppcode*}{gobble=2}
      typedef uint64_t myint;
      myint toto = 17;
    \end{cppcode*}
  \end{alertblock}
  \begin{exampleblock}{\cpp11}
    \begin{cppcode*}{gobble=2}
      using myint = uint64_t;
      myint toto = 17;

      template <typename T> using myvec = std::vector<T>;
      myvec<int> titi;
    \end{cppcode*}
  \end{exampleblock}
\end{frame}


\subsection[Refs]{References}

\begin{frame}[fragile]
  \frametitlegb{References}
  \begin{block}{References}
    \begin{itemize}
      \item References allow for direct access to another object
      \item They can be used as shortcuts / better readability
      \item They can be declared \mintinline{cpp}{const} to allow only read access
      \item They can be used as function arguments
    \end{itemize}
  \end{block}

  \begin{exampleblock}{Example:}
    \begin{cppcode*}{gobble=2}
      int i = 2;
      int &iref = i; // access to i
      iref = 3;      // i is now 3

      // const reference to a member:
      struct A { int x; int y; } a;
      const int &x = a.x; // direct read access to A's x
      x = 4;              // doesn't compile
    \end{cppcode*}
  \end{exampleblock}
\end{frame}

\begin{frame}[fragile]
  \frametitlegb{Pointers vs References}
  \begin{block}{Specificities of reference}
    \begin{itemize}
    \item natural syntax
    \item will never be \mintinline{cpp}{nullptr}
    \item thus cannot reference temporary object
    \end{itemize}
  \end{block}
  \begin{block}{Advantages of pointers}
    \begin{itemize}
    \item can be \mintinline{cpp}{nullptr}
    \item clearly indicates that argument may be modified
    \end{itemize}
  \end{block}
  \pause
  \begin{alertblock}{Good practice}
    \begin{itemize}
      \item Always use references when you can
      \item Consider that a reference will be modified
      \item Use constness when it's not the case
    \end{itemize}
  \end{alertblock}
\end{frame}


\subsection[$f()$]{Functions}

\begin{frame}[fragile]
  \frametitlegb{Functions}
  \begin{multicols}{2}
    \begin{cppcode*}{gobble=2}
      // with return type
      int square(int a) {
        return a * a;
      }

      // multiple parameters
      int mult(int a,
               int b) {
        return a*b;
      }
    \end{cppcode*}
    \columnbreak
    \begin{cppcode*}{gobble=2,firstnumber=11}
      // no return
      void log(char* msg) {
        printf("%s", msg);
      }

      // no parameter
      void hello() {
        printf("Hello World");
      }
    \end{cppcode*}
  \end{multicols}
\end{frame}

\Scontents*[store-cmd=code_bigStruct]{
struct BigStruct {...};
BigStruct s;

// parameter by value
void printBS(BigStruct p) {
  ...
}
printBS(s); // copy

// parameter by reference
void printBSp(BigStruct &q) {
  ...
}
printBSp(s); // no copy
}
\begin{frame}[fragile]
  \frametitlegb{Functions: parameters are passed by value}
  \begin{multicols}{2}
    \begin{overprint}[\columnwidth]
      \onslide<1-2>
      \highlightCppCode{2}{code_bigStruct}
      \onslide<3>
      \highlightCppCode{5,8}{code_bigStruct}
      \onslide<4->
      \highlightCppCode{11,14}{code_bigStruct}
    \end{overprint}
    \columnbreak
    \null \vfill
    \begin{tikzpicture}
      \memorystack[word size=1, nb blocks=7, size x=3cm]
      \onslide<2-> {
        \memorypush{s1}
        \memorypush{...}
        \memorypush{sn}
        \memorystruct{1}{3}{s}
      }
      \onslide<3> {
        \memorypush{p1 = s1}
        \memorypush{...}
        \memorypush{pn = sn}
        \memorystruct{4}{6}{p}
      }
      \memorygoto{4}
      \onslide<4> {
        \memorypushpointer[q =]{1}
      }
    \end{tikzpicture}
    \vfill \null
  \end{multicols}
\end{frame}

\Scontents*[store-cmd=code_smallStruct]{
struct SmallStruct {int a;}
SmallStruct s = {1};

void changeSS(SmallStruct p) {
  p.a = 2;
}
changeSS(s);
// s.a = 1

void changeSS2(SmallStruct &q) {
  q.a = 2;
}
changeSS2(s);
// s.a = 2
}
\begin{frame}[fragile]
  \frametitlegb{Functions: pass by value or reference?}
  \begin{multicols}{2}
    \begin{overprint}[\columnwidth]
      \onslide<1>
      \highlightCppCode{}{code_smallStruct}
      \onslide<2>
      \highlightCppCode{2}{code_smallStruct}
      \onslide<3>
      \highlightCppCode{4,7}{code_smallStruct}
      \onslide<4>
      \highlightCppCode{5}{code_smallStruct}
      \onslide<5>
      \highlightCppCode{8}{code_smallStruct}
      \onslide<6>
      \highlightCppCode{10,13}{code_smallStruct}
      \onslide<7>
      \highlightCppCode{11}{code_smallStruct}
      \onslide<8>
      \highlightCppCode{14}{code_smallStruct}
    \end{overprint}
    \columnbreak
    \null \vfill
    \begin{tikzpicture}
      \memorystack[word size=1, nb blocks=3, size x=3cm]
      \onslide<2-6> {
        \memorypush{s.a = 1}
      }
      \memorygoto{1}
      \onslide<7-> {
        \memorypush{s.a = 2}
      }

      \memorygoto{2}
      \onslide<3> {
        \memorypush{p.a = 1}
      }
      \memorygoto{2}
      \onslide<4> {
        \memorypush{p.a = 2}
      }
      \memorygoto{2}
      \onslide<6-7> {
        \memorypushpointer[q =]{1}
      }
      \end{tikzpicture}
    \vfill \null
  \end{multicols}
\end{frame}

\begin{frame}[fragile]
  \frametitlegb{Value vs. pointers and references}
  \begin{block}{Different ways to pass arguments to a function}
    \begin{itemize}
    \item by default arguments are passed by value (= copy, good for small types, e.g.\ numbers)
    \item references are preferred to avoid copies
    \item same for pointers, but less favoured
    \item use \mintinline{cpp}{const} for safety
    \end{itemize}
  \end{block}
  \pause
  \begin{block}{Syntax}
    \begin{cppcode*}{escapeinside=||}
struct T {...}; T a;
void f(T value);           f(a);      // by value
void fRef(const T &value); fRef(a);   // by reference
void fPtr(const T *value); fPtr(|{\setlength{\fboxsep}{0pt}\color{gray}\colorbox{yellow}{\textsc{&}}}|a);  // by pointer
void fWrite(T &value);     fWrite(a); // non-const ref
    \end{cppcode*}
  \end{block}
\end{frame}

\begin{frame}[fragile]
  \frametitlegb{Functions}
  \begin{alertblock}{Exercise}
    Familiarise yourself with pass by copy / pass by reference.
    \begin{itemize}
      \item go to \texttt{code/functions}
      \item Look at \texttt{functions.cpp}
      \item Compile it (\texttt{make}) and run the program (\texttt{./functions})
      \item Work on the tasks that you find in \texttt{functions.cpp}
    \end{itemize}
  \end{alertblock}
\end{frame}

\begin{frame}[fragile]
  \frametitlegb{Functions: good practices}
  \begin{onlyenv}<1>
    \begin{block}{Good practices with functions}
      \begin{itemize}
        \item Keep functions short
        \item Do one logical thing
        \item Use expressive names
        \item Document the functions
      \end{itemize}
    \end{block}
    \begin{exampleblock}{Example: Good}
      \begin{cppcode*}{gobble=2}
        /// Count number of dilepton events in data.
        /// \param d Dataset to search.
        unsigned int countDileptons(Data d) {
          selectEventsWithMuons(d);
          selectEventsWithElectrons(d);
          return d.size();
        }
      \end{cppcode*}
    \end{exampleblock}
  \end{onlyenv}
  \begin{onlyenv}<2->
    \begin{alertblock}{Example: don't! Everything in one long function}
      \begin{multicols}{2}
        \begin{cppcode*}{gobble=6}
          unsigned int foo() {
            // Step 1: data
            Data data;
            data.resize(123456);
            data.fill(...);

            // Step 2: muons
            for (....) {
              if (...) {
                data.delete(...);
              }
            }
            // Step 3: electrons
            for (....) {
        \end{cppcode*}
        \columnbreak
        \begin{cppcode*}{gobble=6,firstnumber=last}
              if (...) {
                data.delete(...);
              }
            }

            // Step 4: dileptons
            int counter = 0;
            for (....) {
              if (...) {
                counter++;
              }
            }

            return counter;
          }
        \end{cppcode*}
      \end{multicols}
    \end{alertblock}
  \end{onlyenv}
\end{frame}


\subsection[Op]{Operators}

\begin{frame}[fragile]
  \frametitlegb{Operators(1)}
  \begin{block}{Binary \& Assignment Operators}
    \begin{cppcode*}{}
      int i = 1 + 4 - 2;  // 3
      i *= 3;             // 9
      i /= 2;             // 4
      i = 23 % i;         // modulo => 3
    \end{cppcode*}
  \end{block}
  \pause
  \begin{block}{Increment / Decrement \uncover<3->{\hfill \alert{\bf Use wisely}}}
    \begin{cppcode*}{}
      int i = 0; i++; // i = 1
      int j = ++i;    // i = 2, j = 2
      int k = i++;    // i = 3, k = 2
      int l = --i;    // i = 2, l = 2
      int m = i--;    // i = 1, m = 2
    \end{cppcode*}
  \end{block}
\end{frame}

\begin{frame}[fragile]
  \frametitlegb{Operators(2)}
  \begin{block}{Bitwise and Assignment Operators}
    \begin{cppcode*}{}
      int i = 0xee & 0x55;  // 0x44
      i |= 0xee;            // 0xee
      i ^= 0x55;            // 0xbb
      int j = ~0xee;        // 0xffffff11
      int k = 0x1f << 3;    // 0x78
      int l = 0x1f >> 2;    // 0x7
    \end{cppcode*}
  \end{block}
  \pause
  \begin{block}{Boolean Operators}
    \begin{cppcode*}{}
      bool a = true;
      bool b = false;
      bool c = a && b;    // false
      bool d = a || b;    // true
      bool e = !d;        // false
    \end{cppcode*}
  \end{block}
\end{frame}

\begin{frame}[fragile]
  \frametitlegb{Operators(3)}
  \begin{block}{Comparison Operators}
    \begin{cppcode*}{}
      bool a = (3 == 3);  // true
      bool b = (3 != 3);  // false
      bool c = (4 <  4);  // false
      bool d = (4 <= 4);  // true
      bool e = (4 >  4);  // false
      bool f = (4 >= 4);  // true
    \end{cppcode*}
  \end{block}
  \pause
  \begin{block}{Precedences \uncover<3->{\hfill \alert{\bf Don't use}\uncover<4->{\color{green} \bf\ - use parenthesis}}}
    \begin{cppcode*}{linenos=false}
      c &= 1+(++b)|(a--)*4%5^7; // ???
    \end{cppcode*}
  \end{block}
\end{frame}

\subsection[NS]{Scopes / namespaces}

\begin{frame}[fragile]
  \frametitlegb{Scope}
  \begin{block}{Definition}
    Portion of the source code where a given name is valid \\
    Typically :
    \begin{itemize}
    \item simple block of code, within \mintinline{cpp}{{}}
    \item function, class, namespace
    \item source file for global declarations
    \end{itemize}
  \end{block}
  \begin{exampleblock}{Example}
    \begin{cppcode*}{}
      { int a;
        { int b;
        } // end of b scope
      } // end of a scope
    \end{cppcode*}
  \end{exampleblock}
\end{frame}

\Scontents*[store-cmd=code_scopes]{
int a = 1;
{
  int b[4];
  b[0] = a;
}
// Doesn't compile:
// b[1] = a + 1;
}
\begin{frame}[fragile]
  \frametitlegb{Scope and lifetime of resources}
  \begin{block}{Resource life time}
    \begin{itemize}
      \item Resources are allocated when declared
      \item Resources are freed at the end of a scope
      \item Good practice: initialise resources when allocating!
    \end{itemize}
  \end{block}
  \begin{multicols}{2}
    \begin{overprint}[\columnwidth]
    \onslide<1>
    \highlightCppCode{1}{code_scopes}
    \onslide<2>
    \highlightCppCode{3}{code_scopes}
    \onslide<3>
    \highlightCppCode{4}{code_scopes}
    \onslide<4>
    \highlightCppCode{7}{code_scopes}
    \end{overprint}

    \columnbreak

    \begin{tikzpicture}
      \memorystack[word size=1, nb blocks=5, size x = 0.5\columnwidth]
      \onslide<1-> {
        \memorypush{a = 1}
      }
      \onslide<2>{
        \memorypush{b[0] = ?}
      }
      \memorygoto{2}
      \onslide<3>{
        \memorypush{b[0] = 1}
      }
      \memorygoto{3}
      \onslide<2-3>{
        \memorypush{b[1] = ?}
        \memorypush{b[2] = ?}
        \memorypush{b[3] = ?}
      }

      \memorygoto{2}
      \onslide<4>{
        \memorypush{\color{gray} 1}
        \memorypush{\color{gray} ?}
        \memorypush{\color{gray} ?}
        \memorypush{\color{gray} ?}
        }

    \end{tikzpicture}

  \end{multicols}
\end{frame}

\begin{frame}[fragile]
  \frametitlegb{Namespaces}
  \begin{itemize}
  \item Namespaces allow to segment your code to avoid name clashes
  \item They can be embedded to create hierarchies (separator is '::')
  \end{itemize}
  \begin{multicols}{2}
    \begin{cppcode*}{gobble=2}
      int a;
      namespace n {
        int a;   // no clash
      }
      namespace p {
        int a;   // no clash
        namespace inner {
          int a; // no clash
        }
      }
      int f() {
        n::a = 2;
      }
    \end{cppcode*}
    \columnbreak
    \begin{cppcode*}{gobble=2,firstnumber=14}
      namespace p {
        int f() {
          p::a = 2;
          a = 2;  //same as above
          p::inner::a = 4;
          inner::a = 4;
          n::a = 5;
        }
      }
      using namespace p::inner;
      int g() {
        a = 3; // using p::inner
      }
  \end{cppcode*}
  \end{multicols}
\end{frame}

\begin{frame}[fragile]
  \frametitleit{Nested namespaces}
  Easier way to declare nested namespaces
  \begin{alertblock}{\cpp14}
    \begin{cppcode*}{}
      namespace A {
        namespace B {
          namespace C {
            //...
          }
        }
      }
    \end{cppcode*}
  \end{alertblock}
  \begin{exampleblock}{\cpp17}
    \begin{cppcode*}{}
      namespace A::B::C {
        //...
      }
    \end{cppcode*}
  \end{exampleblock}
\end{frame}

\begin{frame}[fragile]
  \frametitlegb{Anonymous namespaces}
  \begin{exampleblock}{A namespace without a name !}
    \begin{cppcode*}{}
      namespace {
        int localVar;
      }
    \end{cppcode*}
  \end{exampleblock}
  \begin{block}{Purpose}
    \begin{itemize}
    \item groups a number of declarations
    \item visible in the current translation unit
    \item but not reusable outside
    \item allows much better compiler optimizations and checking
      \begin{itemize}
      \item e.g. unused function warning
      \item context dependent optimizations
      \end{itemize}
    \end{itemize}
  \end{block}
  \begin{alertblock}{Deprecates static}
    \begin{cppcode*}{gobble=2}
      static int localVar; // equivalent C code
    \end{cppcode*}
  \end{alertblock}
\end{frame}


\subsection[Control]{Control instructions}

\begin{frame}[fragile]
  \frametitlegb{Control instructions : if}
  \begin{block}{if syntax}
    \begin{cppcode*}{}
      if (condition1) {
        Instructions1;
      } else if (condition2) {
        Instructions2;
      } else {
        Instructions3;
      } 
    \end{cppcode*}
    \begin{itemize}
      \item {\it else} and {\it else if} part are optional
      \item {\it else if} part can be repeated
      \item braces are optional if there is a single instruction
    \end{itemize}
  \end{block}
\end{frame}

\begin{frame}[fragile]
  \frametitlegb{Control instructions : if}
  \begin{exampleblock}{Practical example}
    \begin{cppcode*}{}
      int collatz(int a) {
        if (a <= 0) {
          std::cout << "not supported";
          return 0;
        } else if (a == 1) {
          return 1;
        } else if (a%2 == 0) {
          return collatz(a/2);
        } else {
          return collatz(3*a+1);
        }
      }
    \end{cppcode*}
  \end{exampleblock}
\end{frame}

\begin{frame}[fragile]
  \frametitlegb{Control instructions : conditional operator}
  \begin{block}{Syntax}
    \begin{cppcode*}{linenos=false}
      test ? expression1 : expression2;
    \end{cppcode*}
    \vspace{-0.3cm}
    \begin{itemize}
      \item if test is {\it true} expression1 is returned
      \item else expression 2 is returned
    \end{itemize}
  \end{block}
  \pause
  \begin{exampleblock}{Practical example}
    \begin{cppcode*}{}
      int collatz(int a) {
        return a==1 ? 1 : collatz(a%2 ? 3*a+1 : a/2);
      }
    \end{cppcode*}
  \end{exampleblock}
  \pause
  \begin{alertblock}{Do not abuse}
    explicit ifs are easier to read \\
    to be used only when obvious and not nested
  \end{alertblock}
\end{frame}

\begin{frame}[fragile]
  \frametitlegb{Control instructions : switch}
  \begin{block}{Syntax}
    \begin{cppcode*}{gobble=2}
      switch(identifier) {
        case c1 : instructions1; break;
        case c2 : instructions2; break;
        case c3 : instructions3; break;
        ...
        default : instructiond; break;
      }
    \end{cppcode*}
    \begin{itemize}
      \item {\it break} is not mandatory but...
      \item cases are entry points, not independant pieces
      \item execution carries on with the next case if no {\it break} is present !
      \item {\it default} may be omitted
    \end{itemize}
  \end{block}
  \pause
  \begin{alertblock}{Use break}
    Do not try to make use of non breaking cases
  \end{alertblock}
\end{frame}

\begin{frame}[fragile]
  \frametitlegb{Control instructions : switch}
  \begin{exampleblock}{Practical example}
    \begin{cppcode*}{}
      enum Lang { FRENCH, GERMAN, ENGLISH, OTHER };
      ...
      switch (language) {
      case FRENCH:
        printf("Bonjour");
        break;
      case GERMAN:
        printf("Guten tag");
        break;
      case ENGLISH:
        printf("Good morning");
        break;
      default:
        printf("I do not talk your langage");
      }
    \end{cppcode*}
  \end{exampleblock}
\end{frame}

\AtBeginEnvironment{minted}{\renewcommand{\fcolorbox}[4][]{#4}}

\begin{frame}[fragile]
  \frametitleit{\texttt{[[fallthrough]]} attribute}
  \begin{alertblock}{\cpp14}
    \begin{cppcode}
      switch (c) {
        case 'a':
          f(); // Warning emitted
        case 'c':
          h();
      }
    \end{cppcode}
  \end{alertblock}
  \begin{exampleblock}{\cpp17}
    \begin{cppcode*}{}
      switch (c) {
        case 'a':
          f();
          [[fallthrough]]; // Warning suppressed
        case 'c':
          h();
      }
    \end{cppcode*}
  \end{exampleblock}
\end{frame}

\begin{frame}[fragile]
  \frametitleit{init-statements for if and switch}
  Allows to simplify if and switch statements
  \begin{alertblock}{\cpp14}
    \begin{cppcode*}{}
      Value val = GetValue();
      if (condition(val)) {
        // on success
      } else {
        // on false...
      }
    \end{cppcode*}
  \end{alertblock}
  \begin{exampleblock}{\cpp17}
    \begin{cppcode*}{}
      if (Value val = GetValue(); condition(val)) {
        // on success
      } else {
        // on false...
      }
    \end{cppcode*}
    \vspace{-.1cm}
    val is visible only inside the if and else statements
  \end{exampleblock}  
\end{frame}

\begin{frame}[fragile]
  \frametitlegb{Control instructions : for loop}
  \begin{block}{for loop syntax}
    \begin{cppcode*}{}
      for(initializations; condition; increments) {
        instructions;
      }
    \end{cppcode*}
    \vspace{-0.2cm}
    \begin{itemize}
      \item initializations and increments are comma separated
      \item initializations can contain declarations
      \item braces are optional if there is a single instruction
    \end{itemize}
  \end{block}
  \pause
  \begin{exampleblock}{Practical example}
    \begin{cppcode*}{}
      for(int i = 0, j = 0 ; i < 10 ; i++, j = i*i) {
        std::cout << i << "^2 is " << j << "\n";
      }
    \end{cppcode*}
  \end{exampleblock}
  \pause
  \begin{alertblock}{Do not abuse the syntax}
    The for statement should fit in 1-3 lines
  \end{alertblock}
\end{frame}

\begin{frame}[fragile]
  \frametitleii{Range based loops}
  \begin{block}{Reason of being}
    \begin{itemize}
    \item simplifies loops tremendously
    \item especially with STL containers
    \end{itemize}
  \end{block}
  \begin{block}{Syntax}
    \begin{cppcode*}{}
      for ( type iteration_variable : container ) {
        // body using iteration_variable
      }
    \end{cppcode*}
  \end{block}
  \begin{exampleblock}{Example code}
    \begin{cppcode*}{}
      int v[4] = {1,2,3,4};
      int sum = 0;
      for (int a : v) { sum += a; }
    \end{cppcode*}
  \end{exampleblock}
\end{frame}

\begin{frame}[fragile]
  \frametitlegb{Control instructions : while loop}
  \begin{block}{while loop syntax}
    \begin{cppcode*}{}
      while(condition) {
        instructions;
      }
      do {
        Instructions;
      } while(condition);
    \end{cppcode*}
    \vspace{-0.3cm}
    \begin{itemize}
      \item braces are optional if there is a single instruction
    \end{itemize}
  \end{block}
  \pause
  \begin{exampleblock}{Practical example}
    \begin{cppcode*}{}
      while (n != 1)
        if (0 == n%2) n /= 2; 
        else n = 3 * n + 1;
    \end{cppcode*}
  \end{exampleblock}
\end{frame}

\begin{frame}[fragile]
  \frametitlegb{Control instructions : commands}
  \begin{block}{control commands}
    \begin{description}
    \item[break] goes out of the loop
    \item[continue] goes immediately to next iteration
    \item[return] goes out of current function
    \end{description}
  \end{block}
  \pause
  \begin{exampleblock}{Practical example}
    \begin{cppcode*}{}
      while (1) {
        if (n == 1) break;
        if (0 == n%2) {
          std::cout << n << "\n";
          n /= 2;
          continue;
        }
        n = 3 * n + 1;
      }
    \end{cppcode*}
  \end{exampleblock}
\end{frame}

\subsection[.h]{Headers and interfaces}

\begin{frame}[fragile]
  \frametitlegb{Headers and interfaces}
  \begin{block}{Interface}
    Set of declarations defining some functionnality
    \begin{itemize}
    \item defined in a ``header file''
    \item no implementation defined
    \end{itemize}
  \end{block}
  \begin{block}{Header : hello.hpp}
    \begin{cppcode*}{linenos=false}
      void printHello();
    \end{cppcode*}
  \end{block}
  \begin{block}{Usage : myfile.cpp}
    \begin{cppcode*}{}
      #include "hello.hpp"
      int main() {
        printHello();
      }
    \end{cppcode*}  
  \end{block}
\end{frame}

\begin{frame}[fragile]
  \frametitlegb{Preprocessor}
  \begin{cppcode}
    // file inclusion
    #include "hello.hpp"
    // macros
    #define MY_GOLDEN_NUMBER 1746
    // compile time decision
    #ifdef USE64BITS
      typedef uint64_t myint;
    #else
      typedef uint32_t myint;
    #endif
  \end{cppcode}
  \pause
  \begin{block}{Use only in very restricted cases}
    \begin{itemize}
    \item include of headers
    \item hardcoded constants before \cpp11
    \item portability necessity
    \end{itemize}
  \end{block}
\end{frame}
\subsection[auto]{Auto keyword}

\begin{frame}[fragile]
  \frametitleii{Auto keyword}
  \begin{block}{Reason of being}
    \begin{itemize}
    \item many type declarations are redundant
    \item and lead to compiler error if you mess up
    \end{itemize}
    \begin{cppcode*}{}
      std::vector<int> v;
      int a = v[3];
      int b = v.size();  // bug ? unsigned to signed
    \end{cppcode*}
  \end{block}
  \pause
  \begin{block}{Practical usage}
    \begin{cppcode*}{}
      std::vector<int> v;
      auto a = v[3];
      auto b = v.size();
      int sum{0};
      for (auto n : v) { sum += n; }
    \end{cppcode*}
  \end{block}
\end{frame}

\begin{frame}[fragile]
  \frametitlegb{Loops, references, auto}
  \begin{alertblock}{Exercise}
    Familiarise yourself with range-based for loops and references
    \begin{itemize}
      \item go to \texttt{code/loopsRefsAuto}
      \item Look at \texttt{loopsRefsAuto.cpp}
      \item Compile it (\texttt{make}) and run the program (\texttt{./loopsRefsAuto})
      \item Work on the tasks that you find in \texttt{loopsRefsAuto.cpp}
    \end{itemize}
  \end{alertblock}
\end{frame}

