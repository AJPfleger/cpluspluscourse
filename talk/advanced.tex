\section{Advanced Topics}

\subsection[Advanced OO]{Advanced Object orientation}

\begin{frame}[fragile,label=current]
  \frametitle{Polymorphism}
  \begin{block}{the concept}
    \begin{itemize}
    \item objects actually have multiple types concurrently
    \item and can be used as any of them
    \end{itemize}
  \end{block}
  \begin{multicols}{2}
    \begin{cppcode*}{gobble=6}
      Polygon *p = new Polygon();

      int f(Drawable *d) {...};
      f(p);  //ok

      try {
        throw *p;
      } catch (Shape e) {
        // will be caught
      }
    \end{cppcode*}
    \columnbreak
    \center
    \begin{tikzpicture}[node distance=1.5cm]
      \classbox{Drawable}{
      }
      \classbox[below of=Drawable]{Shape}{
      }
      \classbox[below of=Shape]{Polygon}{
      }
      \draw[very thick,->] (Polygon) -- (Shape);
      \draw[very thick,->] (Shape) -- (Drawable);
    \end{tikzpicture}
  \end{multicols}
\end{frame}

\begin{frame}[fragile,label=current]
  \frametitle{Method overloading}
  \begin{block}{the problem}
    \begin{itemize}
    \item a given method of the parent can be overloaded in a child
    \item but which one is called ?
    \end{itemize}
  \end{block}
  \begin{multicols}{2}
    \begin{cppcode*}{gobble=6}
      Polygon *p = new Polygon();
      p->draw(); // ?
      
      Shape* s = p;
      s->draw(); // ?
    \end{cppcode*}
    \columnbreak
    \center
    \begin{tikzpicture}[node distance=1.5cm]
      \classbox{Drawable}{
        void draw();
      }
      \classbox[below of=Drawable]{Shape}{
      }
      \classbox[below of=Shape]{Polygon}{
        void draw();
      }
      \draw[very thick,->] (Polygon) -- (Shape);
      \draw[very thick,->] (Shape) -- (Drawable);
    \end{tikzpicture}
  \end{multicols}
\end{frame}


\begin{frame}[fragile,label=current]
  \frametitle{Virtual methods}
  \begin{block}{the concept}
    \begin{itemize}
    \item methods can be declared {\it virtual}
    \item for these, the most precise object is always considered
    \item for others, the type of the variable decides
    \end{itemize}
  \end{block}
  \pause
  \begin{multicols}{2}
    \begin{cppcode*}{gobble=6}
      Polygon *p = new Polygon();
      p->draw(); // Polygon.draw
      
      Shape* s = p;
      s->draw(); // Drawable.draw
    \end{cppcode*}
    \columnbreak
    \center
    \begin{tikzpicture}[node distance=1.5cm]
      \classbox{Drawable}{
        void draw();
      }
      \classbox[below of=Drawable]{Shape}{
      }
      \classbox[below of=Shape]{Polygon}{
        void draw();
      }
      \draw[very thick,->] (Polygon) -- (Shape);
      \draw[very thick,->] (Shape) -- (Drawable);
    \end{tikzpicture}
  \end{multicols}    
\end{frame}

\begin{frame}[fragile,label=current]
  \frametitle{Virtual methods}
  \begin{block}{the concept}
    \begin{itemize}
    \item methods can be declared {\it virtual}
    \item for these, the most precise object is always considered
    \item for others, the type of the variable decides
    \end{itemize}
  \end{block}
  \pause
  \begin{multicols}{2}
    \begin{cppcode*}{gobble=6}
      Polygon *p = new Polygon();
      p->draw(); // Polygon.draw
      
      Shape* s = p;
      s->draw();  // Polygon.draw
    \end{cppcode*}
    \columnbreak
    \center
    \begin{tikzpicture}[node distance=1.5cm]
      \classbox{Drawable}{
        virtual void draw();
      }
      \classbox[below of=Drawable]{Shape}{
      }
      \classbox[below of=Shape]{Polygon}{
        void draw();
      }
      \draw[very thick,->] (Polygon) -- (Shape);
      \draw[very thick,->] (Shape) -- (Drawable);
    \end{tikzpicture}
  \end{multicols}    
\end{frame}

\begin{frame}[fragile,label=current]
  \frametitle{Pure Virtual methods}
  \begin{block}{Concept}
    \begin{itemize}
    \item methods that exist but are not implemented
      \item marked by an ``{\it = 0}'' in the declaration
    \item makes their class abstract
    \item an object can only be instantiated for a non abstract class
    \end{itemize}
  \end{block}
  \pause
  \begin{multicols}{2}
    \begin{cppcode*}{gobble=6}
      // Error : abstract class
      Shape *s = new Shape();

      // ok, draw has been implemented
      Polygon *p = new Polygon();
      
      // Shape type still usable
      Shape* s = p;
      s->draw();
    \end{cppcode*}
    \columnbreak
    \center
    \begin{tikzpicture}[node distance=1.5cm]
      \classbox{Drawable}{
        void draw() = 0;
      }
      \classbox[below of=Drawable]{Shape}{
      }
      \classbox[below of=Shape]{Polygon}{
        void draw();
      }
      \draw[very thick,->] (Polygon) -- (Shape);
      \draw[very thick,->] (Shape) -- (Drawable);
    \end{tikzpicture}
  \end{multicols}
\end{frame}


\begin{frame}[fragile,label=current]
  \frametitle{Multiple Inheritance}
  \begin{block}{Concept}
    \begin{itemize}
    \item one class can inherit from multiple parents
    \end{itemize}
  \end{block}
  \begin{multicols}{2}
    \begin{tikzpicture}[]
      \classbox[]{Polygon}{
      }
      \classbox[below of=Polygon,node distance=1.5cm]{Rectangle}{
      }
      \classbox[right of=Rectangle,node distance=3cm]{Text}{
      }
      \classbox[below right of=Rectangle,node distance=2cm]{TextBox}{
      }
      \draw[very thick,->] (Polygon) -- (Rectangle);
      \draw[very thick,->] (Rectangle) -- (TextBox);
      \draw[very thick,->] (Text) -- (TextBox);
    \end{tikzpicture}
    \columnbreak
    \vspace{2cm}
    \begin{cppcode*}{gobble=6}
      class TextBox :
        public Rectangle, Text {
        // inherits of both
      }
    \end{cppcode*}
  \end{multicols}
\end{frame}

\begin{frame}[fragile,label=current]
  \frametitle{The diamond shape}
  \begin{block}{Concept}
    \begin{itemize}
    \item one class may inherit several times form a given grand parent
    \item are the members of the grand parent replicated ?
    \end{itemize}
  \end{block}
  \hspace{2.5cm}
  \begin{tikzpicture}[]
    \classbox[]{Drawable}{
    }
    \classbox[below left of=Drawable,node distance=2cm]{Rectangle}{
    }
    \classbox[right of=Rectangle,node distance=3cm]{Text}{
    }
    \classbox[below right of=Rectangle,node distance=2cm]{TextBox}{
    }
    \draw[very thick,->] (Drawable) -- (Rectangle);
    \draw[very thick,->] (Drawable) -- (Text);
    \draw[very thick,->] (Rectangle) -- (TextBox);
    \draw[very thick,->] (Text) -- (TextBox);
  \end{tikzpicture}
\end{frame}

\begin{frame}[fragile,label=current]
  \frametitle{Virtual inheritance}
  \begin{block}{Concept}
    \begin{itemize}
    \item inheritance can be {\it virtual} or not
    \item {\it virtual} inheritance will ``share'' parents
    \item standard inheritance will replicate them
    \end{itemize}
  \end{block}
  \begin{multicols}{2}
    \begin{tikzpicture}[]
      \draw node (title) [rectangle] {virtual};
      \classbox[below of=title]{Drawable}{
      }
      \classbox[below left of=Drawable,node distance=2cm]{Rectangle}{
      }
      \classbox[right of=Rectangle,node distance=3cm]{Text}{
      }
      \classbox[below right of=Rectangle,node distance=2cm]{TextBox}{
      }
      \draw[very thick,->] (Drawable) -- (Rectangle);
      \draw[very thick,->] (Drawable) -- (Text);
      \draw[very thick,->] (Rectangle) -- (TextBox);
      \draw[very thick,->] (Text) -- (TextBox);
    \end{tikzpicture}
    \columnbreak
    \begin{tikzpicture}[]
      \classbox[]{Drawable1}{
      }
      \classbox[below of=Drawable1,node distance=1.5cm]{Rectangle}{
      }
      \draw[very thick,->] (Drawable1) -- (Rectangle);
      \classbox[right of=Drawable1,node distance=3cm]{Drawable2}{
      }
      \classbox[below of=Drawable2,node distance=1.5cm]{Text}{
      }
      \draw[very thick,->] (Drawable2) -- (Text);
      \classbox[below right of=Rectangle,node distance=2cm]{TextBox}{
      }
      \draw[very thick,->] (Rectangle) -- (TextBox);
      \draw[very thick,->] (Text) -- (TextBox);
      \draw node at (1.5cm,1cm) [rectangle] {standard};
    \end{tikzpicture}
  \end{multicols}
\end{frame}

\subsubsection{Copy constructors}
\subsubsection{Statics}

\subsection[Functions]{More around functions}
\subsection{Operators}
\subsubsection{Functors}
\subsubsection{Pointers and references}
\subsubsection{Constness}

\subsection{Templates}
\subsubsection[Functions]{Templated functions}
\subsubsection[Classes]{Templated classes}

\subsection[STL]{The Standard Template Library}
\subsubsection[Containers]{Standard containers}
\subsubsection[Customization]{Allocators. comparators and other containers}
\subsubsection{Algorithms}

\subsection[Tools]{Useful tools}
\subsubsection{Compilers}
\subsubsection{Static analyzer}
\subsubsection{Debugger}
\subsubsection{Memory checker}

