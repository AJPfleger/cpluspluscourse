\subsection{More STL}

\begin{frame}[fragile]
  \frametitlecpp[17]{\texttt{std::string\_view}}
  \begin{block}{Non owning view of a continuous char sequence}
    \begin{itemize}
    \item Doesn't allocate memory
    \item Similar interface to \texttt{std::string}
    \item Easy to copy, faster for some calls eg. \texttt{substr} (O(1) vs O(n))
    \item The data pointed has to outlives the \texttt{string\_view}
    \end{itemize}
  \end{block}
  \begin{exampleblock}{Some example uses}
    \begin{cppcode*}{}
      constexpr std::string_view sv {"Some example"};
      auto first = sv.substr(0, sv.find_first_of(" "));
      std::string some_string {"foo bar"};
      std::string_view sv_str(some_string);
      char foo[3] = {'f','o','o'};
      std::string_view carr(foo, std::size(foo));
    \end{cppcode*}
  \end{exampleblock}
\end{frame}

\begin{frame}[fragile]
  \frametitlecpp[17]{std::optional}
  \begin{block}{Manages an optional contained value}
    \begin{itemize}
    \item Contextually converts to bool telling if it contains something
    \item Has value semantics (Copy, Move, Compare, stack alloc.)
    \item Useful for
      \begin{itemize}
        \item the return value of a function that may fail
        \item Useful in place of pointers where value semantics are intuitive
      \end{itemize}
    \end{itemize}
  \end{block}
  \begin{exampleblock}{Code example}
    \small
    \begin{cppcode*}{}
      std::optional<Phone> parse_phone( std::string_view in) {
        if (is_valid_phone(in)) {
          return in;      // equiv. to optional<Phone>(in);
        } else return {}; // default constructs std::nullopt
      }
      auto v = parse_phone(...);
      if (v) {            // alternatively v.is_valid()
        process_phone(v.value()); // *v is equivalent
      }
    \end{cppcode*}
  \end{exampleblock}
\end{frame}

\begin{frame}[fragile]
  \frametitlecpp[17]{std::variant}
  \begin{block}{a type-safe union}
    \begin{itemize}
    \item Allows the variable to hold any of the given types
    \item \texttt{std::get} reads the value of the variant
    \item and throws \texttt{std::bad\_variant\_access} for bad accesses
    \item Makes it easy to implement visitor pattern
    \end{itemize}
  \end{block}
  \begin{exampleblock}{Code example}
    \small
    \begin{cppcode*}{}
      std::variant<int, float, string> opt{100}; // holding int
      int ival = std::get<int>(opt); // or std::get<0>(opt)
      try {
        float val = std::get<float>(opt) // will throw
      } catch (std::bad_variant_access const& ex) {...}

      // Or check the type before accessing it
      if (std::holds_alternative<float>(opt))
          std::cout << std::get<float>(opt);
    \end{cppcode*}
  \end{exampleblock}

\end{frame}

\begin{frame}[fragile]
  \frametitlecpp[17]{std::variant and the visitor pattern}
  \begin{block}{std::visit}
    \begin{itemize}
    \item Applies a ``visitor'' to given variant
    \item A visitor is a callable able to handle the different types
    \end{itemize}
  \end{block}
  \begin{exampleblock}{Practically}
    \small
    \begin{cppcode*}{}
      using option_t = std::variant<int,float,string>;
      struct Visitor {
        void operator() (int i) { std::cout<< "i32:"<< i;}
        void operator() (float f) { std::cout<< "f32:"<< f;}
        void operator() (string s) { std::cout<< "s:"<< s;}
      };
      void print_opt(option_t opt) {
        std::visit(Visitor{}, opt);
        std::cout << "\n";
      }
      print_opt(100); print_opt(3.14f); print_opt("example");
    \end{cppcode*}
  \end{exampleblock}
\end{frame}

\begin{frame}[fragile]
  \frametitlecpp[17]{std::any}
  \begin{block}{a type-safe container for single values of any type}
    \begin{itemize}
    \item Allows a variable to hold any type (say bye to \texttt{void*})
    \item \texttt{std::any\_cast} reads the internal value
    \item and throws \texttt{std::bad\_any\_cast} for bad accesses
    \item \texttt{any\_cast} will only match concrete types, ignoring inheritance
    \end{itemize}
  \end{block}
  \begin{exampleblock}{Code example}
    \small
    \begin{cppcode*}{}
      std::any val{100};          // holding int
      val = std::string("hello"); // holding string
      std::string s = std::any_cast<std::string>(val);
      try {
        int val = std::any_cast<int>(val); // will throw
      } catch (std::bad_any_cast const& ex) {...}
      // Or check the type before accessing it
      if (val.type() == typeid(int))
        std::cout << std::any_cast<int>(val);
    \end{cppcode*}
  \end{exampleblock}
\end{frame}

\begin{frame}[fragile]
  \frametitlecpp[11]{non-member begin and end}
  \begin{alertblock}{The problem in \cpp98}
    STL containers and arrays have different syntax for loop
    \vspace{-1mm}
    \begin{cppcode*}{}
      std::vector<int> v;
      int a[] = {1,2,3};
      for(auto it = v.begin(); it != v.end(); it++) {...}
      for(int i = 0; i < 3; i++) {...}
    \end{cppcode*}
  \end{alertblock}
  \pause
  \begin{block}{A new syntax}
    \begin{cppcode*}{firstnumber=5}
      for(auto it = begin(v); it != end(v); it++) {...}
      for(auto i = begin(a); i != end(a); i++) {...}
    \end{cppcode*}
  \end{block}
  \pause
  \begin{exampleblock}{Allowing the best syntax}
    \begin{cppcode*}{firstnumber=7}
      for(auto & element : v) {...}
      for(auto & element : a) {...}
    \end{cppcode*}
  \end{exampleblock}
\end{frame}

\begin{frame}[fragile]
  \frametitlecpp[17]{Structured Binding Declarations}
  Helps when using tuples as a return type.\\
  Automatically creates variables and ties them.
  \begin{alertblock}{\cpp14}
    \begin{cppcode*}{}
      void foo(std::tuple<int, double, long> tuple) {
        int a = 0;
        double b = 0.0;
        long c = 0;
        // a, b, c need to be declared first
        std::tie(a, b, c) = tuple;
    \end{cppcode*}
  \end{alertblock}
  \begin{exampleblock}{\cpp17}
    \begin{cppcode*}{firstnumber=7}
      void foo(std::tuple<int, double, long> tuple) {
        auto [ a, b, c ] = tuple; ...
      }
      for (const auto& [key, value] : map) { ... }
    \end{cppcode*}
  \end{exampleblock}
\end{frame}

\begin{frame}[fragile]
  \frametitlecpp[17]{compile-time branches}
  \begin{block}{{\it if constexpr}}
    \begin{itemize}
    \item takes a \texttt{constexpr} expression as condition
    \item evaluates at compile time
    \item key benefit : the discarded branch can contain invalid code
    \end{itemize}
  \end{block}
  \begin{exampleblock}{Example code}
    \small
    \begin{cppcode*}{}
      template <typename T>
      auto remove_ptr(T t) {
        if constexpr (std::is_pointer_v<T>) {
          return *t;
        } else {
          return t;
        }
      }
      int i = ...; int *j = ...;
      int r = remove_ptr(i);  // equivalent to i
      int q = remove_ptr(j);  // equivalent to *j
    \end{cppcode*}
  \end{exampleblock}
\end{frame}
