\subsection[except]{Exceptions}

\begin{frame}[fragile]
  \frametitlecpp[98]{Exceptions}
  \begin{block}{The concept}
    \begin{itemize}
      \item to handle \textit{exceptional} events that happen rarely
      \item and cleanly jump to a place where the error can be handled
    \end{itemize}
  \end{block}
  \begin{block}{In practice}
    \begin{itemize}
      \item add an exception handling block with \mintinline{cpp}{try} ... \mintinline{cpp}{catch}
      \begin{itemize}
        \item when exceptions are possible \textit{and can be handled}
      \end{itemize}
      \item throw an exception using \mintinline{cpp}{throw}
      \begin{itemize}
        \item when a function cannot proceed or recover internally
      \end{itemize}
    \end{itemize}
  \end{block}
  \begin{multicols}{2}
    \begin{cppcode*}{fontsize=\small,gobble=2}
      try {
        process_data(f);
      } catch (const
        std::out_of_range& e) {
        std::cerr << e.what();
      }
    \end{cppcode*}
    \columnbreak
    \begin{cppcode*}{fontsize=\small,gobble=2}
      void process_data(file &f) {
        ...
        if (i >= buffer.size())
          throw std::out_of_range{
            "buf overflow"};
      }
    \end{cppcode*}
  \end{multicols}
\end{frame}

\begin{frame}[fragile]
  \frametitlecpp[98]{Exceptions}
  \begin{block}{Throwing exceptions}
    \begin{itemize}
      \item objects of any type can be thrown (even e.g.\ \mintinline{cpp}{int})
      \begin{itemize}
        \item prefer standard exception classes
      \end{itemize}
      \item throw objects by value
    \end{itemize}
  \end{block}
  \begin{cppcode}
    void process_data(file& f) {
      if (!f.open())
        throw std::invalid_argument{"stream is not open"};

      auto header = read_line(f); // may throw an IO error
      if (!header.starts_with("BEGIN"))
        throw std::runtime_error{"invalid file content"};

      std::string body(f.size()); // may throw std::bad_alloc
      ...
    }
  \end{cppcode}
\end{frame}

\begin{frame}[fragile]
  \frametitlecpp[98]{Exceptions}
  \begin{block}{Catching exceptions}
    \begin{itemize}
      \item a catch clause catches an exception of the same or derived type
      \item multiple catch clauses will be matched in order
      \item if no catch clause matches, the exception propagates
      \item if the exception is never caught, \mintinline{cpp}{std::terminate} is called
    \end{itemize}
  \end{block}
  \begin{cppcode}
    try {
      process_data(f);
    } catch (const std::invalid_argument& e) {
      bad_files.push_back(f);
    } catch (const std::exception& e) {
      std::cerr << "Failed to process file: " << e.what();
    }
  \end{cppcode}
\end{frame}

\begin{frame}[fragile]
  \frametitlecpp[98]{Exceptions}
  \begin{block}{Rethrowing exceptions}
    \begin{itemize}
      \item a caught exception can be rethrown
      \item useful when we want to act on an error, but cannot handle and want to propagate it
    \end{itemize}
  \end{block}
  \begin{cppcode}
    try {
      process_data(f);
    } catch (const std::bad_alloc& e) {
      std::cerr << "Insufficient memory for " << f.name();
      throw; // rethrow
    }
  \end{cppcode}
\end{frame}

\begin{frame}[fragile]
  \frametitlecpp[98]{Exceptions}
  \begin{block}{Catching everything}
    \begin{itemize}
    \item sometimes we need to catch all possible exceptions
    \item e.g.\ in \mintinline{cpp}{main}, a thread, a destructor, interfacing with C, \ldots
    \end{itemize}
  \end{block}
  \begin{cppcode}

    try {
      callUnknownFramework();
    } catch(const std::exception& e) {
      // catches std::exception and all derived types
      std::cerr << "Exception: " << e.what() << std::endl;
    } catch(...) {
      // catches everything else
      std::cerr << "Unknown exception type" << std::endl;
    }
  \end{cppcode}
\end{frame}

\begin{frame}[fragile]
  \frametitlecpp[98]{Exceptions}
  \begin{block}{Stack unwinding}
    \begin{itemize}
      \item all objects on the stack between a \mintinline{cpp}{throw} and the matching \mintinline{cpp}{catch} are destructed automatically
      \item this should cleanly release intermediate resources
      \item make sure you are using the RAII idiom for your own classes
    \end{itemize}
  \end{block}
  \begin{multicols}{2}
    \begin{cppcode*}{gobble=2}
      class C { ... };
      void f() {
        C c1;
        throw exception{};
          // start unwinding
        C c2; // not run
      }
      void g() {
        C c3; f();
      }
    \end{cppcode*}
    \columnbreak
    \begin{cppcode*}{gobble=2,firstnumber=11}
      int main() {
        try {
          C c5;
          g();
          cout << "done"; // not run
        } catch(const exception&) {
          // c1, c3 and c5 have been
          // destructed
        }
      }
    \end{cppcode*}
  \end{multicols}
\end{frame}

\begin{frame}[fragile]
  \frametitlecpp[98]{Exceptions}
  \begin{block}{Standard exceptions}
    \begin{itemize}
      \item \mintinline{cpp}{std::exception}, defined in header \mintinline{cpp}{<exception>}
      \begin{itemize}
        \item Base class of all standard exceptions
        \item Get error message: \mintinline{cpp}{virtual const char* what() const;}
        \item Please derive your own exception classes from this one
      \end{itemize}
      \item From \mintinline{cpp}{<stdexcept>}:
      \begin{itemize}
        \item \mintinline{cpp}{std::runtime_error}, \mintinline{cpp}{std::logic_error}, \mintinline{cpp}{std::out_of_range}, \mintinline{cpp}{std::invalid_argument}, ...
        \item Store a string: \mintinline{cpp}{throw std::runtime_error{"msg"}}
        \item You should use these the most
      \end{itemize}
      \item \mintinline{cpp}{std::bad_alloc}, defined in header \mintinline{cpp}{<new>}
      \begin{itemize}
        \item Thrown by standard allocation functions (e.g.\ \mintinline{cpp}{new})
        \item Signals failure to allocate
        \item Carries no message
      \end{itemize}
      \item ...
    \end{itemize}
  \end{block}
\end{frame}

\begin{frame}[fragile]
  \frametitlecpp[17]{Exceptions}
  \begin{goodpractice}{Exceptions}
    \begin{itemize}
      \item throw exceptions by value, catch them by (const) reference
      \item use exceptions for \textit{unlikely} runtime errors outside the program's control
      \begin{itemize}
        \item bad inputs, files unexpectedly not found, DB connection, \ldots
      \end{itemize}
      \item \textit{don't} use exceptions for logic errors in your code
      \begin{itemize}
        \item consider \mintinline{cpp}{assert} and tests
      \end{itemize}
      \item \textit{don't} use exceptions to provide alternative/skip return values
      \begin{itemize}
        \item you can use \mintinline{cpp}{std::optional} or \mintinline{cpp}{std::variant}
        \item avoid using the global C-style \mintinline{cpp}{errno}
      \end{itemize}
      \item never throw in destructors
      \item see also the \href{https://isocpp.github.io/CppCoreGuidelines/CppCoreGuidelines#S-errors}{\cpp core guidelines} and the \href{https://isocpp.org/wiki/faq/exceptions}{ISO \cpp FAQ}
    \end{itemize}
  \end{goodpractice}
\end{frame}

\begin{frame}[fragile]
  \frametitlecpp[98]{Exceptions}
  \begin{block}{A more illustrative example}
    \begin{itemize}
      \item exceptions are very powerful when there is much code between the error and where the error is handled
      \item they can also rather cleanly handle different types of errors
      \item \mintinline{cpp}{try}/\mintinline{cpp}{catch} statements can also be nested
    \end{itemize}
  \end{block}
  \begin{multicols}{2}
    \begin{cppcode*}{fontsize=\scriptsize,gobble=2}
      try {
        for (File const &f : files) {
          try {
            process_file(f);
          }
          catch (bad_file const & e) {
            ... // loop continues
          }
        }
      } catch (bad_db const & e) {
        ... // loop aborted
      }
    \end{cppcode*}
    \columnbreak
    \begin{cppcode*}{fontsize=\scriptsize,gobble=2}
      void process_file(File const & file) {
        ...
        if (handle = open_file(file))
          throw bad_file(file.status());
        while (!handle) {
          line = read_line(handle);
          database.insert(line); // can throw
                                 // bad_db
        }
      }
    \end{cppcode*}
  \end{multicols}
\end{frame}

\begin{frame}[fragile]
  \frametitlecpp[98]{Exceptions}
  \begin{block}{Cost}
    \begin{itemize}
      \item exceptions have little cost if no exception is thrown
      \begin{itemize}
        \item they are recommended to report \textit{exceptional} errors
      \end{itemize}
      \item for performance, when error raising and handling are close, or errors occur often, prefer error codes or a dedicated class
      \item when in doubt about which error strategy is better, profile!
   \end{itemize}
  \end{block}
  \begin{multicols}{2}
    \begin{minipage}{5cm}
      \begin{alertblock}{Avoid}
        \begin{cppcode*}{fontsize=\scriptsize,gobble=6,linenos=false}
          for (string const &num: nums) {
            try {
              int i = convert(num); // can
                                    // throw
              process(i);
            } catch (not_an_int const &e) {
              ... // log and continue
            }
          }
        \end{cppcode*}
      \end{alertblock}
    \end{minipage}
    \columnbreak
    \begin{minipage}{5cm}
      \begin{exampleblock}{Prefer}
        \begin{cppcode*}{fontsize=\scriptsize,gobble=6,linenos=false}
          for (string const &num: nums) {
            optional<int> i = convert(num);
            if (i) {
              process(*i);
            } else {
              ... // log and continue
            }
          }
        \end{cppcode*}
      \end{exampleblock}
    \end{minipage}
  \end{multicols}
\end{frame}

\begin{frame}[fragile]
  \frametitlecpp[11]{noexcept specifier}
  \begin{block}{}
    \begin{itemize}
      \item a function with the \mintinline{cpp}{noexcept} specifier states that it guarantees to not throw an exception
      \begin{cppcode*}{gobble=2,linenos=false}
        int f() noexcept;
      \end{cppcode*}
      \begin{itemize}
        \item either no exceptions is thrown or they are handled internally
        \item checked at compile time
        \item so allows the compiler to optimise around that knowledge
      \end{itemize}
      \item a function with \mintinline{cpp}{noexcept(expression)} is only \mintinline{cpp}{noexcept} when \mintinline{cpp}{expression} evaluates to \mintinline{cpp}{true} at compile-time
      \begin{cppcode*}{gobble=2,linenos=false}
        int safe_if_8B() noexcept(sizeof(long)==8);
      \end{cppcode*}
      \item Use \mintinline{cpp}{noexcept} on leaf functions where you know the behaviour
      \item C++11 destructors are \mintinline{cpp}{noexcept} - never throw from them
    \end{itemize}
  \end{block}
\end{frame}

\begin{advanced}
\begin{frame}[fragile]
  \frametitlecpp[11]{noexcept operator}
  \begin{block}{}
    \begin{itemize}
      \item the \mintinline{cpp}{noexcept(expression)} operator checks at compile-time whether an expression can throw exceptions
      \item returns a \mintinline{cpp}{bool}, which is \mintinline{cpp}{true} if no exceptions can be thrown
    \end{itemize}
  \end{block}
  \begin{block}{}
    \begin{cppcode*}{gobble=2}
      constexpr bool callCannotThrow = noexcept(f());
      if constexpr (callCannotThrow) { ... }
    \end{cppcode*}
  \end{block}
  \begin{block}{}
    \begin{cppcode*}{gobble=2,firstnumber=3}
      template <typename Function>
      void g(Function f) noexcept(noexcept(f())) {
        ...
        f();
      }
    \end{cppcode*}
  \end{block}
\end{frame}

\begin{frame}[fragile]
  \frametitlecpp[98]{Exceptions}
  \begin{block}{Exception Safety Guarantees}
    A function can offer the following guarantees:
    \begin{itemize}
      \item No guarantee
      \begin{itemize}
        \item An exception will lead to undefined program state
      \end{itemize}
      \item Basic exception safety guarantee
      \begin{itemize}
        \item An exception will leave the program in a defined state
        \begin{itemize}
          \item At least destructors must be able to run
          \item This is similar to the moved from state
        \end{itemize}
        \item No resources are leaked
      \end{itemize}
      \item Strong exception safety guarantee
      \begin{itemize}
        \item The operation either succeeds or has no effect
        \item When an exception occurs, the original state must be restored
        \item This is hard to do, possibly expensive, maybe impossible
      \end{itemize}
      \item No-throw guarantee
      \begin{itemize}
        \item The function never throws (e.g.\ is marked \mintinline{cpp}{noexcept})
        \item Errors are handled internally or lead to termination
      \end{itemize}
    \end{itemize}
  \end{block}
\end{frame}

\begin{frame}[fragile]
  \frametitle{Exceptions}
  \begin{block}{Alternatives to exceptions}
    \begin{itemize}
      \item The global variable \mintinline{cpp}{errno} (avoid)
      \item Return an error code (e.g.\ an \mintinline{cpp}{int} or an \mintinline{cpp}{enum})
      \item Return a \mintinline{cpp}{std::error_code} (\cpp11)
      \item If failing to produce a value is not a hard error, consider returning \mintinline{cpp}{std::optional<T>} (\cpp17)
      \item Return \mintinline{cpp}{std::expected<T, E>} (\cpp23), where \mintinline{cpp}{T} is the return type on success and \mintinline{cpp}{E} is the type on failure
      \item Terminate the program
        \begin{itemize}
        \item e.g.\ by calling \mintinline{cpp}{std::terminate()} or \mintinline{cpp}{std::abort()}
        \end{itemize}
      \item Contracts: Specify function pre- and postconditions. Planned for \cpp20, but removed. Will come back in the future
    \end{itemize}
  \end{block}
\end{frame}

\begin{frame}[fragile]
  \frametitle{Further resources}
  \begin{block}{}
    \begin{itemize}
      \item \href{https://isocpp.org/wiki/faq/exceptions}{Exceptions and Error Handling} - ISO C++ FAQ
      \item \href{https://www.youtube.com/watch?v=0ojB8c0xUd8&t=1505s}{Back to Basics: Exceptions - Klaus Iglberger - CppCon 2020}
      \item \href{https://www.youtube.com/watch?v=fOV7I-nmVXw}{The Unexceptional Exceptions - Fedor Pikus - CppCon 2015}
      \item \href{https://www.youtube.com/watch?v=ZUH8p1EQswA}{The Dawn of a New Error, (C++ error-handling and exceptions) - Phil Nash - CppCon 2019} (alternatives)
      \item \href{https://www.youtube.com/watch?v=mkkaAWNE-Ig}{Exceptions the Other Way Round - Sean Parent - CppNow 2022} (theoretical approach to errors)
      \item \href{https://wg21.link/P0709}{P0709}: Zero-overhead deterministic exceptions: Throwing values - Herb Sutter (proposal - distant future)
      \begin{itemize}
        \item \href{https://www.youtube.com/watch?v=ARYP83yNAWk}{De-fragmenting C++: Making Exceptions and RTTI More Affordable and Usable - Herb Sutter CppCon 2019}
      \end{itemize}
    \end{itemize}
  \end{block}
\end{frame}
\end{advanced}
