\subsection[Core]{Core syntax and types}

\begin{frame}[fragile]
  \frametitlecpp[98]{Hello World}
  \begin{cppcode}
    #include <iostream>

    // This is a function
    void print(int i) {
      std::cout << "Hello, world " << i << std::endl;
    }

    int main(int argc, char** argv) {
      int n = 3;
      for (int i = 0; i < n; i++) {
        print(i);
      }
      return 0;
    }
  \end{cppcode}
\end{frame}

\begin{frame}[fragile]
  \frametitlecpp[98]{Comments}
  \begin{cppcode}
    // simple comment for integer declaration
    int i;

    /* multiline comment
     * in case we need to say more
     */
    double d;

    /**
     * Best choice : doxygen compatible comments
     * \fn bool isOdd(int i)
     * \brief checks whether i is odd
     * \param i input
     * \return true if i is odd, otherwise false
     */
    bool isOdd(int i);
  \end{cppcode}
\end{frame}

\begin{frame}[fragile]
  \frametitlecpp[98]{Basic types(1)}
  \begin{cppcode}
    bool b = true;            // boolean, true or false

    char c = 'a';             // min 8 bit integer (signed or not)
                              // can store an ASCII character
    signed char c = 4;        // min 8 bit signed integer
    unsigned char c = 4;      // min 8 bit unsigned integer

    char* s = "a C string";   // array of chars ended by \0
    string t = "a C++ string";// class provided by the STL

    short int s = -444;       // min 16 bit signed integer
    unsigned short s = 444;   // min 16 bit unsigned integer
    short s = -444;           // int is optional
  \end{cppcode}
\end{frame}
\begin{frame}[fragile]
  \frametitlecpp[98]{Basic types(2)}
  \begin{cppcode}
    int i = -123456;          // min 16, usually 32 bit
    unsigned int i = 1234567; // min 16, usually 32 bit

    long l = 0L               // min 32 bit
    unsigned long l = 0UL;    // min 32 bit

    long long ll = 0LL;          // min 64 bit
    unsigned long long l = 0ULL; // min 64 bit

    float f = 1.23f;          // 32 (23+7+1) bit float
    double d = 1.23E34;       // 64 (52+11+1) bit float
    long double ld = 1.23E34L // min 64 bit float
  \end{cppcode}
\end{frame}

\begin{frame}[fragile]
  \frametitlecpp[98]{Portable numeric types}
  \alert{Requires inclusion of a specific header}
  \begin{cppcode}
    #include <cstdint>

    int8_t c = -3;     // 8 bit signed integer
    uint8_t c = 4;     // 8 bit unsigned integer

    int16_t s = -444;  // 16 bit signed integer
    uint16_t s = 444;  // 16 bit unsigned integer

    int32_t s = -0674; // 32 bit signed integer
    uint32_t s = 0674; // 32 bit unsigned integer

    int64_t s = -0x1bc;// 64 bit signed integer
    uint64_t s = 0x1bc;// 64 bit unsigned int
    \end{cppcode}
\end{frame}

\begin{frame}[fragile]
  \frametitlecpp[98]{Useful aliases}
    \alert{Requires inclusion of headers}
  \begin{cppcode}
    #include <cstddef> // and many other headers

    size_t s = sizeof(int); // unsigned integer
                            // can hold any variable's size

    #include <cstdint>

    ptrdiff_t c = &s - &s;  // signed integer, can hold any
                            // diff between two pointers

    // int, which can hold any pointer value:
    intptr_t i = reinterpret_cast<intptr_t>(&s);   // signed
    uintptr_t i = reinterpret_cast<uintptr_t>(&s); // unsigned
    \end{cppcode}
\end{frame}
