\subsection[$f()$]{Functions}

\begin{frame}[fragile]
  \frametitlecpp[98]{Functions}
  \begin{multicols}{2}
    \begin{cppcode*}{gobble=2}
      // with return type
      int square(int a) {
        return a * a;
      }

      // multiple parameters
      int mult(int a,
               int b) {
        return a * b;
      }
    \end{cppcode*}
    \columnbreak
    \begin{cppcode*}{gobble=2,firstnumber=11}
      // no return
      void log(char* msg) {
        std::cout << msg;
      }

      // no parameter
      void hello() {
        std::cout << "Hello World";
      }
    \end{cppcode*}
  \end{multicols}
\end{frame}

\begin{frame}[fragile]
  \frametitlecpp[98]{Function default arguments}
  \begin{multicols}{2}
    \begin{cppcode*}{gobble=2}
      // must be the trailing
      // argument
      int add(int a,
              int b = 2) {
        return a + b;
      }
      // add(1) == 3
      // add(3,4) == 7

    \end{cppcode*}
    \columnbreak
    \begin{cppcode*}{gobble=2,firstnumber=11}
      // multiple default
      // arguments are possible
      int add(int a = 2,
              int b = 2) {
        return a + b;
      }
      // add() == 4
      // add(3) == 5
    \end{cppcode*}
  \end{multicols}
\end{frame}


\Scontents*[store-cmd=code_bigStruct]{
struct BigStruct {...};
BigStruct s;

// parameter by value
void printBS(BigStruct p) {
  ...
}
printBS(s); // copy

// parameter by reference
void printBSp(BigStruct &q) {
  ...
}
printBSp(s); // no copy
}
\begin{frame}[fragile]
  \frametitlecpp[98]{Functions: parameters are passed by value}
  \begin{multicols}{2}
    \begin{overprint}[\columnwidth]
      \onslide<1-2>
      \highlightCppCode{2}{code_bigStruct}
      \onslide<3>
      \highlightCppCode{5,8}{code_bigStruct}
      \onslide<4->
      \highlightCppCode{11,14}{code_bigStruct}
    \end{overprint}
    \columnbreak
    \null \vfill
    \begin{tikzpicture}
      \memorystack[word size=1, nb blocks=7, size x=3cm]
      \onslide<2-> {
        \memorypush{s1}
        \memorypush{...}
        \memorypush{sn}
        \memorystruct{1}{3}{s}
      }
      \onslide<3> {
        \memorypush{p1 = s1}
        \memorypush{...}
        \memorypush{pn = sn}
        \memorystruct{4}{6}{p}
      }
      \memorygoto{4}
      \onslide<4> {
        \memorypushpointer[q =]{1}
      }
    \end{tikzpicture}
    \vfill \null
  \end{multicols}
\end{frame}

\Scontents*[store-cmd=code_smallStruct]{
struct SmallStruct {int a;};
SmallStruct s = {1};

void changeSS(SmallStruct p) {
  p.a = 2;
}
changeSS(s);
// s.a == 1

void changeSS2(SmallStruct &q) {
  q.a = 2;
}
changeSS2(s);
// s.a == 2
}
\begin{frame}[fragile]
  \frametitlecpp[98]{Functions: pass by value or reference?}
  \begin{multicols}{2}
    \begin{overprint}[\columnwidth]
      \onslide<1>
      \highlightCppCode{}{code_smallStruct}
      \onslide<2>
      \highlightCppCode{2}{code_smallStruct}
      \onslide<3>
      \highlightCppCode{4,7}{code_smallStruct}
      \onslide<4>
      \highlightCppCode{5}{code_smallStruct}
      \onslide<5>
      \highlightCppCode{8}{code_smallStruct}
      \onslide<6>
      \highlightCppCode{10,13}{code_smallStruct}
      \onslide<7>
      \highlightCppCode{11}{code_smallStruct}
      \onslide<8>
      \highlightCppCode{14}{code_smallStruct}
    \end{overprint}
    \columnbreak
    \null \vfill
    \begin{tikzpicture}
      \memorystack[word size=1, block size=4, nb blocks=3, size x=3cm]
      \onslide<2-6> {
        \memorypush{s.a = 1}
      }
      \memorygoto{1}
      \onslide<7-> {
        \memorypush{s.a = 2}
      }

      \memorygoto{2}
      \onslide<3> {
        \memorypush{p.a = 1}
      }
      \memorygoto{2}
      \onslide<4> {
        \memorypush{p.a = 2}
      }
      \memorygoto{2}
      \onslide<6-7> {
        \memorypushpointer[q =]{1}
      }
      \end{tikzpicture}
    \vfill \null
  \end{multicols}
\end{frame}

\begin{frame}[fragile]
  \frametitlecpp[98]{Pass by value, reference or pointer}
  \begin{block}{Different ways to pass arguments to a function}
    \begin{itemize}
    \item by default, arguments are passed by value (= copy) \\
          good for small types, e.g.\ numbers
    \item prefer references for mandatory parameters to avoid copies
    \item use pointers for optional parameters to allow \mintinline{cpp}{nullptr}
    \item use \mintinline{cpp}{const} for safety and readability whenever possible
    \end{itemize}
  \end{block}
  \pause
  \begin{block}{Syntax}
    \begin{cppcode*}{escapeinside=||}
struct T {...}; T a;
void f(T value);           f(a);      // by value
void fRef(const T &value); fRef(a);   // by reference
void fPtr(const T *value); fPtr(|{\setlength{\fboxsep}{0pt}\color{gray}\colorbox{yellow}{\textsc{&}}}|a);  // by pointer
void fWrite(T &value);     fWrite(a); // non-const ref
    \end{cppcode*}
  \end{block}
\end{frame}

\begin{frame}[fragile]
  \frametitlecpp[98]{Functions}
  \begin{alertblock}{Exercise}
    Familiarise yourself with pass by value / pass by reference.
    \begin{itemize}
      \item go to \texttt{code/functions}
      \item Look at \texttt{functions.cpp}
      \item Compile it (\texttt{make}) and run the program (\texttt{./functions})
      \item Work on the tasks that you find in \texttt{functions.cpp}
    \end{itemize}
  \end{alertblock}
\end{frame}

\begin{frame}[fragile]
  \frametitlecpp[98]{Functions: good practices}
  \begin{onlyenv}<1>
    \begin{block}{Ensure good readability/maintainability:}
      \begin{itemize}
        \item Keep functions short
        \item Do one logical thing (single-responsibility principle)
        \item Use expressive names
        \item Document non-trivial functions
      \end{itemize}
    \end{block}
    \begin{exampleblock}{Example: Good}
      \begin{cppcode*}{gobble=2}
        /// Count number of dilepton events in data.
        /// \param d Dataset to search.
        unsigned int countDileptons(Data d) {
          selectEventsWithMuons(d);
          selectEventsWithElectrons(d);
          return d.size();
        }
      \end{cppcode*}
    \end{exampleblock}
  \end{onlyenv}
  \begin{onlyenv}<2->
    \begin{alertblock}{Example: don't! Everything in one long function}
      \begin{multicols}{2}
        \begin{cppcode*}{gobble=6}
          unsigned int runJob() {
            // Step 1: data
            Data data;
            data.resize(123456);
            data.fill(...);

            // Step 2: muons
            for (....) {
              if (...) {
                data.erase(...);
              }
            }
            // Step 3: electrons
            for (....) {
        \end{cppcode*}
        \columnbreak
        \begin{cppcode*}{gobble=6,firstnumber=last}
              if (...) {
                data.erase(...);
              }
            }

            // Step 4: dileptons
            int counter = 0;
            for (....) {
              if (...) {
                counter++;
              }
            }

            return counter;
          }
        \end{cppcode*}
      \end{multicols}
    \end{alertblock}
  \end{onlyenv}
\end{frame}
