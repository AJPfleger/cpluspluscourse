\subsection[assert]{Assertions}

\begin{frame}[fragile]
  \frametitlecpp[98]{Assertions}
  \begin{block}{Checking invariants in a program}
    \begin{itemize}
      \item An invariant is a property that is guaranteed to be true during certain phases of a program, and the program might crash or yield wrong results if it is violated
      \begin{itemize}
        \item ``Here, `a' should always be positive''
      \end{itemize}
      \item This can be checked using \cppinline{assert}
      \item The program will be aborted if the assertion fails
    \end{itemize}
  \end{block}
  \begin{exampleblockGB}{\texttt{assert} in practice}{https://godbolt.org/z/eo78cY7hM}{\texttt{assert}}
    \begin{overprint}
    \onslide<1>
    \begin{cppcode*}{}
      #include <cassert>
      double f(double a) {
        // [...] do stuff with a
        // [...] that should leave it positive
        assert(a > 0.);
        return std::sqrt(a);
      }
    \end{cppcode*}
    \onslide<2>
    \begin{Verbatim}
% ./testAssert
Assertion failed: (a > 0.), function f,
                  file testAssert.cpp, line 5.
Abort trap: 6
    \end{Verbatim}
    \end{overprint}
  \end{exampleblockGB}
\end{frame}

\begin{frame}[fragile]
  \frametitlecpp[98]{Assertions}
  \begin{goodpractice}{Assert}
    \begin{itemize}
      \item Assertions are mostly for developers and debugging
      \item Use them to check important invariants of your program
      \item Prefer handling user-facing errors with helpful error messages/exceptions
      \item Assertions can impact the speed of a program
      \begin{itemize}
        \item Assertions are disabled when the macro \cppinline{NDEBUG} is defined
        \item Decide if you want to disable them when you release code
      \end{itemize}
    \end{itemize}
  \end{goodpractice}
  \begin{exampleblock}{Disabling assertions}
    \small
    Compile a program with NDEBUG defined:\\
    \texttt{g++ \textcolor{blue}{-DNDEBUG} -O2 -W[...] test.cpp -o test.exe}

    \begin{cppcode*}{}
      double f(double a) {
        assert(a > 0.); // no effect
        return std::sqrt(a);
      }
    \end{cppcode*}
  \end{exampleblock}
\end{frame}

\begin{frame}[fragile]
  \frametitlecpp[11]{Static Assert}
  \begin{block}{Checking invariants at compile time}
    \begin{itemize}
      \item To check invariants at compile time, use \cppinline{static_assert}
      \item The assertion can be any constant expression (see later)
      \item The message argument is optional in \cpp 17 and later
    \end{itemize}
  \end{block}
  \begin{exampleblock}{\texttt{static\_assert}}
    \small
    \begin{cppcode*}{}
      double f(UserType a) {
        static_assert(
          std::is_floating_point<UserType>::value,
          "This function expects floating-point types.");
        return std::sqrt(a);
      }
    \end{cppcode*}
  \end{exampleblock}
  \scriptsize
  \begin{Verbatim}[commandchars=\\\{\}]
a.cpp: In function 'double f(UserType)':
\textcolor{blue}{a.cpp:3:9:} \textcolor{red}{error:} \textcolor{blue}{static assertion failed: This function}
                                           \textcolor{blue}{expects floating-point types.}
   2 | static_assert(
     |   \textcolor{red}{std::is_floating_point<UserType>::value},
     |   \textcolor{red}{~~~~~^~~~~~~~~~~~~~~~~~~~~~~~~~~~~~~~~~}
  \end{Verbatim}
\end{frame}
