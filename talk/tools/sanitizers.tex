\subsection[sani]{Sanitizers}

\begin{frame}[fragile]
  \frametitle{Address Sanitizer (ASan)}
  \begin{block}{ASan introduction}
    \begin{itemize}
    \item Compiler instrumentation
    \item Program stops on invalid memory access, e.g.\
      \begin{itemize}
        \item Invalid read/write on heap and stack
        \item Double free/delete, use after free
        \item Buffer overflow on stack (few tools can do this)
        \item Only linux: memory leaks
      \end{itemize}
    \end{itemize}
  \end{block}
  \pause
  \begin{exampleblock}{Usage (gcc/clang syntax)}
    \begin{itemize}
    \item Compile with \mintinline{bash}{-fsanitize=address -fno-omit-frame-pointer -g}
    \item With clang, add optionally: \texttt{-fsanitize-address-use-after-return=always -fsanitize-address-use-after-scope}
    \item Link with \mintinline{bash}{-fsanitize=address}
    \item Run the program
    \end{itemize}
  \end{exampleblock}
\end{frame}

\begin{frame}[fragile]
  \frametitle{Address Sanitizer (ASan)}
  \begin{block}{How it works}
    \begin{itemize}
      \item Compiler adds run-time checks ($\sim$2x slow down)
      \item \mintinline{cpp}{IsPoisoned(address)} looks up state of address in asan's ``shadow memory''
      \item Shadow memory: memory where 1 shadow byte tracks state of 8 application bytes (state = accessible, poisoned, \ldots)
      \item Functions that deal with memory (\mintinline{cpp}{new() / delete()} / strings / ...) update entries in shadow memory when called
    \end{itemize}
  \end{block}
  \begin{exampleblock}{asan instrumentation (mock code)}
    \begin{overprint}
      \onslide<1>
      \vfill
      \begin{cppcode*}{gobble=4}
        int i = *address;
      \end{cppcode*}
      \onslide<2->
      \vfill
      \begin{cppcode*}{gobble=4}
        if (IsPoisoned(address)) {
          ReportError(address, kAccessSize, kIsWrite);
        }
        int i = *address;
      \end{cppcode*}
    \end{overprint}
  \end{exampleblock}
\end{frame}


\begin{frame}[fragile]
  \begin{block}{ASan red zones}
    \begin{itemize}
      \item If adjacent data blocks are owned by the process, the operating system will allow an access
      \item<2> ASan surrounds blocks of memory by poisoned red zones
      \item<2> Program stops when accessing a red zone
    \end{itemize}
  \end{block}
  \begin{exampleblock}{Illegal access (not detected without ASan)}
    \begin{multicols}{2}
      \begin{overprint}
        \onslide<1>
        \begin{minted}{cpp}
 void foo() {
   char a[8];
   char b[8];
   a[8] = '1';
 }
        \end{minted}
        \onslide<2>
        \begin{minted}{diff}
 void foo() {
+  char redzone1[32];
   char a[8];
+  char redzone2[24];
   char b[8];
+  char redzone3[24];
+  // <poison redzones>
   a[8] = '1';
+  // <unpoison redzones>
 }
        \end{minted}
      \end{overprint}
      \columnbreak
      \begin{tikzpicture}
        \clip (0,0) rectangle (6cm, 3cm);
        \memorystack[word size=4,nb blocks=4]
        \onslide<1>{
          \draw[fill=green!70,opacity=0.5] (0.,0.*\stacksizey) rectangle (\stacksizex/4.,1.*\stacksizey) node[midway]{\footnotesize a[0-7]};
          \draw[fill=orange,opacity=0.5] (\stacksizex/4.,0.*\stacksizey) rectangle (\stacksizex/2.,1.*\stacksizey) node[midway]{\footnotesize b[0-7]};
        }
        \memorygoto{2}
        \onslide<2->{
          \draw[fill=red!70,opacity=0.5] (0.,0.*\stacksizey) rectangle (\stacksizex,1.*\stacksizey) node[midway]{redzone1};
          \memorypush{a[0-7]}
          \draw[fill=red!70,opacity=0.5] (0.+\stacksizex/4.,1.*\stacksizey) rectangle (\stacksizex,2.*\stacksizey) node[midway]{redzone2};
          \memorypush{b[0-7]}
          \draw[fill=red!70,opacity=0.5] (0.+\stacksizex/4.,2.*\stacksizey) rectangle (\stacksizex,3.*\stacksizey) node[midway]{redzone3};
        }
      \end{tikzpicture}
    \end{multicols}
    \vspace{1mm}
  \end{exampleblock}
\end{frame}

\begin{frame}[fragile]
  \vspace{-1\baselineskip}
  \begin{columns}
    \column{\textwidth+1cm}
    \scriptsize
    \begin{Verbatim}[commandchars=\\\{\}]
    \ttfamily
\textcolor{teal}{==34015==ERROR: AddressSanitizer: stack-buffer-overflow on address 0x7ffee93ed968 at pc 0x000106812df4 bp 0x7ffee93ed930 sp 0x7ffee93ed928}
\textcolor{blue}{WRITE of size 1 at 0x7ffee93ed968 thread T0}
    #0 0x106812df3 in foo() asan.cpp:4
    #1 0x106812ed8 in main asan.cpp:9
    #2 0x7fff6d3923d4 in start (libdyld.dylib:x86_64+0x163d4)

\textcolor{teal}{Address 0x7ffee93ed968 is located in stack of thread T0 at offset 40 in frame}
    #0 0x106812cdf in foo() asan.cpp:1

  This frame has 2 object(s):
    [32, 40) 'a' (line 2) \textcolor{teal}{<== Memory access at offset 40 overflows this variable}
    [64, 72) 'b' (line 3)
Shadow bytes around the buggy address:
=>0x1fffdd27db20: 00 00 00 00 00 00 00 00 \textcolor{red}{f1 f1 f1 f1} 00[\textcolor{red}{f2}]\textcolor{red}{f2 f2}
  0x1fffdd27db30: 00 \textcolor{red}{f3 f3 f3} 00 00 00 00 00 00 00 00 00 00 00 00
  0x1fffdd27db40: 00 00 00 00 00 00 00 00 00 00 00 00 00 00 00 00
  0x1fffdd27db50: 00 00 00 00 00 00 00 00 00 00 00 00 00 00 00 00
Shadow byte legend (one shadow byte represents 8 application bytes):
  Addressable:           00
  Partially addressable: 01 02 03 04 05 06 07
  Heap left redzone:       \textcolor{red}{fa}
  Freed heap region:       \textcolor{pink}{fd}
  Stack left redzone:      \textcolor{red}{f1}
  Stack mid redzone:       \textcolor{red}{f2}
  Stack right redzone:     \textcolor{red}{f3}
  Stack after return:      \textcolor{pink}{f5}
    \end{Verbatim}
  \end{columns}
\end{frame}

\begin{frame}[fragile]
  \begin{columns}
    \column{\textwidth+1cm}
    \begin{block}{Finding memory leaks with ASan}
      \begin{itemize}
        \item On linux, ASan can display memory leaks
        \item Start executable with \mintinline{bash}{ASAN_OPTIONS=detect_leaks=1 ./myProgram}
      \end{itemize}
    \end{block}
    \scriptsize
    \begin{Verbatim}[commandchars=\\\{\}]
    \ttfamily
\textcolor{red}{==113262==ERROR: LeakSanitizer: detected memory leaks}

\textcolor{blue}{Direct leak of 32 byte(s) in 1 object(s) allocated from:}
  #0 0x7f2671201647 in operator new(unsigned long) /build/dkonst/WORK/build/contrib/gcc-10.1.0/src/gcc/10.1.0/libsanitizer/asan/asan_new_delete.cpp:99
  #1 0x4033c7 in memoryLeak[abi:cxx11]() /afs/cern.ch/user/s/shageboe/asan.cpp:33
  #2 0x403633 in main /afs/cern.ch/user/s/shageboe/asan.cpp:40
  #3 0x7f2670a15492 in __libc_start_main (/lib64/libc.so.6+0x23492)

\textcolor{blue}{Indirect leak of 22 byte(s) in 1 object(s) allocated from:}
  #0 0x7f2671201647 in operator new(unsigned long) /build/dkonst/WORK/build/contrib/gcc-10.1.0/src/gcc/10.1.0/libsanitizer/asan/asan_new_delete.cpp:99
  #1 0x403846 in void std::__cxx11::basic_string<char, std::char_traits<char>, std::allocator<char> >::_M_construct<char const*>(char const*, char const*, std::forward_iterator_tag) /cvmfs/sft.cern.ch/lcg/releases/gcc/10.1.0.c82-6f386/x86_64-centos8/include/c++/10.1.0/bits/basic_string.tcc:219
  #2 0x4033f4 in std::__cxx11::basic_string<char, std::char_traits<char>, std::allocator<char> >::basic_string<std::allocator<char> >(char const*, std::allocator<char> const&) /cvmfs/sft.cern.ch/lcg/releases/gcc/10.1.0.c82-6f386/x86_64-centos8/include/c++/10.1.0/bits/basic_string.h:247
  #3 0x4033f4 in memoryLeak[abi:cxx11]() /afs/cern.ch/user/s/shageboe/asan.cpp:33
  #4 0x403633 in main /afs/cern.ch/user/s/shageboe/asan.cpp:40
  #5 0x7f2670a15492 in __libc_start_main (/lib64/libc.so.6+0x23492)

SUMMARY: AddressSanitizer: 54 byte(s) leaked in 2 allocation(s).
    \end{Verbatim}
  \end{columns}
\end{frame}

\begin{frame}[fragile]
  \frametitle{Address sanitizer (ASan)}
  \begin{block}{Wrap up}
    \begin{itemize}
      \item If a program crashes, run it with asan
      \item Should be part of every \cpp{} continuous integration system
      \item It will also find bugs that by luck didn't crash the program
      \item It doesn't generate false positives
    \end{itemize}
  \end{block}

  \begin{exampleblock}{More info}
    \begin{itemize}
      \item \url{https://github.com/google/sanitizers/wiki/AddressSanitizer}
      \item Compile with asan, and start executable using \mintinline{bash}{ASAN_OPTIONS=help=1 <executable>}
    \end{itemize}
  \end{exampleblock}
\end{frame}

\begin{frame}[fragile]
  \frametitle{Address sanitizer (ASan)}
  \begin{exercise}{address sanitizer}
    \begin{itemize}
      \item Go to \texttt{code/asan}
      \item Compile and run the program \texttt{./asan}
      \item There are two bugs and one memory leak. Use asan to trace them down.
    \end{itemize}
  \end{exercise}

\end{frame}

\begin{frame}[fragile]
  \frametitle{Thread sanitizer (TSan)}
  \begin{block}{TSan}
    \begin{itemize}
      \item Thread sanitizer detects many data races in MT programs
      \item Recompile your program with e.g.\ \mintinline{shell}{clang++ -fsanitize=thread -g -O1 datarace.cpp}
    \end{itemize}
  \end{block}

  \footnotesize
  \begin{verbatim}
% ./a.out
WARNING: ThreadSanitizer: data race (pid=19219)
Write of size 4 at 0x7fcf47b21bc0 by thread T1:
  #0 Thread1 datarace.c:4 (exe+0x00000000a360)

Previous write of size 4 at 0x7fcf47b21bc0 by main thread:
  #0 main datarace.c:10 (exe+0x00000000a3b4)

Thread T1 (running) created at:
  #0 pthread_create tsan_interceptors.cc:705 (exe+0x00000000c790)
  #1 main datarace.c:9 (exe+0x00000000a3a4)
  \end{verbatim}

  \begin{block}{}
    \scriptsize
    \url{https://github.com/google/sanitizers/wiki/ThreadSanitizerCppManual}
  \end{block}
\end{frame}

\begin{frame}[fragile]
  \frametitle{Undefined Behaviour Sanitizer (UBSan)}
  \begin{block}{UBSan}
    \begin{itemize}
      \item Tracks uninitialised memory, broken arithmetic, wrong array indexing and other undefined behaviour
      \item Recompile your program with e.g.\ \mintinline{bash}{clang++ -fsanitize=undefined -g -O1 ub.cpp}
    \end{itemize}
  \end{block}
  \small
  \begin{verbatim}
% ./a.out
up.cpp:3:5: runtime error: signed integer overflow:
            2147483647 + 1 cannot be represented in type 'int'
  \end{verbatim}
  \begin{block}{}
    \footnotesize
    \url{https://clang.llvm.org/docs/UndefinedBehaviorSanitizer.html}
  \end{block}
\end{frame}
