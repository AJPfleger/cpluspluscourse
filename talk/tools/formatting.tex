\subsection[format]{Code formatting}

\begin{frame}[fragile]
\frametitle{clang-format}
\begin{block}{.clang-format}
	\begin{itemize}
		\item file describing your formatting preferences
		\item should be checked-in at the repository root (project wide)
		\item \mintinline{bash}{clang-format -style=LLVM -dump-config >} \\
		  \mintinline{bash}{.clang-format}
		\item adapt style options with help from: \url{https://clang.llvm.org/docs/ClangFormatStyleOptions.html}
	\end{itemize}
\end{block}
\begin{block}{Run clang-format}
	\begin{itemize}
		\item \mintinline{bash}{clang-format --style=LLVM -i <file.cpp>}
		\item \mintinline{bash}{clang-format -i <file.cpp>} (looks for .clang-format file)
		\item \mintinline{bash}{git clang-format} (formats local changes)
		\item \mintinline{bash}{git clang-format <ref>} (formats changes since git \textless{}ref\textgreater{})
		\item Some editors/IDEs find a .clang-format file and adapt
	\end{itemize}
\end{block}
\end{frame}

\begin{frame}[fragile]
\frametitle{clang-format}
\begin{exercise}{clang-format}
	\begin{itemize}
		\item go to any example
		\item format code with: \mintinline{bash}{clang-format --style=GNU -i <file.cpp>}
		\item inspect changes, try \mintinline{bash}{git diff .}
		\item revert changes using \mintinline{bash}{git checkout -- <file.cpp>} or \mintinline{bash}{git checkout .}
		\item go to code directory and create a .clang-format file \\
		  \mintinline{bash}{clang-format -style=LLVM -dump-config >} \\
		  \mintinline{bash}{.clang-format}
		\item run \mintinline{bash}{clang-format -i <any_exercise>/*.cpp}
		\item revert changes using \mintinline{bash}{git checkout <any_exercise>}
	\end{itemize}
\end{exercise}
\end{frame}
