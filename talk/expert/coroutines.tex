\subsection{Coroutines}

\begin{frame}[fragile]
  \frametitlecpp[20]{Why do we need coroutines?}
  \begin{block}{The generator use case (one of several)}
    In python one can write
    {\scriptsize
      \begin{pythoncode*}{gobble=2}
        for i in range(0, 10):
          print(i)
      \end{pythoncode*}
    }
    What about it in \cpp?
    {\scriptsize
      \begin{cppcode*}{gobble=2, firstnumber=3}
        someType range(int first, int last) { ... }
        for (auto i : range(0, 10)) {
          std::cout << i << "\n";
        }
      \end{cppcode*}
    }
    How can we implement range?
  \end{block}
  \begin{exampleblock}{Requirements}
    \begin{itemize}
    \item \cppinline{someType} needs to be some iterable type
    \item \cppinline{range} should execute lazily, only creating values on demand
    \item meaning \cppinline{range} should be suspended and resumed
    \end{itemize}
    That's what coroutines are for!
  \end{exampleblock}
\end{frame}

\begin{frame}[fragile]
  \frametitlecpp[20]{What \cpp20 allows}
  \begin{exampleblock}{Interesting part of the implementation (see \href{https://en.cppreference.com/w/cpp/coroutine/coroutine\_handle\#Example}{\color{blue!50!white}cppreference})}
    {\scriptsize
      \begin{cppcode*}{gobble=2}
        template<std::integral T>
        Generator<T> range(T first, const T last) {
          while (first < last) {
            co_yield first++;
          }
        }
        for (const char i : range(65, 91)) {
          std::cout << i << ' ';
        }
        std::cout << '\n';
      \end{cppcode*}
    }
  \end{exampleblock}
  \begin{alertblock}{The bad news}
    \begin{itemize}
    \item class \cppinline{Generator} is not provided by the STL
    \item and is $\sim80$ lines of boiler plate code...
    \item may be sorted out in \cpp23
    \item in the meantime, we'll have to dive into some details!
    \end{itemize}
  \end{alertblock}
\end{frame}

\begin{frame}
  \frametitlecpp[20]{A few concepts first}
  \begin{block}{Definition of a coroutine}
    \begin{itemize}
    \item a function which can be suspended and resumed
      \begin{itemize}
      \item potentially by another caller (even on a different thread)
      \end{itemize}
    \item practically a function using one of:
      \begin{description}
      \item[\cppinline{co_await}] suspends execution
      \item[\cppinline{co_yield}] suspends execution and returns a value
      \item[\cppinline{co_return}] completes execution
      \end{description}
    \end{itemize}
  \end{block}
  \begin{alertblock}{Implications}
    \begin{itemize}
    \item on suspension, the state of the function needs to be preserved
    \item the suspended function becomes pure data on the heap
    \item access to this data is given via the returned value
      \begin{itemize}
      \item so the returned type must follow some convention
      \item it needs to contain a \cppinline{struct promise_type}
      \end{itemize}
    \end{itemize}
  \end{alertblock}
\end{frame}

\begin{frame}[fragile]
  \frametitlecpp[20]{Minimal code}
  \begin{block}{Definition of an empty coroutine}
    {\scriptsize
      \begin{cppcode*}{gobble=2, firstnumber=3}
        #include <coroutine>
        struct Task {
          struct promise_type {
            Task get_return_object() { return {}; }
            std::suspend_never initial_suspend() { return {}; }
            std::suspend_never final_suspend() noexcept { return {}; }
            void return_void() {}
            void unhandled_exception() { throw; }
          };
        };
        Task myCoroutine() { co_return; }
        int main() { Task x = myCoroutine(); }
      \end{cppcode*}
    }
  \end{block}
  \begin{exampleblock}{\texttt{promise\_type} interface}
    \begin{description}[\small \cppinline{initial/final_suspend}]
      \setlength{\itemsep}{0pt}
    \item[\small \cppinline{get_return_object}] builds a Task from the given promise
    \item[\small \cppinline{initial/final_suspend}] called before/after coroutine starts/ends
    \item[\small \cppinline{unhandled_exception}] called in case of exception in coroutine
    \item[\small \cppinline{return_void/value}] returns final value of the coroutine
    \end{description}
  \end{exampleblock}
\end{frame}

\begin{frame}[fragile]
  \frametitlecpp[20]{Minimal sequence of events (nothing suspended)}
  \begin{block}{Compiler generated set of calls when calling \cppinline{myCoroutine}}
    \center
    \begin{tikzpicture}[->, node distance=.75cm, font=\tiny, scale=0.8, every node/.style={scale=0.8},xscale=6,-{Latex[length=1mm,width=1mm]}]
      \tikzstyle{call}=[rectangle,draw=black,thick,inner sep=6pt,rounded corners,minimum width=4.5cm,minimum height=.6cm,fill=red!30]
      \tikzstyle{comp}=[trapezium,trapezium left angle=70,trapezium right angle=110,draw=black,thick,inner sep=6pt,trapezium stretches=true,minimum width=4.5cm,minimum height=.6cm,fill=blue!30]
      \tikzstyle{box}=[rectangle,draw=black,thick,inner sep=6pt,minimum width=4.5cm,minimum height=.6cm,fill=green!30]
      \node[call] at (0, 0) (initCall) {call to coroutine};
      \node[comp] at (0,-1) (allocState) {allocate coroutine state on the heap};
      \node[comp] at (1,-1) (construct) {construct promise inside coroutine state};
      \node[box]  at (0,-2) (getReturn) {\texttt{ro = promise.get\_return\_object()}};
      \node[comp] at (1,-2) (localVar) {\texttt{ro} returned to caller};
      \node[box]  at (0,-3) (initSuspend) {\texttt{promise.initial\_suspend()}};
      \node[box] at (1,-3) (initWait) {\texttt{co\_await} on result};
      \node[call] at (0,-4) (body) {coroutine body};
      \node[box]  at (1,-4) (coreturn) {\texttt{co\_return}};
      \node[box] at (1,-5) (finalSuspend) {\texttt{promise.final\_suspend()}};
      \node[box]  at (0,-6) (finalWait) {\texttt{co\_await} on result};
      \node[box]  at (0,-5) (return) {\texttt{promise.return\_void()}};
      \node[comp]  at (1,-6) (destroyP) {destroy promise};
      \node[call] at (0,-7) (caller) {caller};
      \node[comp]  at (1,-7) (destroyC) {destroy coroutine};
      \draw (initCall)      edge (allocState)
            (allocState)    edge (construct)
            (construct)     --   +(0,-.5) -| +(-1,-0.5) edge (getReturn)
            (getReturn)     edge (localVar)
            (localVar.east) --   +(0.1,0) |- (1,-7.5) -| (caller);
      \draw (getReturn)    edge (initSuspend)
            (initSuspend)  edge (initWait)
            (initWait)     -- +(0,-.5) -| (body)
            (body)         edge (coreturn)
            (coreturn)     -- +(0,-.5) -| (return)
            (return)       edge (finalSuspend)
            (finalSuspend) -- +(0,-.5) -| (finalWait)
            (finalWait)    edge (destroyP)
            (destroyP)     edge (destroyC)
            (destroyC)     edge (caller);
      \node[call,minimum width=1cm] at (.7, 0) {user code};
      \node[comp,minimum width=1cm] at (1, 0) {compiler};
      \node[box, minimum width=1cm] at (1.3, 0) {promise};
    \end{tikzpicture}
  \end{block}
  \vspace{-2mm}
  \tiny Credits : original idea for this graph comes from \href{https://itnext.io/c-20-coroutines-complete-guide-7c3fc08db89d}{this web site}
\end{frame}

\begin{frame}[fragile]
  \frametitlecpp[20]{Suspending a coroutine}
  \begin{block}{Main way to suspend a coroutine: \cppinline{co_wait}}
    \begin{itemize}
    \item it's an expression taking an ``awaitable''/ ``awaiter''
      \begin{itemize}
      \item we'll ignore the distinction for now
      \end{itemize}
    \item 2 trivial awaiters are provided by the STL: \cppinline{std::suspend_always} and \cppinline{std::suspend_never}
    \item obviously only the first one will trigger a suspension
      \begin{itemize}
      \item note the other one is used in the minimal code
      \end{itemize}
    \item more details on awaiters later
    \end{itemize}
  \end{block}
  \begin{exampleblock}{Code}
    {\scriptsize
      \begin{cppcode*}{gobble=4}
        Task myCoroutine() {
          std::cout << "Step 1 of coroutine\n";
          co_await std::suspend_always{};
          std::cout << "Step 2 of coroutine\n";
          co_await std::suspend_always{};
          std::cout << "final step\n";
        }
      \end{cppcode*}
    }
  \end{exampleblock}
\end{frame}

\begin{frame}[fragile]
  \frametitlecpp[20]{Resuming a coroutine}
  \begin{block}{Principles}
    \begin{itemize}
    \item one needs to call \cppinline{resume()} on the \cppinline{coroutine_handle}
    \item BUT user code has no such handle
      \begin{itemize}
      \item only the coroutine return object (here a \cppinline{Task} instance)
      \end{itemize}
    \item so one has to amend the Task class
      \begin{itemize}
      \item and the \cppinline{promise_type} as it's building the \cppinline{Task} instance
      \end{itemize}
    \end{itemize}
  \end{block}
  \begin{exampleblock}{Code}
    {\scriptsize
      \begin{cppcode*}{gobble=4}
        struct Task {
          struct promise_type {
            Task get_return_object() {
              // build Task with the coroutine handle
              return {std::coroutine_handle<promise_type>::from_promise(*this)};
            }
            ... // rest is unchanged
          };
          std::coroutine_handle<promise_type> h_;
          void resume() { h_.resume(); }
        };
      \end{cppcode*}
    }
  \end{exampleblock}
\end{frame}

\begin{frame}[fragile]
  \frametitlecpp[20]{Resuming a coroutine}
  \scriptsize
  \begin{exampleblockGB}{User code}{https://godbolt.org/z/qx46Pa4v3}{Resuming a coroutine}
    \begin{cppcode*}{gobble=2}
      Task myCoroutine() {
        std::cout << "Step 1 of coroutine\n";
        co_await std::suspend_always{};
        std::cout << "Step 2 of coroutine\n";
        co_await std::suspend_always{};
        std::cout << "final step\n";
      }
      int main() {
        auto c = myCoroutine(); // Step 1 runs
        std::cout << "In main, between Step 1 and 2\n";
        c.resume();             // Step 2 runs
        std::cout << "In main, between Step 2 and final step\n";
        c.resume();             // final Step runs
        // c.resume(); // would segfault!
      }
    \end{cppcode*}
  \end{exampleblockGB}
  \begin{block}{}
    \begin{minted}[gobble=4]{text}
      Step 1 of coroutine
      In main, between Step 1 and 2
      Step 2 of coroutine
      In main, between Step 2 and final step
      final step
    \end{minted}
  \end{block}
\end{frame}

\begin{frame}[fragile]
  \frametitlecpp[20]{A few more details}
  \begin{block}{About suspension of coroutines}
    \begin{itemize}
    \item \cppinline{initial/final_suspend} methods allow to suspend a coroutine at the very start/end (after \cppinline{co_return})
    \item a suspended coroutine is not deallocated
      \begin{itemize}
      \item i.e.\ its state will stay on the heap
      \item so you can leak coroutines!
      \end{itemize}
    \item one can directly call \cppinline{coroutine_handle::destroy} if the execution of a suspended coroutine should be aborted
      \begin{itemize}
      \item which requires another piece of code in \cppinline{Task}, e.g.
        {\scriptsize
          \begin{cppcode*}{gobble=8,linenos=false}
            ~Task() { if (h_) h_.destroy(); }
          \end{cppcode*}
        }
      \end{itemize}
    \item the \cppinline{Task} class should be move only
     \end{itemize}
  \end{block}
  \begin{alertblock}{We need standard \cppinline{Task} classes in the STL!}
    \begin{itemize}
    \item \cppinline{std::generator} may come in \cpp23
     \end{itemize}
    \end{alertblock}
\end{frame}

\begin{frame}[fragile]
  \frametitlecpp[20]{\texttt{co\_yield} - returning a value upon suspension}
  \begin{block}{What \texttt{co\_yield} really is}
    \begin{itemize}
    \item an expression like \cppinline{co_await}
    \item a shortcut for \cppinline{co_await promise.yield_value(expr)}
    \item and we have to provide the \cppinline{yield_value} method of \cppinline{promise_type} and the \cppinline{Task} accessor
    \end{itemize}
  \end{block}
  \begin{exampleblock}{Typical code}
    {\scriptsize
      \begin{cppcode*}{gobble=4}
        template<typename T> struct Task {
          struct promise_type {
            T value_;
            std::suspend_always yield_value(T&& from) {
              value_ = std::forward<T>(from); // caching the result
              return {};
            }
            ... // rest is unchanged
          };
          T getValue() { return h_.promise().value_; } // accessor for user code
        };
      \end{cppcode*}
    }
  \end{exampleblock}
\end{frame}

\begin{frame}[fragile]
  \frametitlecpp[20]{\texttt{co\_yield} - generator behavior}
  \begin{block}{Generator behavior - more boiler plate code}
    {\scriptsize
      \begin{cppcode*}{gobble=4}
        template<typename T>
        struct Task {
          struct iterator {
            std::coroutine_handle<promise_type> handle;
            auto &operator++() {
              handle.resume();
              if (handle.done()) { handle = nullptr; } // == end iterator
              return *this;
            }
            auto operator++(int) { ++*this; }
            T const& operator*() const { return handle.promise().value_; }
            ...
          };
          iterator begin() { resume(); return {h_}; }
          iterator end() { return {nullptr}; }
          ... // plus error handling is missing
        };
      \end{cppcode*}
    }
  \end{block}
\end{frame}

\begin{frame}[fragile]
  \frametitlecpp[20]{\texttt{co\_yield} - final usage}
  \begin{exampleblockGB}{Finally, we can use the generator nicely}{https://godbolt.org/z/6bbrYrss5}{\texttt{co\_yield}}
    {\scriptsize
      \begin{cppcode*}{gobble=4}
        Task range(int first, int last) {
          while (first != last) {
            co_yield first++;
          }
        }
        for (int i : range(0, 10)) {
          std::cout << i << "";
        } // 1 2 3 4 5 6 7 8 9
      \end{cppcode*}
    }
  \end{exampleblockGB}
  \pause
  \begin{alertblock}{Wait a minute: 0 is missing in the output!}
    \begin{itemize}
    \item resume called in \cppinline{operator++} before 0 is printed
    \item we should pause in \cppinline{initial_suspend}
    \end{itemize}
     {\scriptsize
      \begin{cppcode*}{gobble=4,firstnumber=10}
        struct Task {
          struct promise_type {
            ...
            std::suspend_always initial_suspend() { return {}; }
            ...
          }
        };
      \end{cppcode*}
    }
  \end{alertblock}
\end{frame}

\begin{frame}[fragile]
  \frametitlecpp[20]{Laziness allows for infinite ranges}
  \begin{exampleblockGB}{Generators do not need to be finite}{https://godbolt.org/z/MP6qdGWYe}{Infinite generator}
    \begin{cppcode*}{gobble=2}
      Task range(int first) {
        while (true) {
          co_yield first++;
        }
      }
      for (int i : range(0) | std::views::take(10)) {
        std::cout << i << "\n";
      }
    \end{cppcode*}
  \end{exampleblockGB}
\end{frame}

\begin{frame}[fragile]
  \frametitlecpp[20]{Real life example}
  \begin{exampleblockGB}{A Fibonacci generator, from \cpprefLink{https://en.cppreference.com/w/cpp/language/coroutines}}{https://godbolt.org/z/3Pev1M8se}{Fibonacci generator}
    \scriptsize
     \begin{cppcode*}{gobble=2}
       Task<long long int> fibonacci(unsigned n) {
         if (n==0) co_return;
         co_yield 0;
         if (n==1) co_return;
         co_yield 1;
         if (n==2) co_return;
         uint64_t a=0;
         uint64_t b=1;
         for (unsigned i = 2; i < n; i++) {
           uint64_t s=a+b;
           co_yield s;
           a=b;
           b=s;
         }
       }

       int main() {
         for (int i : fibonacci(10)) {
           std::cout << i << "\n";
         }
       }
    \end{cppcode*}
  \end{exampleblockGB}
\end{frame}

\begin{frame}
  \frametitlecpp[20]{Coming back to awaitables/awaiters}
  \begin{block}{awaitables vs awaiters}
    \begin{itemize}
    \item awaitable is the type required by \cppinline{co_await}/\cppinline{co_yield}
      \begin{itemize}
      \item or the expression given has to be convertible to awaitable
      \item alternatively the \cppinline{promise_type} can have an \cppinline{await_transform} method returning an awaitable
      \end{itemize}
    \item an awaiter is retrieved from the awaitable by the compiler
      \begin{itemize}
      \item either by calling operator \cppinline{co_wait} on it
      \item or via direct conversion if no such operator exists
      \end{itemize}
    \item an awaiter has 3 main methods
      \begin{itemize}
      \item \cppinline{await_ready}, \cppinline{await_suspend} \cppinline{and await_resume}
      \end{itemize}
    \end{itemize}
  \end{block}
\end{frame}

\begin{frame}[fragile]
  \frametitlecpp[20]{How awaiters work}
  \begin{block}{Compiler generated set of calls}
    \center
    \begin{tikzpicture}[->, node distance=.75cm, font=\tiny, scale=0.8, every node/.style={scale=0.8},xscale=6,-{Latex[length=1mm,width=1mm]}]
      \tikzstyle{call}=[rectangle,draw=black,thick,inner sep=6pt,rounded corners,minimum width=4.5cm,minimum height=.6cm,fill=red!30]
      \tikzstyle{comp}=[trapezium,trapezium left angle=70,trapezium right angle=110,draw=black,thick,inner sep=6pt,trapezium stretches=true,minimum width=4.5cm,minimum height=.6cm,fill=blue!30]
      \tikzstyle{box}=[rectangle,draw=black,thick,inner sep=6pt,minimum width=4.5cm,minimum height=.6cm,fill=green!30]
      \tikzstyle{choice}=[diamond,draw=black,thick,inner sep=3pt,minimum width=4.5cm,minimum height=.6cm,fill=green!30,aspect=5]
      \node[call]   at (0, 0) (co_await) {\texttt{co\_await} awaitable};
      \node[choice] at (0,-1) (await_ready) {\texttt{awaiter.await\_ready()}};
      \node[comp]   at (1,-1) (suspend) {coroutine suspended};
      \node[choice] at (.5,-2.5) (await_suspend) {\texttt{awaiter.await\_suspend(handle)}};
      \node[comp]   at (0,-4) (handleresume) {\texttt{handle.resume()}};
      \node[call]   at (1,-4) (caller) {coroutine caller};
      \node[comp]   at (0,-5) (corofin) {coroutine behind handle suspended or finished};
      \node[call]   at (1,-5) (callresume) {caller calls \texttt{resume}};
      \node[comp]   at (0,-6) (resumed) {coroutine resumed};
      \node[box]    at (1,-6) (await_resume) {\texttt{awaiter.await\_resume()}};
      \node[call]   at (1,-7) (body) {continue coroutine body};
      \draw (co_await)      edge (await_ready)
            (await_ready)   edge node[above] {false}  (suspend)
            (await_ready.west) -- node[above] {true} +(-0.1,0) |- (body);
      \draw (suspend)       -- +(0,-.5)  -|   (await_suspend);
      \draw (await_suspend) -- node[above,pos=.1] {false} +(-.95,0) |- (resumed);
      \draw (await_suspend) edge node[text width=1cm,left,pos=.3] {coroutine handle} (handleresume)
            (handleresume)  edge (corofin)
            (corofin)       edge (resumed)
            (await_suspend) |- node[right,pos=0.1] {\texttt{true}}  (caller)
            (caller)        edge (callresume)
            (callresume)    --   +(0,-.5) -| (resumed)
            (await_suspend) -|  node[right,pos=0.6] {\texttt{void}}  (caller)
            (resumed)       edge (await_resume)
            (await_resume)       edge (body);
      \node[call,minimum width=1cm] at (.7, 0) {user code};
      \node[comp,minimum width=1cm] at (1, 0) {compiler};
      \node[box, minimum width=1cm] at (1.3, 0) {awaiter};
    \end{tikzpicture}
  \end{block}
\end{frame}

\begin{frame}[fragile]
  \frametitlecpp[20]{\texttt{awaiter} example}
  \begin{exampleblockGB}{Switch to new thread, from \cpprefLink{https://en.cppreference.com/w/cpp/language/coroutines}}{https://godbolt.org/z/K7svE4P7s}{Thread switching}
    \scriptsize
    \begin{cppcode*}{gobble=2}
      auto switch_to_new_thread(std::jthread& out) {
        struct awaiter {
          std::jthread* th;
          bool await_ready() { return false; }
          void await_suspend(std::coroutine_handle<> h) {
            *th = std::jthread([h] { h.resume(); });
            std::cout << "New thread ID: " << th->get_id() << '\n';
          }
          void await_resume() {}
        };
        return awaiter{&out};
      }
      Task<int> resuming_on_new_thread(std::jthread& out) {
        std::cout << "Started on " << std::this_thread::get_id() << '\n';
        co_await switch_to_new_thread(out);
        std::cout << "Resumed on " << std::this_thread::get_id() << std::endl;
      }
      int main() {
        std::jthread out;
        resuming_on_new_thread(out);
      }
     \end{cppcode*}
  \end{exampleblockGB}
  \pause
  \vspace{-2.1cm}
  \begin{columns}
    \begin{column}{.45\textwidth}
    \end{column}
    \begin{column}{.55\textwidth}
      \setlength{\textwidth}{5.4cm}
      \raggedright
      \begin{block}{Output}
        \scriptsize
        \begin{minted}[gobble=9]{text}
          Started on thread: 140144191063872
          New thread ID: 140144191059712
          Resumed on thread: 140144191059712
        \end{minted}
      \end{block}
    \end{column}
  \end{columns}
\end{frame}

\begin{frame}[fragile]
  \frametitlecpp[20]{Summary on coroutines}
  \begin{exampleblock}{Coroutines are usable in \cpp20}
    \begin{itemize}
    \item allow to have nice generators
    \item also very useful for async/IO code
      \begin{itemize}
      \item another big use case which we did not touch
      \end{itemize}
    \end{itemize}
  \end{exampleblock}
  \begin{alertblock}{But library support is poor for now}
    \begin{itemize}
    \item standard Task/Generator classes are not provided
    \item leading to \emph{lots of boiler plate code} for the end user
    \item \cppinline{std::generator} is part of \cpp23's draft
    \item in the meantime, libraries exist, e.g. \href{https://github.com/lewissbaker/cppcoro}{CppCoro}
    \end{itemize}
  \end{alertblock}
\end{frame}
