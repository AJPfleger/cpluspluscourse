\subsection[sfinae]{SFINAE}

%https://jguegant.github.io/blogs/tech/sfinae-introduction.html
\begin{frame}[fragile]
  \frametitlecpp[11]{Substitution Failure Is Not An Error (SFINAE)}
  \begin{block}{The main idea}
    \begin{itemize}
    \item substitution replaces template parameters with the provided arguments (types or values)
    \item if it leads to invalid code, do not fail but try other overloads
    \end{itemize}
  \end{block}
  \begin{exampleblock}{Example}
    \begin{cppcode*}{}
      template <typename T>
      void f(typename T::type arg) { ... }

      void f(int a) { ... }

      f(1); // Calls void f(int)
    \end{cppcode*}
  \end{exampleblock}
  Note : SFINAE is largely superseded by concepts in \cpp20
\end{frame}

\begin{frame}[fragile]
  \frametitlecpp[11]{decltype}
  \begin{block}{The main idea}
    \begin{itemize}
    \item gives the type of the result of an expression
    \item the expression is not evaluated
    \item at compile time
    \end{itemize}
  \end{block}
  \begin{exampleblock}{Example}
    \begin{cppcode*}{}
      struct A { double x; };
      A a;
      decltype(a.x) y;       // double
      decltype((a.x)) z = y; // double& (lvalue)
      decltype(1 + 2u) i = 4; // unsigned int

      template<typename T, typename U>
      auto add(T t, U u) -> decltype(t + u);
      // return type depends on template parameters
    \end{cppcode*}
  \end{exampleblock}
\end{frame}

\begin{frame}[fragile]
  \frametitlecpp[11]{declval}
  \begin{block}{The main idea}
    \begin{itemize}
    \item gives you a reference to a ``fake'' object at compile time
    \item useful for types that cannot easily be constructed
    \item use only in unevaluated contexts, e.g. inside \mintinline{cpp}{decltype}
    \end{itemize}
  \end{block}
  \begin{exampleblock}{}
    \begin{cppcode*}{}
      struct Default {
        int foo() const { return 1; }
      };
      struct NonDefault {
        NonDefault(int i) { }
        int foo() const { return 1; }
      };
      decltype(Default().foo()) n1 = 1;     // int
      decltype(NonDefault().foo()) n2 = n1; // error
      decltype(std::declval<NonDefault>().foo()) n2 = n1;
    \end{cppcode*}
  \end{exampleblock}
\end{frame}

\begin{frame}[fragile]
  \frametitlecpp[11]{true\_type and false\_type}
  \begin{block}{The main idea}
    \begin{itemize}
    \item encapsulate a compile-time boolean as type
    \item can be inherited
    \end{itemize}
  \end{block}
  \begin{exampleblock}{Example}
    \begin{cppcode*}{}
      struct truth : std::true_type { };

      constexpr bool test = truth::value; // true
      constexpr truth t;
      constexpr bool test = t(); // true
      constexpr bool test = t;   // true
    \end{cppcode*}
  \end{exampleblock}
\end{frame}

\begin{frame}[fragile]
  \frametitlecpp[11]{Using SFINAE for introspection}
  \begin{block}{The main idea}
    \begin{itemize}
    \item use a template specialization
      \begin{itemize}
      \item that may or may not create valid code
      \end{itemize}
    \item use SFINAE to choose between them
    \item inherit from true/false\_type
    \end{itemize}
  \end{block}
  \begin{exampleblock}{Example}
    \small
    \begin{cppcode*}{}
      template <typename T, typename = void>
      struct hasFoo : std::false_type {};
      template <typename T>
      struct hasFoo<T, decltype(std::declval<T>().foo())>
        : std::true_type {};
      struct A{}; struct B{ void foo(); };
      static_assert(!hasFoo<A>::value, "A has no foo()");
      static_assert(hasFoo<B>::value, "B has foo()");
    \end{cppcode*}
  \end{exampleblock}
\end{frame}

\begin{frame}[fragile]
  \frametitlecpp[11]{Not so easy actually...}
  \begin{exampleblock}{Example}
    \small
    \begin{cppcode*}{}
      template <typename T, typename = void>
      struct hasFoo : std::false_type {};
      template <typename T>
      struct hasFoo<T, decltype(std::declval<T>().foo())>
        : std::true_type {};

      struct A{};
      struct B{void foo();};
      struct C{int foo();};

      static_assert(!hasFoo<A>::value, "A has no foo()");
      static_assert(hasFoo<B>::value, "B has foo()");
      static_assert(!hasFoo<C>::value, "C has foo()");
      static_assert(hasFoo<C,int>::value, "C has foo()");
    \end{cppcode*}
  \end{exampleblock}
\end{frame}

\begin{frame}[fragile]
  \frametitlecpp[17]{Using \texttt{void\_t}}
  \begin{block}{Concept}
    \begin{itemize}
    \item Maps a sequence of given types to \mintinline{cpp}{void}
    \item Introduced in \cpp17 though trivial to implement in \cpp11
    \item Can be used in specializations to check the validity of an expression
    \end{itemize}
  \end{block}
  \begin{block}{Implementation in header type\_traits}
    \begin{cppcode*}{gobble=2}
      template <typename...>
      using void_t = void;
    \end{cppcode*}
  \end{block}
\end{frame}

\begin{frame}[fragile]
  \frametitlecpp[17]{Previous example using \texttt{void\_t}}
    \begin{exampleblock}{Example}
      \begin{cppcode*}{}
      template <typename T, typename = void>
      struct hasFoo : std::false_type {};

      template <typename T>
      struct hasFoo<T,
         std::void_t<decltype(std::declval<T>().foo())>>
      : std::true_type {};

      struct A{}; struct B{ void foo(); };
      struct C{ int foo(); };

      static_assert(!hasFoo<A>::value,"unexpected foo()");
      static_assert(hasFoo<B>::value, "expected foo()");
      static_assert(hasFoo<C>::value, "expected foo()");
      \end{cppcode*}
    \end{exampleblock}
\end{frame}

\begin{frame}[fragile]
  \frametitle{SFINAE and the STL \hfill \cpp11/\cpp14/\cpp17}
  \begin{block}{enable\_if / enable\_if\_t}
    \begin{cppcode*}{gobble=2}
      template<bool B, typename T=void> struct enable_if {};
      template<typename T>
      struct enable_if<true, T> { using type = T; };
      template<bool B, typename T = void>
      using enable_if_t = typename enable_if<B,T>::type;
    \end{cppcode*}
    \begin{itemize}
    \item If \mintinline{cpp}{B} is true, has an alias \mintinline{cpp}{type} to type \mintinline{cpp}{T}
    \item otherwise, has no \mintinline{cpp}{type} alias
    \end{itemize}
  \end{block}
  \begin{block}{is\_*$<T>$/is\_*\_v$<T>$ (float/signed/object/final/abstract/...)}
    \begin{itemize}
    \item Standard type traits in header \mintinline{cpp}{<type_traits>}
    \item Checks at compile time whether \mintinline{cpp}{T} is ...
    \item Result in boolean member \mintinline{cpp}{value}
    \end{itemize}
  \end{block}
\end{frame}

\begin{frame}[fragile]
  \begin{exampleblock}{Gaudi usage example}
    \begin{cppcode*}{}
      constexpr struct deref_t {
        template
          <typename In,
           typename = typename std::enable_if_t
                      <!std::is_pointer_v<In>>>
        In& operator()(In& in) const { return in; }

        template <typename In>
        In& operator()(In* in) const {
          assert(in!=nullptr); return *in;
        }
      } deref{};
    \end{cppcode*}
  \end{exampleblock}
\end{frame}

\begin{frame}[fragile]
  \frametitlecpp[11]{Back to variadic templated class}
  \begin{block}{The tuple get method}
    \begin{cppcode*}{}
      template <size_t I, typename Tuple>
      struct elem_type;

      template <typename T, typename... Ts>
      struct elem_type<0, tuple<T, Ts...>> {
        using type = T;
      };

      template <size_t I, typename T, typename... Ts>
      struct elem_type<I, tuple<T, Ts...>> {
        using type = typename elem_type
           <I - 1, tuple<Ts...>>::type;
      };
    \end{cppcode*}
  \end{block}
\end{frame}

\begin{frame}[fragile]
  \frametitlecpp[14]{Back to variadic templated class}
  \begin{block}{The tuple get function}
    \begin{cppcode*}{}
      template <size_t I, typename... Ts>
      std::enable_if_t<I == 0,
        typename elem_type<0, tuple<Ts...>>::type&>
      get(tuple<Ts...>& t) {
        return t.m_head;
      }
      template <size_t I, typename T, typename... Ts>
      std::enable_if_t<I != 0,
        typename elem_type<I - 1, tuple<Ts...>>::type&>
      get(tuple<T, Ts...>& t) {
        tuple<Ts...>& base = t;
        return get<I - 1>(base);
      }
    \end{cppcode*}
  \end{block}
\end{frame}

\begin{frame}[fragile]
  \frametitlecpp[17]{with if constexpr}
  \begin{block}{The tuple get function}
    \begin{cppcode*}{gobble=2}
      template <size_t I, typename T, typename... Ts>
      auto& get(tuple<T, Ts...>& t) {
        if constexpr(I == 0)
          return t.m_head;
        else
          return get<I - 1>(static_cast<tuple<Ts...>&>(t));
      }
    \end{cppcode*}
  \end{block}
  \begin{goodpractice}{SFINAE vs.\ if constexpr}
    \begin{itemize}
      \item \mintinline{cpp}{if constexpr} can replace SFINAE in many places.
      \item It is usually more readable as well. Use it if you can.
    \end{itemize}
  \end{goodpractice}
\end{frame}
