\subsection[forward]{Perfect forwarding}

%http://eli.thegreenplace.net/2014/perfect-forwarding-and-universal-references-in-c/
\begin{frame}[fragile]
  \frametitlecpp[11]{The problem}
  How to write a generic wrapper function?
  \begin{block}{}
    \begin{cppcode*}{}
      template <typename T>
      void wrapper(T arg) {
        // code before
        func(arg);
        // code after
      }
    \end{cppcode*}
  \end{block}
  Example usage :
  \begin{itemize}
  \item \cppinline{emplace_back}
  \item \cppinline{make_unique}
  \end{itemize}
\end{frame}

\begin{frame}[fragile]
  \frametitlecpp[11]{Why is it not so simple?}
  \begin{block}{}
    \begin{cppcode*}{}
      template <typename T>
      void wrapper(T arg) {
        func(arg);
      }
    \end{cppcode*}
  \end{block}
  \begin{alertblock}{What about references?}
    \begin{itemize}
      \item what if \cppinline{func} takes a reference to avoid copies?
      \item wrapper would force a copy and we fail to use references
    \end{itemize}

  \end{alertblock}
\end{frame}

\begin{frame}[fragile]
  \frametitlecpp[11]{Second try, second failure ?}
  \begin{block}{}
    \begin{cppcode*}{}
      template <typename T>
      void wrapper(T& arg) {
        func(arg);
      }
      wrapper(42);
      // invalid initialization of
      // non-const reference from
      // an rvalue
    \end{cppcode*}
  \end{block}
  \begin{alertblock}{}
     and \cppinline{const T&} won't work when passing something non const
  \end{alertblock}
\end{frame}

\begin{frame}[fragile]
  \frametitlecpp[11]{Solution: cover all cases}
  \begin{block}{}
    \begin{cppcode*}{}
      template <typename T>
      void wrapper(T& arg) { func(arg); }

      template <typename T>
      void wrapper(const T& arg) { func(arg); }

      template <typename T>
      void wrapper(T&& arg) { func(std::move(arg)); }
    \end{cppcode*}
  \end{block}{}
\end{frame}

\begin{frame}[fragile]
  \frametitlecpp[11]{The new problem: scaling to more arguments}
  \begin{block}{}
    \begin{cppcode*}{}
      template <typename T1, typename T2>
      void wrapper(T1& arg1, T2& arg2)
      { func(arg1, arg2); }

      template <typename T1, typename T2>
      void wrapper(const T1& arg1, T2& arg2)
      { func(arg1, arg2); }

      template <typename T1, typename T2>
      void wrapper(T1& arg1, const T2& arg2)
      { func(arg1, arg2); }
      ...
    \end{cppcode*}
  \end{block}{}
  \begin{alertblock}{Exploding complexity}
    \begin{itemize}
      \item for $n$ arguments, 3$^{n}$ overloads
      \item you do not want to try n = 5...
    \end{itemize}
  \end{alertblock}
\end{frame}

\begin{frame}[fragile]
  \frametitlecpp[11]{Reference collapsing}
  \begin{block}{Reference to references}
  	\begin{itemize}
  	\item Are formally forbidden, but can occur sometimes
    \end{itemize}
  \end{block}
  \begin{block}{}
    \begin{cppcode*}{}
      template <typename T>
      void foo(T t) { T& k = t; } //int& &, error in C++98
      int ii = 4;
      foo<int&>(ii); // want to pass by reference
    \end{cppcode*}
  \end{block}
  \begin{block}{\cpp11 added rvalue-references}
    \begin{itemize}
      \item More combinations possible, like: \cppinline{int&& &}, or \cppinline{int&& &&}
    \end{itemize}
  \end{block}
  \begin{exampleblock}{Reference collapsing}
    \begin{itemize}
      \item Rule: When multiple references are involved, \cppinline{&} always wins
      \item \cppinline{T&& &, T& &&, T& &} $\rightarrow$ \cppinline{T&}
      \item \cppinline{T&& &&} $\rightarrow$ \cppinline{T&&}
    \end{itemize}
  \end{exampleblock}
\end{frame}

\begin{frame}[fragile]
  \frametitlecpp[11]{Forwarding references}
  \begin{exampleblock}{Rvalue reference in type-deducing context}
    \begin{cppcode*}{}
      template <typename T>
      void f(T&& t) { ... }
    \end{cppcode*}
  \end{exampleblock}
  \begin{block}{Forwarding reference}
    \begin{itemize}
      \item Next to a template parameter, that can be deduced, \cppinline{&&} is not an rvalue reference, but a ``forwarding reference''
      \begin{itemize}
        \item aka.\ ``universal reference''
      	\item So, this applies only to functions
      \end{itemize}
      \item The template parameter is additionally allowed to be deduced as an lvalue reference, and reference collapsing proceeds:
      \begin{itemize}
        \item if an lvalue of type \cppinline{U} is given, \cppinline{T} is deduced as \cppinline{U&} and the parameter type \cppinline{U& &&} collapses to \cppinline{U&}
        \item otherwise (rvalue), T is deduced as \cppinline{U}
        \begin{itemize}
          \item and forms the parameter type \cppinline{U&&}
        \end{itemize}
      \end{itemize}
    \end{itemize}
  \end{block}
\end{frame}

\begin{frame}[fragile]
  \frametitlecpp[11]{Forwarding references}
  \begin{exampleblock}{Examples}
    \begin{cppcode*}{}
      template <typename T>
      void f(T&& t) { ... }

      f(4);            // rvalue -> T&& is int&&
      double d = 3.14;
      f(d);            // lvalue -> T&& is double&
      float g() {...}
      f(g());          // rvalue -> T&& is float&&
      std::string s = "hello";
      f(s);            // lvalue -> T&& is std::string&
      f(std::move(s)); // rvalue -> T&& is std::string&&
      f(std::string{"hello"}); // rvalue -> std::string&&
    \end{cppcode*}
  \end{exampleblock}
\end{frame}

\begin{frame}[fragile]
  \frametitlecpp[20]{Forwarding references}
  \begin{alertblock}{Careful!}
    \begin{cppcode*}{gobble=2}
      template <typename T> struct S {
        S(T&& t) { ... }       // rvalue references
        void f(T&& t) { ... }  // NOT forwarding references
      };
    \end{cppcode*}
  \end{alertblock}
  \vspace{-1.2\baselineskip}
  \begin{overprint}
  \onslide<1>
  \begin{exampleblock}{Half-way correct version}
    \begin{cppcode*}{gobble=2}
      template <typename T> struct S {

        template <typename U>
        S(U&& t) { ... } // deducing context -> fwd. ref.
        template <typename U>
        void f(U&& t)    // deducing context -> fwd. ref.
        // ... but now U can be a different type than T
      };
    \end{cppcode*}
  \end{exampleblock}
  \onslide<2>
  \begin{exampleblock}{Correct version}
    \begin{cppcode*}{gobble=2}
      template <typename T> struct S {
        template <typename U, std::enable_if_t<
          std::is_same_v<std::decay_t<U>, T>, int> = 0>
        S(U&& t) { ... } // deducing context -> fwd. ref.
        template <typename U>
        void f(U&& t)    // deducing context -> fwd. ref.
          requires std::same_as<std::decay_t<U>, T> { ... }
      };
    \end{cppcode*}
  \end{exampleblock}
  \end{overprint}
\end{frame}

\begin{frame}[fragile]
  \frametitlecpp[11]{Perfect forwarding}
  \begin{block}{\texttt{std::remove\_reference}}
    \begin{itemize}
    \item Type trait to remove reference from a type
    \item If \cppinline{T} is a reference type, \cppinline{remove_reference_t<T>} is the type referred to by \cppinline{T}, otherwise it is \cppinline{T}.
    \end{itemize}
  \end{block}
  \begin{block}{}
    \begin{cppcode*}{}
      template <typename T>
      struct remove_reference      { using type = T; };
      template <typename T>
      struct remove_reference<T&>  { using type = T; };
      template <typename T>
      struct remove_reference<T&&> { using type = T; };

      template <typename T>
      using remove_reference_t =
        typename remove_reference<T>::type; // C++14
    \end{cppcode*}
  \end{block}
\end{frame}

\begin{frame}[fragile]
  \frametitlecpp[11]{Perfect forwarding}
  \begin{block}{\texttt{std::forward}}
    \begin{itemize}
      \item Forward lvalue-/rvalueness
      \item Keeps references and maps non-references to rvalue references
    \end{itemize}
  \end{block}
  \begin{block}{}
    \small
    \begin{cppcode*}{}
      template<typename T>
      T&& forward(remove_reference_t<T>& t) noexcept {  // 1.
        return static_cast<T&&>(t);
      }
      template<typename T>
      T&& forward(remove_reference_t<T>&& t) noexcept { // 2.
        return static_cast<T&&>(t);
      }
    \end{cppcode*}
  \end{block}
  \begin{block}{}
    \begin{itemize}
    \item if \cppinline{T} is \cppinline{int}, selects 2., returns \cppinline{int&&}
    \item if \cppinline{T} is \cppinline{int&}, selects 1., returns \cppinline{int& &&}, i.e.\ \cppinline{int&}
    \item if \cppinline{T} is \cppinline{int&&}, selects 2., returns \cppinline{int&& &&}, i.e.\ \cppinline{int&&}
    \end{itemize}
  \end{block}
\end{frame}

\begin{frame}[fragile]
  \frametitlecpp[11]{Perfect forwarding}
    \begin{exampleblock}{Example - putting it all together}
    \begin{cppcode}
      template <typename... T>
      void wrapper(T&&... args) {
        func(std::forward<T>(args)...);
      }
    \end{cppcode}
    \end{exampleblock}
  \begin{block}{}
    \begin{itemize}
    \item if we pass an rvalue reference \cppinline{U&&} to wrapper
      \begin{itemize}
      \item \cppinline{T=U}, arg is of type \cppinline{U&&}
      \item func will be called with a \cppinline{U&&}
      \end{itemize}
    \item if we pass an lvalue reference \cppinline{U&} to wrapper
      \begin{itemize}
      \item \cppinline{T=U&}, arg is of type \cppinline{U&} (reference collapsing)
      \item func will be called with a \cppinline{U&}
      \end{itemize}
    \item if we pass a plain \cppinline{U} (rvalue) to wrapper
      \begin{itemize}
      \item \cppinline{T=U}, arg is of type \cppinline{U&&} (no copy in wrapper)
      \item func will be called with a \cppinline{U&&}
      \item if func takes a plain \cppinline{U}, copy happens there, as expected
      \end{itemize}
    \end{itemize}
  \end{block}
\end{frame}

\begin{frame}[fragile]
  \frametitlecpp[11]{Real life example}
  \begin{block}{}
    \begin{cppcode*}{}
      template<typename T, typename... Args>
      unique_ptr<T> make_unique(Args&&... args) {
        return unique_ptr<T>
          (new T(std::forward<Args>(args)...));
      }
    \end{cppcode*}
  \end{block}
\end{frame}
