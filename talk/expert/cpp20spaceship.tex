\subsection[spaceship]{The \texttt{<=>} operator}

\begin{frame}[fragile]
  \frametitle{The burden of comparison operators}
  \begin{block}{Motivation}
    \begin{itemize}
    \item One often needs \mintinline{cpp}{operator<} for a user-defined class
      \begin{itemize}
      \item e.g.\ when sorting a \mintinline{cpp}{std::vector}
      \item e.g.\ when using it as a key for \mintinline{cpp}{std::set} or \mintinline{cpp}{std::map}
      \end{itemize}
    \item Many other operators are also desirable for completeness
      \begin{itemize}
      \item \mintinline{cpp}{operator>}, \mintinline{cpp}{operator>=}, \mintinline{cpp}{operator<=}, ...
      \item often implemented reusing e.g.\ \mintinline{cpp}{operator<} and \mintinline{cpp}{operator==}
      \end{itemize}
    \item Should be defined as hidden friend functions
    \item Much boilerplate code to write. Too much...
    \end{itemize}
  \end{block}
\end{frame}

\begin{frame}[fragile]
  \frametitlecpp[20]{The three-way comparison operator}
  \begin{block}{Idea}
    \begin{itemize}
    \item C++20 introduces \mintinline{cpp}{operator<=>}
    \item named the \textbf{three-way comparison operator}.
      \begin{itemize}
      \item unofficially called \textbf{spaceship operator}
      \end{itemize}
    \item allowing to implement all comparisons in one go
    \end{itemize}
  \end{block}
  \begin{exampleblock}{How it works}
    \begin{itemize}
    \item it returns \emph{something} which can be compared to \mintinline{cpp}{0}
      \begin{itemize}
      \item similar to \mintinline{cpp}{std::strcmp}
      \end{itemize}
    \item lower, greater, or equal to \mintinline{cpp}{0} means respectively \emph{lower than}, \emph{greater than} and \emph{equivalent to}
    \item It is provided by default for all built-in types and many types in the standard library
    \end{itemize}
  \end{exampleblock}
\end{frame}

\begin{frame}[fragile]
  \frametitlecpp[20]{The three-way comparison operator practically}
  \begin{exampleblock}{}
    \begin{cppcode*}{}
    template <typename T>
    void three_way_compare( T lhs, T rhs ) {
      auto res = lhs <=> rhs;
      std::cout << lhs << "<=>" << rhs << ": "
                << (res<0) << (res==0) << (res>0)
                << '\n';
    }
    int main() {
      three_way_compare(1, 2); // 1<=>2: 100
      three_way_compare(2, 2); // 2<=>2: 010
      three_way_compare(2, 1); // 2<=>1: 001
    }
    \end{cppcode*}
  \end{exampleblock}
\end{frame}

\begin{frame}[fragile]
  \frametitlecpp[20]{Different kinds of ordering}
  \begin{block}{The return type of \texttt{operator<=>}}
    \begin{itemize}
    \item for integers \mintinline{cpp}{operator<=>} returns \mintinline{cpp}{std::strong_ordering}
    \item \mintinline{cpp}{weak_ordering} and \mintinline{cpp}{partial_ordering} also exist
    \end{itemize}
  \end{block}
  \begin{exampleblock}{3 types of ordering}
    \begin{description}[partial]
    \item[strong] exactly one test among \mintinline{cpp}{<0}, \mintinline{cpp}{==0}, and \mintinline{cpp}{>0} will return \mintinline{cpp}{true}
    \item[weak] like strong but two \emph{equivalent} values may differ
      \begin{itemize}
      \item they are however \emph{equivalent} for ranking
      \item e.g.\ rational numbers $2/3$ and $4/6$
      \end{itemize}
    \item[partial] like weak but some values are incomparable
      \begin{itemize}
      \item for some values all tests may return \mintinline{cpp}{false}
      \item e.g.\ compare a floating point to \mintinline{cpp}{NaN}
      \end{itemize}
    \end{description}
  \end{exampleblock}
\end{frame}

\begin{frame}[fragile]
  \frametitlecpp[20]{Exercising different orderings}
  \scriptsize
  \begin{exampleblock}{Example}
    \begin{cppcode*}{}
    struct Ratio {
      unsigned n, d ;
      friend std::weak_ordering operator<=>( Ratio const & a,
                                             Ratio const & b ) {
        return (a.n * b.d) <=> (a.d * b.n);
      }
      friend std::ostream & operator<<( std::ostream & os, Ratio const & r ) {
        return (os << r.n << '/' << r.d);
      }
    };
    int main() {
      // Ratio uses weak_ordering
      three_way_compare(Ratio{1, 2}, Ratio{2, 3}); // 1/2<=>2/3 : 100
      three_way_compare(Ratio{2, 3}, Ratio{4, 6}); // 2/3<=>4/6 : 010
      three_way_compare(Ratio{2, 3}, Ratio{1, 2}); // 2/3<=>1/2 : 001

      // floats use partial_ordering
      three_way_compare(+0., -0.);  // 0<=>-0   : 010
      three_way_compare(0., 1./0.); // 0<=>inf  : 100
      three_way_compare(0., 0./0.); // 0<=>-nan : 000
    }
    \end{cppcode*}
  \end{exampleblock}
\end{frame}

\begin{frame}[fragile]
  \frametitlecpp[20]{Compiler-generated comparison operators}
  \begin{block}{For a given user-defined class}
    \begin{itemize}
    \item defining \mintinline{cpp}{operator<=>} allows the compiler to use it when encountering the comparison operators \mintinline{cpp}{<},  \mintinline{cpp}{<=}, \mintinline{cpp}{>} and \mintinline{cpp}{>=}
    \item of course, one can still provide a custom implementation
    \item the compiler will \emph{NOT} add a default implementation for \mintinline{cpp}{operator==} and \mintinline{cpp}{operator!=}
      \begin{itemize}
      \item as those operators mean \textbf{equal}, rather than \textbf{equivalent}
      \item if \mintinline{cpp}{operator<=>} does not provide a strong order, it is advised not to define \mintinline{cpp}{operator==}
      \end{itemize}
    \end{itemize}
  \end{block}
\end{frame}

\begin{frame}[fragile]
  \frametitlecpp[20]{Compiler-generated \texttt{operator<=>}}
  \begin{block}{Default \texttt{operator<=>} implementation}
    \begin{itemize}
    \item One can ask the compiler to provide a default implementation for \mintinline{cpp}{operator<=>} and/or \mintinline{cpp}{operator==}
      \begin{itemize}
        \item declaring them with ``\mintinline{cpp}{= default}''
        \item it will compare the member variables, one by one
      \end{itemize}
    \item Can be useful e.g.\ for tuples
    \item Can be wrong, e.g.\ for the Ratio class
    \item If \mintinline{cpp}{operator<=>} is defaulted and no \mintinline{cpp}{operator==} is defined, then the compiler also provides \mintinline{cpp}{operator==}
    \end{itemize}
  \end{block}
\end{frame}

\begin{frame}[fragile]
  \frametitlecpp[20]{\texttt{operator<=>} summary}
  \begin{block}{Summary}
    \begin{itemize}
      \item Defining \mintinline{cpp}{operator<=>} allows you to use \mintinline{cpp}{operator<}, \mintinline{cpp}{operator>}, \mintinline{cpp}{operator<=}, and \mintinline{cpp}{operator>=} for free
      \item The standard library defines a few kinds of orderings
        \begin{itemize}
          \item strong, weak and partial
        \end{itemize}
      \item If \mintinline{cpp}{operator<=>} does not define a strong ordering, avoid defining \mintinline{cpp}{operator==}
    \end{itemize}
  \end{block}
\end{frame}
