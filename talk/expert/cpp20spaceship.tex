\subsection[spaceship]{The three-way comparison operator}

\begin{frame}[fragile]
  \frametitlecpp[20]{The three-way comparison operator}
  \begin{block}{Motivation}
    \begin{itemize}
    \item There are frequent situations where one needs {\it <} operator for a home-made class. Typically, if you want to make a {\it std::vector} of such objects, and sort it. But also if you want to use it as a key for e.g. {\it std::set} or {\it std::map}.
    \item For completeness, one should also add {\it >}, {\it >=}, and {\it <=}, implemented reusing either {\it <} and {\it ==}, or {\it <} and {\it >}.
    \item Those operators should be defined as free functions, optionally friends, so that left and right arguments will be similarly convertible.
    \item Much boilerplate code to write. Too much..
    \end{itemize}
  \end{block}
\end{frame}

\begin{frame}[fragile]
  \frametitlecpp[20]{The three-way comparison operator}
  \begin{block}{Idea}
    \begin{itemize}
    \item C++20 introduces the \textbf{spaceship} operator: {\it <=>}. Well, the real official name is \textbf{three-way comparison operator}.
    \item It is provided by default for all predefined types, and return \textbf{something} which can be compared to {\it 0} (such as the {\it std::strcmp}), meaning \textbf{lower than} if this \textbf{something} is lower than {\it 0}, \textbf{greater than} if it is greater than {\it 0}, and \textbf{equivalent} if it is equal {\it 0}:
    \end{itemize}
  \end{block}
  \begin{exampleblock}{Example}
    \begin{cppcode*}{}
#include <iostream>

template <typename T>
void three_way_compare( T lhs, T rhs )
 {
  auto result = (lhs<=>rhs) ;
  std::cout
    <<lhs<<" <=> "<<rhs<<": "<<'<'
    <<(result<0)<<'|'<<(result==0)<<'|'<<(result>0)
    <<'>'<<std::endl ;
 }

int main()
 {
  three_way_compare(1,2) ;
  three_way_compare(2,2) ;
  three_way_compare(2,1) ;
 }
    \end{cppcode*}
  \end{exampleblock}
\end{frame}

\begin{frame}[fragile]
  \frametitlecpp[20]{The three-way comparison operator}
  \begin{block}{Different kinds of ordering}
    \begin{itemize}
    \item The real return type of {\it <=>} for integers is {\it std::strong_ordering} : whatever the values, you will always get {\it true} for exactly one test among {\it <0}, {\it ==0}, and {\it >0}.
    \item On the contrary, the return type of {\it <=>} for floating point numbers is {\it std::partial_ordering}, because sometimes the tree tests may return {\it false}, typically if one one the number is {\it NaN}.
    \end{itemize}
  \end{block}
  \begin{exampleblock}{Example}
    \begin{cppcode*}{}
#include <iostream>

template <typename T>
void three_way_compare( T lhs, T rhs )
 {
  auto result = (lhs<=>rhs) ;
  std::cout
    <<lhs<<" <=> "<<rhs<<": "<<'<'
    <<(result<0)<<'|'<<(result==0)<<'|'<<(result>0)
    <<'>'<<std::endl ;
 }

int main()
 {
  three_way_compare(+0.,-0.) ;
  three_way_compare(0./1.,1./0.) ;
  three_way_compare(0.,0./0.) ;
 }
    \end{cppcode*}
  \end{exampleblock}
\end{frame}

\begin{frame}[fragile]
  \frametitlecpp[20]{The three-way comparison operator}
  \begin{block}{Different kinds of ordering}
    \begin{itemize}
      \item The type {\it std::strong_ordering} is convertible to {\it std::partial_ordering}, but not the contrary.
      \item Between the two, we have a class {\it std::weak_ordering}, where {\it ==0} mean that the two compared values are equivalent from a ranking point of view, but not necessarily \textbf{equal}. In a given expression, one cannot substitutes one value for the other and be sure to have the same result.
      \item I am not aware of some predefined type whose {\it <=>} would return an instance of {\it std::weak_ordering}, but it may make sense for some home-made class, such as the following.
    \end{itemize}
  \end{block}
\end{frame}

\begin{frame}[fragile]
  \frametitlecpp[20]{The three-way comparison operator}
  \begin{block}{Home-made class}
    \begin{itemize}
    \item In the example below, we define a very basic class for positive rational numbers, and provide an implementation of {\it <=>}.
    \end{itemize}
  \end{block}
  \begin{exampleblock}{Example}
    \begin{cppcode*}{}
#include <iostream>
#include <compare>

struct Rational
 {
  unsigned n, d ;

  std::weak_ordering operator<=>( Rational const & other ) const
   { return (n*other.d)<=>(other.n*d) ; }
 } ;

std::ostream & operator<<( std::ostream & os, std::weak_ordering cmp )
 { return (os<<'<'<<(cmp<0)<<'|'<<(cmp==0)<<'|'<<(cmp>0)<<'>') ; }

std::ostream & operator<<( std::ostream & os, Rational const & r )
 { return (os<<r.n<<'/'<<r.d) ; }

template <typename T>
void compare( T lhs, T rhs )
 {
  std::cout<<std::endl ;
  //std::cout<<lhs<<" ==  "<<rhs<<": "<<(lhs==rhs)<<std::endl ;
  //std::cout<<lhs<<" !=  "<<rhs<<": "<<(lhs!=rhs)<<std::endl ;
  std::cout<<lhs<<" <=> "<<rhs<<": "<<(lhs<=>rhs)<<std::endl ;
  std::cout<<lhs<<" <   "<<rhs<<": "<<(lhs<  rhs)<<std::endl ;
  std::cout<<lhs<<" <=  "<<rhs<<": "<<(lhs<= rhs)<<std::endl ;
  std::cout<<lhs<<" >   "<<rhs<<": "<<(lhs>  rhs)<<std::endl ;
  std::cout<<lhs<<" >=  "<<rhs<<": "<<(lhs>= rhs)<<std::endl ;
 }

int main()
 {
  compare<Rational>({ 3, 4 },{ 2, 3 }) ;
  compare<Rational>({ 3, 6 },{ 2, 3 }) ;
  compare<Rational>({ 1, 2 },{ 2, 4 }) ;
 }
     \end{cppcode*}
  \end{exampleblock}
\end{frame}

\begin{frame}[fragile]
  \frametitlecpp[20]{The three-way comparison operator}
  \begin{block}{Home-made class}
    \begin{itemize}
      \item Despites the {\it <=>} for {\it unsigned} returns some instance of {\it std::strong_ordering}, we prefer here to cast it to {\it std::weak_ordering}. This way, we emphasize that if {\it a<=>b} is equal to {\it 0}, it only means that {\it a} and {\it b} are logically equivalent, but may lead to different results in some expressions.
      \item As one can check in the code, we have defined {\it <=>}, and the compiler has added, for free, some default implementations for {\it <}, {\it >}, {\it <=} and {\it =>}. Of course, one can overload them with home-made implementations.
      \item It has NOT added a default implementation for {\it ==} and {\it !=}. Those operators are generally expected to mean \textbf{equal}, rather than \textbf{equivalent}. If {\it <=>} does not enable a strong order, the operator {\it ==}generally deserves a separate implementation.
    \end{itemize}
  \end{block}
\end{frame}

\begin{frame}[fragile]
  \frametitlecpp[20]{The three-way comparison operator}
  \begin{block}{Default {\it <=>} implementation}
    \begin{itemize}
      \item One can ask the compiler to provide some default implementation for {\it <=>}. Logically enough, it will compare the first member variable of the two objects, and goes on to the next member variable as long as the current ones are equivalent.
      \item In the previous example, that would not be relevant, because it will compare the numerators first, and conclude that {\it 3/6} is greater than {\it 2/3}:
    \end{itemize}
  \end{block}
  \begin{exampleblock}{Example}
    \begin{cppcode*}{}
#include <iostream>

struct Rational
 {
  unsigned n, d ;
  std::strong_ordering operator<=>( Rational const & other ) const = default ;
 } ;

std::ostream & operator<<( std::ostream & os, std::weak_ordering cmp )
 { return (os<<'<'<<(cmp<0)<<'|'<<(cmp==0)<<'|'<<(cmp>0)<<'>') ; }

std::ostream & operator<<( std::ostream & os, Rational const & r )
 { return (os<<r.n<<'/'<<r.d) ; }

int main()
 {
  Rational r1 { 3, 6 }, r2 { 2, 3 } ;
  std::cout<<r1<<" <=> "<<r2<<": "<<(r1<=>r2)<<std::endl ;
 }
    \end{cppcode*}
  \end{exampleblock}
\end{frame}

\begin{frame}[fragile]
  \frametitlecpp[20]{The three-way comparison operator}
  \begin{block}{Default {\it <=>} implementation}
    \begin{itemize}
    \item On the contrary, for some tuple-like class, it may make sense.
    \end{itemize}
  \end{block}
  \begin{exampleblock}{Example}
    \begin{cppcode*}{}
#include <iostream>
#include <string>
#include <set>

struct Grade
 {
  double number ;
  char letter ;
  std::string name ;

  auto operator<=>( Grade const & other ) const = default ;
 } ;

std::ostream & operator<<( std::ostream & os, Grade const & g )
 { return (os<<g.letter<<", "<<g.number<<", "<<g.name) ; }

int main()
 {
  std::set<Grade> grades
   {
     { 19, 'A', "Djamila" },
     { 12, 'C', "Charles" },
     { 16.5, 'A', "Marc" },
   } ;

  for ( auto const & grade : grades )
   { std::cout<<grade<<std::endl ; }
 }
    \end{cppcode*}
  \end{exampleblock}
\end{frame}

\begin{frame}[fragile]
  \frametitlecpp[20]{The three-way comparison operator}
  \begin{block}{Default {\it <=>} implementation}
    \begin{itemize}
      \item We see above that the definition of {\it <=>} has been completed by the compiler with a default implementation of {\it <} (among others), which has been used by {\it std::set} so to sorts its elements.
      \item Worth to note : if you use the \textbf{default} implementation of {\it <=>}, you will also get additional {\it ==} and {\it !=}. Not sure this special behavior is such a good idea...
          \end{itemize}
  \end{block}
\end{frame}

\begin{frame}[fragile]
    \frametitlecpp[20]{The three-way comparison operator}
    \begin{block}{What to retain}
      \begin{itemize}
        \item Defining {\it <=>} brings you a free default implementation of {\it <}, {\it >}, {\it <=}, and {\it >=}.
        \item The standard library defines few different kinds of order (strong, weak and partial).
        \item Do not confuse equivalence ({\it (a<=>b)==0}) with equality ({\it a==b}).      \end{itemize}
    \end{block}
\end{frame}
