\section[exp]{Expert \cpp}

\subsection[tmpl]{Variadic templates}

%http://eli.thegreenplace.net/2014/variadic-templates-in-c/

\begin{frame}[fragile]
  \frametitleii{Basic variadic template}
  \begin{block}{The idea}
    \begin{itemize}
    \item a template with an arbitrary number of parameters
    \item ... syntax as in good old printf
    \item using recursivity and specialization for stopping it
    \end{itemize}
  \end{block}
  \begin{exampleblock}{Practically}
    \begin{cppcode*}{}
      template<typename T>
      T adder(T v) {
        return v;
      }
      template<typename T, typename... Args>
      T adder(T first, Args... args) {
        return first + adder(args...);
      }
      long sum = adder(1, 2, 3, 8, 7);
    \end{cppcode*}
  \end{exampleblock}
\end{frame}

\begin{frame}
  \frametitleii{A couple of remarks}
  \begin{block}{About performance}
    \begin{itemize}
    \item do not be afraid of recursion
    \item everything is at compile time !
    \item unlike C-style variadic functions
    \end{itemize}
  \end{block}
  \begin{block}{Why is it better than variadic functions}
    \begin{itemize}
    \item it's more performant
    \item type safety is included
    \item it applies to everything, including objects
    \end{itemize}    
  \end{block}
\end{frame}
  
\begin{frame}[fragile]
  \frametitleii{Variadic templated class}
  \begin{block}{The tuple example, simplified}
    \begin{cppcode*}{}
      template <class... Ts> struct tuple {};
      
      template <class T, class... Ts>
      struct tuple<T, Ts...> : tuple<Ts...> {
        tuple(T t, Ts... ts) :
          tuple<Ts...>(ts...), m_tail(t) {}
        T m_tail;
      };

      template <> struct tuple<>{};
      
      tuple<double, uint64_t, const char*>
        t1(12.2, 42, "big");
    \end{cppcode*}
  \end{block}
\end{frame}

\subsection[forward]{Perfect forwarding}

%http://eli.thegreenplace.net/2014/perfect-forwarding-and-universal-references-in-c/
\begin{frame}[fragile]
  \frametitleii{The problem}
  Trying to write a generic wrapper function
  \begin{cppcode*}{}
    template <typename T>
    void wrapper(T arg) {
      func(arg);
    }
  \end{cppcode*}
  Example usage :
  \begin{itemize}
  \item emplace\_back
  \item make\_unique
  \end{itemize}
\end{frame}

\begin{frame}[fragile]
  \frametitleii{Why is it not so simple?}
  \begin{cppcode*}{}
    template <typename T>
    void wrapper(T arg) {
      func(arg);
    }
  \end{cppcode*}
  \begin{alertblock}{What about references ?}
    what if func takes references to avoid copies ?\\
    wrapper would force a copy and we fail to use references
  \end{alertblock}
\end{frame}

\begin{frame}[fragile]
  \frametitleii{Second try, second failure ?}
  \begin{cppcode*}{}
    template <typename T>
    void wrapper(T &arg) {
      func(arg);
    }
    wrapper(42);
    // invalid initialization of
    // non-const reference from
    // an rvalue
  \end{cppcode*}
  \begin{alertblock}{}
    const ref won't help : you may want to pass something non const\\
    and rvalue are not yet supported...
  \end{alertblock}
\end{frame}

\begin{frame}[fragile]
  \frametitleii{The solution : cover all cases}
  \begin{cppcode*}{}
    template <typename T>
    void wrapper(T& arg) { func(arg); }

    template <typename T>
    void wrapper(const T& arg) { func(arg); }

    template <typename T>
    void wrapper(T&& arg) { func(arg); }
  \end{cppcode*}
\end{frame}

\begin{frame}[fragile]
  \frametitleii{The new problem : scaling to n arguments}
  \begin{cppcode*}{}
    template <typename T1, typename T2>
    void wrapper(T1& arg1, T2& arg2)
    { func(arg1, arg2); }

    template <typename T1, typename T2>
    void wrapper(const T1& arg1, T2& arg2)
    { func(arg1, arg2); }
    
    template <typename T1, typename T2>
    void wrapper(T1& arg1, const T2& arg2)
    { func(arg1, arg2); }
    ...
  \end{cppcode*}
  \begin{alertblock}{Exploding complexity}
    3$^{n}$ complexity\\
    you do not want to try n = 5...
  \end{alertblock}
\end{frame}

\begin{frame}[fragile]
  \frametitleii{Reference collapsing in \cpp98}
  \begin{block}{Reference to references}
    They are forbidden in \cpp\\
    But still may happen
    \begin{cppcode*}{}
      template <typename T>
      void foo(T t) {
        T& k = t;
        ...
      }
      int ii = 4;
      foo<int&>(ii);
    \end{cppcode*}
  \end{block}
  \begin{exampleblock}{Practically}
    all compilers were collapsing the 2 references
  \end{exampleblock}
\end{frame}

\begin{frame}
  \frametitleii{Reference collapsing in \cpp11}
  \begin{block}{rvalues have been added}
    \begin{itemize}
    \item what about int\&\& \& ?
    \item and int \&\& \&\& ?
    \end{itemize}
  \end{block}
  \begin{exampleblock}{\cpp11 standardization}
    The rule is simple : \& always wins\\
    \&\& \&, \& \&\&, \& \& $\rightarrow$ \&\\
    \&\& \&\& $\rightarrow$ \&\&
  \end{exampleblock}
\end{frame}

\begin{frame}[fragile]
  \frametitleii{rvalue in type-deducing context}
  \begin{cppcode*}{}
    template <class T>
    void func(T&& t) {}
  \end{cppcode*}
  Next to a template parameter, \&\& is not an rvalue\\
  T\&\& actual type depends on the arguments passed to func
  \begin{itemize}
  \item if an lvalue of type U is given, T is deduced to U\&
  \item otherwise, collapse references normally
  \end{itemize}
  \begin{cppcode*}{firstnumber=3}
    func(4);        // rvalue -> T&& is int&&
    double d = 3.14;
    func(d);        // lvalue -> T&& is double&
    float f() {...}
    func(f());      // rvalue -> T&& is float&&
    int foo(int i) {
      func(i);      // lvalue -> T&& is int&
    }
  \end{cppcode*}
\end{frame}

\begin{frame}[fragile]
  \frametitleii{std::remove\_reference}
  Some template trickery removing reference from a type
  \begin{cppcode*}{}
    template< class T >
    struct remove_reference
    {using type = T;};

    template< class T >
    struct remove_reference<T&>
    {using type = T;};

    template< class T >
    struct remove_reference<T&&>
    {using type = T;};
  \end{cppcode*}
  If {\ttfamily T} is a reference type, {\ttfamily remove\_reference<T>::type} is the type referred to by T,
  otherwise it is T.
\end{frame}

\begin{frame}[fragile]
  \frametitleii{std::forward}
  Another template trickery keeping references and mapping non reference types to rvalue references
  \begin{cppcode*}{}
    template<class T>
    T&& forward(typename std::remove_reference<T>::type& t)
      noexcept {
      return static_cast<T&&>(t);
    }
  \end{cppcode*}
  \begin{block}{How it works}
    \begin{itemize}
    \item if T is int, it returns int \&\&
    \item if T is int\&, it returns int\& \&\& ie int\&
    \item if T is int\&\&, it returns int\&\& \&\& ie int\&\&
    \end{itemize}
  \end{block}
\end{frame}

\begin{frame}[fragile]
  \frametitleii{Perfect forwarding}
  Putting it all together
  \begin{cppcode*}{}
    template <typename... T>
    void wrapper(T&&... args) {
      func(std::forward<T>(args)...);
    }
  \end{cppcode*}
  \begin{block}{How it works}
    \begin{itemize}
    \item if we pass an rvalue to wrapper (U\&\&)
      \begin{itemize}
      \item arg will be of type U\&\&
      \item func will be called with a U\&\&
      \end{itemize}
    \item if we pass an lvalue to wrapper (U\&)
      \begin{itemize}
      \item arg will be of type U\&
      \item func will be called with a U\&
      \end{itemize}
    \item if we pass a plain value (U)
      \begin{itemize}
      \item arg will be of type U\&\& (no copy in wrapper)
      \item func will be called with a U\&\&
      \item but func takes a U, so copy happens there, as expected
      \end{itemize}
    \end{itemize}
  \end{block}  
\end{frame}

\begin{frame}[fragile]
  \frametitleii{Real life example}
  \begin{cppcode*}{}
    template<typename T, typename... Args>
    unique_ptr<T> make_unique(Args&&... args) {
      return unique_ptr<T>
        (new T(std::forward<Args>(args)...));
    }
  \end{cppcode*}  
\end{frame}

\subsection[sfinae]{SFINAE}

%https://jguegant.github.io/blogs/tech/sfinae-introduction.html
\begin{frame}[fragile]
  \frametitleii{Substitution Failure Is Not An Error (SFINAE)}
  \begin{block}{The main idea}
    \begin{itemize}
    \item substitution replaces template parameters with the provided types or values
    \item if it leads to an invalid code, do not fail but try other overloads
    \end{itemize}
  \end{block}
  \begin{exampleblock}{Example}
    \begin{cppcode*}{}
      template <typename T>
      void f(typename T::type arg) { ... }
      void f(int a) { ... }

      f(1); // Calls void f(int)
    \end{cppcode*}
  \end{exampleblock}
  Note : SFINAE will be largely superseded by concepts in \cpp20
\end{frame}

\begin{frame}[fragile]
  \frametitleii{decltype}
  \begin{block}{The main idea}
    \begin{itemize}
    \item gives the type of the expression it will evaluate
    \item at compile time
    \end{itemize}
  \end{block}
  \begin{exampleblock}{Example}
    \begin{cppcode*}{}
      struct A { double x; };
      A a;
      decltype(a.x) y;       // double
      decltype((a.x)) z = y; // double& (lvalue)
 
      template<typename T, typename U>
      auto add(T t, U u) -> decltype(t + u);
      // return type depends on template parameters
    \end{cppcode*}
  \end{exampleblock}  
\end{frame}

\begin{frame}[fragile]
  \frametitleii{declval}
  \begin{block}{The main idea}
    \begin{itemize}
    \item gives you a ``fake reference'' to an object at compile time
    \item useful for types that cannot be easily constructed
    \end{itemize}
  \end{block}
  \begin{exampleblock}{Example}
    \begin{cppcode*}{}
      struct Default {
        int foo() const { return 1; }
      };
      struct NonDefault {
        NonDefault(int i) { }
        int foo() const { return 1; }
      }; 
      decltype(Default().foo()) n1 = 1;     // int
      decltype(NonDefault().foo()) n2 = n1; // error
      decltype(std::declval<NonDefault>().foo()) n2 = n1;
    \end{cppcode*}
  \end{exampleblock}  
\end{frame}

\begin{frame}[fragile]
  \frametitleii{true\_type and false\_type}
  \begin{block}{The main idea}
    \begin{itemize}
    \item encapsulate a constexpr boolean ``true'' and ``false''
    \item can be inherited
    \item constexpr
    \end{itemize}
  \end{block}
  \begin{exampleblock}{Example}
    \begin{cppcode*}{}
      struct testStruct : std::true_type { };
      constexpr bool testVar = testStruct();
      bool test = testStruct::value; // true
    \end{cppcode*}
  \end{exampleblock}  
\end{frame}

\begin{frame}[fragile]
  \frametitleii{Using SFINAE for introspection}
  \begin{block}{The main idea}
    \begin{itemize}
    \item use a template specialization
      \begin{itemize}
      \item that may or may not create valid code
      \end{itemize}
    \item use SFINAE to choose between them
    \item inherit from true/false\_type
    \end{itemize}
  \end{block}
  \begin{exampleblock}{Example}
    \small
    \begin{cppcode*}{}
      template <typename T, typename = void>
      struct hasFoo : std::false_type {};
      template <typename T>
      struct hasFoo<T, decltype(std::declval<T>().foo())>
        : std::true_type {};
      struct A{}; struct B{void foo();};
      static_assert(!hasFoo<A>::value, "A has no Foo()");
      static_assert(hasFoo<B>::value, "B has Foo()");
    \end{cppcode*}
  \end{exampleblock}  
\end{frame}

\begin{frame}[fragile]
  \frametitleii{Not so easy actually...}
  \begin{exampleblock}{Example}
    \small
    \begin{cppcode*}{}
      template <typename T, typename = void>
      struct hasFoo : std::false_type {};
      template <typename T>
      struct hasFoo<T, decltype(std::declval<T>().foo())>
        : std::true_type {};
      
      struct A{};
      struct B{void foo();};
      struct C{int foo();};
      
      static_assert(!hasFoo<A>::value, "A has no Foo()");
      static_assert(hasFoo<B>::value, "B has Foo()");
      static_assert(!hasFoo<C>::value, "C has Foo()");
      static_assert(hasFoo<C,int>::value, "C has Foo()");
    \end{cppcode*}
  \end{exampleblock}
\end{frame}

\begin{frame}[fragile]
  \frametitleit{Using \texttt{void\_t}}
  \begin{block}{Concept}
    \begin{cppcode*}{gobble=2}
      #include <type_traits>
      template <typename... >
      using void_t = void;
    \end{cppcode*}
    \begin{itemize}
    \item Maps a sequence of given types to void
    \item Introduced in \cpp17 though trivial to copy to \cpp11
    \item Can thus be used in specialization to check the validity of an expression
    \end{itemize}
  \end{block}
\end{frame}

\begin{frame}[fragile]
  \frametitleit{Previous example using \texttt{void\_t}}
    \begin{exampleblock}{Example}
      \begin{cppcode*}{}
      template <typename T, typename = void>
      struct hasFoo : std::false_type {};

      template <typename T>
      struct hasFoo<T,
         std::void_t<decltype(std::declval<T>().foo())>>
      : std::true_type {};

      struct A{}; struct B{ void foo(); };
      struct C{ int foo(); };

      static_assert(!hasFoo<A>::value,"Unexpected Foo()");
      static_assert(hasFoo<B>::value, "expected Foo()");
      static_assert(hasFoo<C>::value, "expected Foo()");
      \end{cppcode*}
    \end{exampleblock}
\end{frame}

\begin{frame}[fragile]
  \frametitle{SFINAE and the STL \hfill \cpp11/\cpp14/\cpp17}
  \begin{block}{enable\_if / enable\_if\_t}
    \begin{cppcode*}{gobble=2}
      template<bool B, class T = void> struct enable_if {};
      template<class T>
      struct enable_if<true, T> { using type = T; };
      template< bool B, class T = void >
      using enable_if_t = typename enable_if<B,T>::type;
    \end{cppcode*}
    \begin{itemize}
    \item If B is true, has a alias \texttt{type} to type T
    \item otherwise, has no \texttt{type} alias
    \end{itemize}
  \end{block}
  \begin{block}{is\_*$<T>$/is\_*\_v$<T>$ (float/signed/object/final/abstract/...)}
    \begin{itemize}
    \item Checks whether T is ...
    \item At compile time
    \item Has member \texttt{value}, as boolean telling whether it was
    \end{itemize}
  \end{block}
\end{frame}

\begin{frame}[fragile]
  \begin{exampleblock}{Gaudi usage example}
    \begin{cppcode*}{}
      constexpr struct deref_t {
        template
          <typename In,
           typename = typename std::enable_if_t
                      <!std::is_pointer_v<In>>>
        In& operator()( In& in ) const { return in; }

        template <typename In>
        In& operator()( In* in ) const {
          assert(in!=nullptr); return *in;
        }
      } deref {};
    \end{cppcode*}
  \end{exampleblock}  
  
\end{frame}


\begin{frame}[fragile]
  \frametitleii{Back to variadic templated class}
  \begin{block}{The tuple get method}
    \begin{cppcode*}{}
      template <size_t k, typename TUPLE>
      struct elem_type_holder;

      template <class T, class... Ts>
      struct elem_type_holder<0, tuple<T, Ts...>> {
        using type = T;
      };
      
      template <size_t k, class T, class... Ts>
      struct elem_type_holder<k, tuple<T, Ts...>> {
        using type = typename elem_type_holder
           <k - 1, tuple<Ts...>>::type;
      };
    \end{cppcode*}
  \end{block}
\end{frame}

\begin{frame}[fragile]
  \frametitleii{Back to variadic templated class}
  \begin{block}{The tuple get method}
    \begin{cppcode*}{}
      template <size_t k, class... Ts>
      typename std::enable_if_t<k == 0,
        typename elem_type_holder
          <0, tuple<Ts...>>::type&>
      get(tuple<Ts...>& t) {
        return t.m_tail;
      }      
      template <size_t k, class T, class... Ts>
      typename std::enable_if_t<k != 0,
        typename elem_type_holder
           <k-1, tuple<Ts...>>::type&>
      get(tuple<T, Ts...>& t) {
        tuple<Ts...>& base = t;
        return get<k - 1>(base);
      }
    \end{cppcode*}
  \end{block}
\end{frame}
