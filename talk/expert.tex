\section[exp]{Expert C$^{++}$11/14/17}

\subsection[forward]{Perfect forwarding}

%http://eli.thegreenplace.net/2014/perfect-forwarding-and-universal-references-in-c/
\begin{frame}[fragile]
  \frametitle{The problem}
  Trying to write a generic wrapper function
  \begin{cppcode*}{}
    template <typename T>
    void wrapper(T arg) {
      func(arg);
    }
  \end{cppcode*}
  Example usage :
  \begin{itemize}
  \item emplace\_back
  \item make\_unique
  \end{itemize}
\end{frame}

\begin{frame}[fragile]
  \frametitle{Why is is not so simple ?}
  \begin{cppcode*}{}
    template <typename T>
    void wrapper(T arg) {
      func(arg);
    }
  \end{cppcode*}
  \begin{alertblock}{What about references ?}
    what if func takes references to avoid copies ?\\
    wrapper would force a copy and we fail to use references
  \end{alertblock}
\end{frame}

\begin{frame}[fragile]
  \frametitle{Second try, second failure ?}
  \begin{cppcode*}{}
    template <typename T>
    void wrapper(T &arg) {
      func(arg);
    }
    wrapper(42);
    // invalid initialization of
    // non-const reference from
    // an rvalue
  \end{cppcode*}
  \begin{alertblock}{}
    const ref won't help : you may want to pass something non const\\
    and rvalue are not yet supported...
  \end{alertblock}
\end{frame}

\begin{frame}[fragile]
  \frametitle{The solution : cover all cases}
  \begin{cppcode*}{}
    template <typename T>
    void wrapper(T& arg) { func(arg); }

    template <typename T>
    void wrapper(const T& arg) { func(arg); }

    template <typename T>
    void wrapper(T&& arg) { func(arg); }
  \end{cppcode*}
\end{frame}

\begin{frame}[fragile]
  \frametitle{The new problem : scaling to n arguments}
  \begin{cppcode*}{}
    template <typename T1, typename T2>
    void wrapper(T1& arg1, T2& arg2)
    { func(arg1, arg2); }

    template <typename T1, typename T2>
    void wrapper(const T1& arg1, T2& arg2)
    { func(arg1, arg2); }
    
    template <typename T1, typename T2>
    void wrapper(T1& arg1, const T2& arg2)
    { func(arg1, arg2); }
    ...
  \end{cppcode*}
  \begin{alertblock}{Exploding complexity}
    3$^{n}$ complexity\\
    you do not want to try n = 5...
  \end{alertblock}
\end{frame}

\begin{frame}[fragile]
  \frametitle{Reference collapsing in \cpp98}
  \begin{block}{Reference to references}
    They are forbidden in \cpp\\
    But still may happen
    \begin{cppcode*}{}
      template <typename T>
      void foo(T t) {
        T& k = t;
        ...
      }
      int ii = 4;
      foo<int&>(ii);
    \end{cppcode*}
  \end{block}
  \begin{exampleblock}{Practically}
    all compilers were collapsing the 2 references
  \end{exampleblock}
\end{frame}

\begin{frame}
  \frametitle{Reference collapsing in \cpp11}
  \begin{block}{rvalues have been added}
    \begin{itemize}
    \item what about int\&\& \& ?
    \item and int \&\& \&\& ?
    \end{itemize}
  \end{block}
  \begin{exampleblock}{\cpp11 standardization}
    The rule is simple : \& always wins\\
    \&\& \&, \& \&\&, \& \& $\rightarrow$ \&\\
    \&\& \&\& $\rightarrow$ \&\&
  \end{exampleblock}
\end{frame}

\begin{frame}[fragile]
  \frametitle{rvalue in type-deducing context}
  \begin{cppcode*}{}
    template <class T>
    void func(T&& t) {}
  \end{cppcode*}
  In this context, \&\& is not an rvalue\\
  It means that the T type depends on the arguments passed to func
  \begin{itemize}
  \item if an lvalue of type U is given, T is deduced to U\&
  \item if an rvalue, T is deduced to U
  \end{itemize}
  \begin{cppcode*}{}
    func(4);        // rvalue -> T is int
    double d = 3.14;
    func(d);        // lvalue -> T is double&
    float f() {...}
    func(f());      // rvalue -> T is float
    int foo(int i) {
      func(i);      // lvalue -> T is int&
    }
  \end{cppcode*}
\end{frame}

\begin{frame}[fragile]
  \frametitle{std::remove\_reference}
  Some template trickery removing reference from a type
  \begin{cppcode*}{}
    template< class T >
    struct remove_reference
    {typedef T type;};

    template< class T >
    struct remove_reference<T&>
    {typedef T type;};

    template< class T >
    struct remove_reference<T&&>
    {typedef T type;};
  \end{cppcode*}
  If T is a reference type, T::type is the type referred to by T.\\
  Otherwise T::type is T.
\end{frame}

\begin{frame}[fragile]
  \frametitle{std::forward}
  Another template trickery keeping references and mapping non reference types to rvalue references
  \begin{cppcode*}{}
    template<class T>
    T&& forward(typename std::remove_reference<T>::type& t)
      noexcept {
      return static_cast<T&&>(t);
    }
  \end{cppcode*}
  \begin{block}{How it works}
    \begin{itemize}
    \item if T is int, if returns int \&\&
    \item if T is int\&, it returns int\& \&\& ie int\&
    \end{itemize}
  \end{block}
\end{frame}

\begin{frame}[fragile]
  \frametitle{Perfect forwarding}
  Putting it all together
  \begin{cppcode*}{}
    template <typename T>
    void wrapper(T&& arg) {
      func(forward<T>(arg));
    }
  \end{cppcode*}
  \begin{block}{How it works}
    \begin{itemize}
    \item if we pass an rvalue to T (U\&\&)
      \begin{itemize}
      \item arg will be of type U\&\&
      \item func will be called with a U\&\&
      \end{itemize}
    \item if we pass an lvalue to T (U\&)
      \begin{itemize}
      \item arg will be of type U\&
      \item func will be called with a U\&
      \end{itemize}
    \end{itemize}
  \end{block}  
\end{frame}

\begin{frame}[fragile]
  \frametitle{Real life example}
  \begin{cppcode*}{}
    template<typename T, typename... Args>
    unique_ptr<T> make_unique(Args&&... args) {
      return unique_ptr<T>
        (new T(std::forward<Args>(args)...));
    }
  \end{cppcode*}  
\end{frame}

\subsection[tmpl]{Variadic templates}

%http://eli.thegreenplace.net/2014/variadic-templates-in-c/

\begin{frame}[fragile]
  \frametitle{Basic variadic template}
  \begin{block}{The idea}
    \begin{itemize}
    \item a template with an arbitrary number of parameters
    \item ... syntax as in good old printf
    \item using recursivity and specialization for stopping it
    \end{itemize}
  \end{block}
  \begin{exampleblock}{Practically}
    \begin{cppcode*}{}
      template<typename T, typename... Args>
      T adder(T first, Args... args) {
        return first + adder(args...);
      }
      template<typename T>
      T adder(T v) {
        return v;
      }
      long sum = adder(1, 2, 3, 8, 7);
    \end{cppcode*}
  \end{exampleblock}
\end{frame}

\begin{frame}
  \frametitle{A couple of remarks}
  \begin{block}{About performance}
    \begin{itemize}
    \item do not be afraid by recursivity
    \item everything is at compile time !
    \item unlike C style variadic functions
    \end{itemize}
  \end{block}
  \begin{block}{Why is it better than variadic functions}
    \begin{itemize}
    \item it's more performant
    \item type safety is included
    \item it applies to everything, including objects
    \end{itemize}    
  \end{block}
\end{frame}
  
\begin{frame}[fragile]
  \frametitle{Variadic templated class}
  \begin{block}{The tuple example}
    \begin{cppcode*}{}
      template <class... Ts> struct tuple {};
      
      template <class T, class... Ts>
      struct tuple<T, Ts...> : tuple<Ts...> {
        tuple(T t, Ts... ts) :
          tuple<Ts...>(ts...), m_tail(t) {}
        T m_tail;
      };
      
      tuple<double, uint64_t, const char*>
        t1(12.2, 42, "big");
    \end{cppcode*}
  \end{block}
\end{frame}


\subsection[snifae]{SFINAE}

%https://jguegant.github.io/blogs/tech/sfinae-introduction.html
\begin{frame}[fragile]
  \frametitle{Substitution Failure Is Not An Error}
  \begin{block}{The main idea}
    \begin{itemize}
    \item substitution replaces template parameters with the provided types or values
    \item if it leads to an invalid code, do not fail but try other overloads
    \end{itemize}
  \end{block}
  \begin{exampleblock}{Example}
    \begin{cppcode*}{}
      template <typename T>
      void f(const T& t,
             typename T::iterator* it = nullptr) { }
      void f(...) { }   // ``sink'' function

      f(1); // Calls void f(...)
    \end{cppcode*}
  \end{exampleblock}
\end{frame}

\begin{frame}[fragile]
  \frametitle{decltype}
  \begin{block}{The main idea}
    \begin{itemize}
    \item gives the type of the of the expression it will evaluate
    \item at compile time
    \end{itemize}
  \end{block}
  \begin{exampleblock}{Example}
    \begin{cppcode*}{}
      struct A { double x; };
      A a;
      decltype(a.x) y;       // double
      decltype((a.x)) z = y; // double& (lvalue)
 
      template<typename T, typename U>
      auto add(T t, U u) -> decltype(t + u);
      // return type depends on template parameters
    \end{cppcode*}
  \end{exampleblock}  
\end{frame}

\begin{frame}[fragile]
  \frametitle{declval}
  \begin{block}{The main idea}
    \begin{itemize}
    \item gives you a "fake reference" to an object at compile time
    \item useful for types that cannot be easily constructed
    \end{itemize}
  \end{block}
  \begin{exampleblock}{Example}
    \begin{cppcode*}{}
      struct Default {
        int foo() const { return 1; }
      };
      struct NonDefault {
        NonDefault(int i) { }
        int foo() const { return 1; }
      }; 
      decltype(Default().foo()) n1 = 1;     // int
      decltype(NonDefault().foo()) n2 = n1; // error
      decltype(std::declval<NonDefault>().foo()) n2 = n1;
    \end{cppcode*}
  \end{exampleblock}  
\end{frame}

\begin{frame}[fragile]
  \frametitle{true\_type and false\_type}
  \begin{block}{The main idea}
    \begin{itemize}
    \item encapsulate a constexpr boolean ``true'' and ``false''
    \item can be inherited
    \item constexpr
    \end{itemize}
  \end{block}
  \begin{exampleblock}{Example}
    \begin{cppcode*}{}
      struct testStruct : std::true_type { };
      constexpr bool testVar = testStruct();
      bool test = testStruct::value; // true
    \end{cppcode*}
  \end{exampleblock}  
\end{frame}

\begin{frame}[fragile]
  \frametitle{Using SFINAE for instrospection}
  \begin{block}{The main idea}
    \begin{itemize}
    \item use a template specialization that may or may not create valid code
    \item use SFINAE to choose between them
    \item inherit from true/false\_type
    \end{itemize}
  \end{block}
  \begin{exampleblock}{Example}
    \begin{cppcode*}{}
      template <typename T, typename = void>
      struct hasFoo : std::false_type {};

      template <typename T>
      struct hasFoo<T, decltype(std::declval<T>().foo())>
        : std::true_type {};

      std::cout << hasFoo<MyType>::value << std::endl;
    \end{cppcode*}
  \end{exampleblock}  
\end{frame}

\begin{frame}[fragile]
  \frametitle{SFINAE and the STL}
  Lot's of useful stuff there
  \begin{block}{enable\_if}
    \begin{cppcode*}{}
      template<bool B, class T = void>
      struct enable_if {};
      template<class T>
      struct enable_if<true, T> { typedef T type; };
    \end{cppcode*}
    \begin{itemize}
    \item If B is true, has a typedef \texttt{type} to type T
    \item otherwise, has no \texttt{type} typedef
    \end{itemize}
  \end{block}
  \begin{block}{is\_*$<T>$ (float/signed/object/final/abstract/...)}
    \begin{itemize}
    \item Checks whether T is ...
    \item At compile time
    \item Has member \texttt{value}, as boolean telling whether it was
    \end{itemize}
  \end{block}
\end{frame}

\begin{frame}[fragile]
  \begin{exampleblock}{Gaudi usage example}
    \begin{cppcode*}{}
      constexpr struct deref_t {
        template
          <typename In,
           typename = typename std::enable_if
                      <!std::is_pointer<In>::value>::type>
        In& operator()( In& in ) const { return in; }

        template <typename In>
        In& operator()( In* in ) const {
          assert(in!=nullptr); return *in;
        }
      } deref {};
    \end{cppcode*}
  \end{exampleblock}  
  
\end{frame}


\begin{frame}[fragile]
  \frametitle{Back to variadic templated class}
  \begin{block}{The tuple get method}
    \begin{cppcode*}{}
      template <class T, class... Ts>
      struct elem_type_holder<0, tuple<T, Ts...>> {
        typedef T type;
      };
      
      template <size_t k, class T, class... Ts>
      struct elem_type_holder<k, tuple<T, Ts...>> {
        typedef typename elem_type_holder
           <k - 1, tuple<Ts...>>::type type;
      };
    \end{cppcode*}
  \end{block}
\end{frame}

\begin{frame}[fragile]
  \frametitle{Back to variadic templated class}
  \begin{block}{The tuple get method}
    \begin{cppcode*}{}
      template <size_t k, class... Ts>
      typename std::enable_if<k == 0,
        typename elem_type_holder
          <0, tuple<Ts...>>::type&>::type
      get(tuple<Ts...>& t) {
        return t.m_tail;
      }      
      template <size_t k, class T, class... Ts>
      typename std::enable_if<k != 0,
        typename elem_type_holder
           <k-1, tuple<Ts...>>::type&>::type
      get(tuple<T, Ts...>& t) {
        tuple<Ts...>& base = t;
        return get<k - 1>(base);
      }
    \end{cppcode*}
  \end{block}
\end{frame}

