\subsection[Op]{Operators}

\begin{frame}[fragile]
  \frametitlecpp[98]{Operators' example}
  \begin{cppcode}
    struct Complex {
      float m_real, m_imaginary;
      Complex(float real, float imaginary);
      Complex operator+(const Complex& other) {
        return Complex(m_real + other.m_real,
                       m_imaginary + other.m_imaginary);
      }
    };

    Complex c1{2, 3}, c2{4, 5};
    Complex c3 = c1 + c2; // (6, 8)
  \end{cppcode}
\end{frame}

\begin{frame}
  \frametitlecpp[98]{Operators}
  \begin{block}{Defining operators of a class}
    \begin{itemize}
    \item implemented as a regular method
      \begin{itemize}
      \item either inside the class, as a member function
      \item or outside the class (not all)
      \end{itemize}
    \item with a special name (replace @ by anything)
      \begin{tabular}{llll}
        Expression & As member & As non-member \\
        \hline
        @a & (a).operator@() & operator@(a) \\
        a@b & (a).operator@(b) & operator@(a,b) \\
        a=b & (a).operator=(b) & cannot be non-member \\
        a(b...) & (a).operator()(b...) & cannot be non-member \\
        a[b] & (a).operator[](b) & cannot be non-member \\
        a-\textgreater & (a).operator-\textgreater() & cannot be non-member \\
        a@ & (a).operator@(0) & operator@(a,0) \\
      \end{tabular}
    \end{itemize}
  \end{block}
\end{frame}

\begin{frame}[fragile]
  \frametitlecpp[98]{Why to have non-member operators?}
  \begin{block}{Symmetry}
    \begin{cppcode}
      struct Complex {
        float m_real, m_imaginary;
        Complex operator+(float other) {
          return Complex(m_real + other, m_imaginary);
        }
      };
      Complex c1{2.f, 3.f};
      Complex c2 = c1 + 4.f;  // ok
      Complex c3 = 4.f + c1;  // not ok !!
    \end{cppcode}
    \pause
    \begin{cppcode*}{firstnumber=10}
      Complex operator+(float a, const Complex& obj) {
        return Complex(a + obj.m_real, obj.m_imaginary);
      }
    \end{cppcode*}
  \end{block}
\end{frame}

\begin{frame}[fragile]
  \frametitlecpp[98]{Other reason to have non-member operators?}
  \begin{block}{Extending existing classes}
    \begin{cppcode}
      struct Complex {
        float m_real, m_imaginary;
        Complex(float real, float imaginary);
      };

      std::ostream& operator<<(std::ostream& os,
                               const Complex& obj) {
        os << "(" << obj.m_real << ", "
                  << obj.m_imaginary << ")";
        return os;
      }
      Complex c1{2.f, 3.f};
      std::cout << c1 << std::endl; // Prints '(2, 3)'
    \end{cppcode}
  \end{block}
\end{frame}

\begin{frame}[fragile]
  \frametitlecpp[98]{Operators}
  \begin{alertblock}{Exercise}
    Write a simple class representing a fraction and pass all tests
    \begin{itemize}
      \item go to \texttt{code/operators}
      \item look at \texttt{operators.cpp}
      \item inspect \mintinline{cpp}{main} and complete the implementation of \mintinline{cpp}{class Fraction} step by step
      \item you can comment out parts of \mintinline{cpp}{main} to test in between
    \end{itemize}
  \end{alertblock}
\end{frame}
