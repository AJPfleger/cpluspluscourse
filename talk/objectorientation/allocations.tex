\subsection[new]{Allocating objects}

\begin{frame}[fragile]
  \frametitlecpp[98]{Process memory organization}
  \begin{block}{4 main areas}
    \begin{description}
    \item[the code segment] for the code of the executable
    \item[the data segment] for global variables
    \item[the heap] for dynamically allocated variables
    \item[the stack] for parameters of functions and local variables
    \end{description}
  \end{block}
  \hspace{2.5cm}
  \begin{tikzpicture}
    \memorystack[size x=5cm,word size=1,nb blocks=6,addresses=0]
    \memorypush{code segment}
    \memorypush{data segment}
    \memorypush{stack}
    \memorypush{...}
    \memorypush{...}
    \memorypush{heap}
    \draw[->] (stack3-1.north) ++(-1cm,0) -- +(0,.3cm);
    \draw[->] (stack3-1.north) ++(1cm,0) -- +(0,.3cm);
    \draw[->] (stack6-1.south) ++(-1cm,0) -- +(0,-.3cm);
    \draw[->] (stack6-1.south) ++(1cm,0) -- +(0,-.3cm);
  \end{tikzpicture}
\end{frame}

\begin{frame}[fragile]
  \frametitlecpp[98]{The Stack}
  \begin{block}{Main characteristics}
    \begin{itemize}
    \item allocation on the stack stays valid for the duration of the current scope.
    It is destroyed when it is popped off the stack.
    \item memory allocated on the stack is known at compile time and can thus be accessed through a variable.
    \item the stack is relatively small, it is not a good idea to allocate large arrays, structures or classes
    \item each thread in a process has its own stack
      \begin{itemize}
      \item allocations on the stack are thus ``thread private''
      \item and do not introduce any thread safety issues
      \end{itemize}
    \end{itemize}
  \end{block}
\end{frame}

\begin{frame}[fragile]
  \frametitlecpp[98]{Object allocation on the stack}
  \begin{block}{On the stack}
    \begin{itemize}
    \item objects are created when declared (constructor called)
    \item objects are destructed when out of scope (destructor is called)
    \end{itemize}
  \end{block}
  \begin{cppcode}
    int f() {
      MyFirstClass a{3}; // constructor called
      ...
    } // destructor called

    {
      MyFirstClass a; // default constructor called
      ...
    }  // destructor called
  \end{cppcode}
\end{frame}

\begin{frame}[fragile]
  \frametitlecpp[98]{The Heap}
  \begin{block}{Main characteristics}
    \begin{itemize}
    \item Allocated memory stays allocated until it is specifically deallocated
      \begin{itemize}
      \item beware memory leaks
      \end{itemize}
    \item Dynamically allocated memory must be accessed through pointers
    \item large arrays, structures, or classes should be allocated here
    \item there is a single, shared heap per process
      \begin{itemize}
      \item allows to share data between threads
      \item introduces race conditions and thread safety issues !
      \end{itemize}
    \end{itemize}
  \end{block}
\end{frame}

\begin{frame}[fragile]
  \frametitlecpp[98]{Object allocation on the heap}
  \begin{block}{On the heap}
    \begin{itemize}
    \item object are created by calling {\it new} (constructor is called)
    \item object are destructed by calling {\it delete} (destructor is called)
    \end{itemize}
  \end{block}
  \begin{cppcode}
    {
      // default constructor called
      MyFirstClass *a = new MyFirstClass;
      ...
      delete a; // destructor is called
    }

    int f() {
      // constructor called
      MyFirstClass *a = new MyFirstClass(3);
      ...
    } // memory leak !!!
  \end{cppcode}
\end{frame}

\begin{frame}[fragile]
  \frametitlecpp[98]{Array allocation on the heap}
  \begin{block}{Arrays on the heap}
    \begin{itemize}
    \item arrays of objects are created by calling {\it new[]} \\
      default constructor is called for each object of the array
    \item arrays of object are destructed by calling {\it delete[]} \\
      destructor is called for each object of the array
    \end{itemize}
  \end{block}
  \begin{cppcode}
    {
      // default constructor called 10 times
      MyFirstClass *a = new MyFirstClass[10];
      ...
      delete[] a; // destructor called 10 times
    }
  \end{cppcode}
\end{frame}
