\subsection[construct]{Constructors/destructors}

\begin{frame}[fragile]
  \frametitlecpp[98]{Class Constructors and Destructors}
  \begin{block}{Concept}
    \begin{itemize}
    \item special functions called when building/destroying an object
    \item a class can have several constructors, but only one destructor
    \item the constructors have the same name as the class
    \item same for the destructor with a leading \cppinline{~}
    \end{itemize}
  \end{block}
  \begin{multicols}{2}
    \begin{cppcode*}{gobble=2}
      class MyFirstClass {
      public:
        MyFirstClass();
        MyFirstClass(int a);
        ~MyFirstClass();
        ...
      protected:
        int a;
      };
    \end{cppcode*}
    \columnbreak
    \begin{cppcode*}{gobble=2,firstnumber=10}
      // note: special notation for
      // initialization of members
      MyFirstClass() : a(0) {}

      MyFirstClass(int a_):a(a_) {}

      ~MyFirstClass() {}
    \end{cppcode*}
  \end{multicols}
\end{frame}


\begin{frame}[fragile]
  \frametitlecpp[98]{Class Constructors and Destructors}
  \begin{cppcode}
    class Vector {
    public:
      Vector(int n);
      ~Vector();
      void setN(int n, int value);
      int getN(int n);
    private:
      int len;
      int* data;
    };
    Vector::Vector(int n) : len(n) {
      data = new int[n];
    }
    Vector::~Vector() {
      delete[] data;
    }
  \end{cppcode}
\end{frame}

\begin{frame}[fragile]
  \frametitlecpp[98]{Constructors and inheritance}
  \begin{cppcode}
    struct MyFirstClass {
      int a;
      MyFirstClass();
      MyFirstClass(int a);
    };
    struct MySecondClass : MyFirstClass {
      int b;
      MySecondClass();
      MySecondClass(int b);
      MySecondClass(int a, int b);
    };
    MySecondClass::MySecondClass() : MyFirstClass(), b(0) {}
    MySecondClass::MySecondClass(int b_)
      : MyFirstClass(), b(b_) {}
    MySecondClass::MySecondClass(int a_, int b_)
      : MyFirstClass(a_), b(b_) {}
  \end{cppcode}
\end{frame}

\begin{frame}[fragile]
  \frametitlecpp[11]{Copy constructor}
  \begin{block}{Concept}
    \begin{itemize}
    \item special constructor called for replicating an object
    \item takes a single parameter of type \cppinline{const &} to class
    \item provided by the compiler if not declared by the user
    \item in order to forbid copy, use \cppinline{= delete} (see next slides)
      \begin{itemize}
      \item or private copy constructor with no implementation in \cpp98
      \end{itemize}
    \end{itemize}
  \end{block}
  \pause
  \begin{cppcode}
    struct MySecondClass : MyFirstClass {
      MySecondClass();
      MySecondClass(const MySecondClass &other);
    };
  \end{cppcode}
  \pause
  \begin{exampleblock}{The rule of 3/5/0 (\cpp98/\cpp11 and newer) - {\color{blue!50!white} \href{https://en.cppreference.com/w/cpp/language/rule_of_three}{cppreference}}}
    \begin{itemize}
    \item if a class has a destructor, a copy/move constructor or a copy/move assignment operator, it should have all three/five. strive for having none by using RAII types as members.
    \end{itemize}
  \end{exampleblock}
\end{frame}

\begin{frame}[fragile]
  \frametitlecpp[98]{Class Constructors and Destructors}
  \begin{cppcode}
    class Vector {
    public:
      Vector(int n);
      Vector(const Vector &other);
      ~Vector();
      ...
    };
    Vector::Vector(int n) : len(n) {
      data = new int[n];
    }
    Vector::Vector(const Vector &other) : len(other.len) {
      data = new int[len];
      std::copy(other.data, other.data + len, data);
    }
    Vector::~Vector() { delete[] data; }
  \end{cppcode}
\end{frame}

\begin{frame}[fragile]
  \frametitlecpp[98]{Explicit unary constructor}
  \begin{block}{Concept}
    \begin{itemize}
    \item A constructor with a single non-default parameter can be used by the compiler for an implicit conversion.
    \end{itemize}
  \end{block}
  \begin{exampleblockGB}{Example}{https://godbolt.org/z/TvqT185fz}{Unary constructor in action}
    \begin{cppcode}
    void print(const Vector & v) {
      std::cout<<"printing v elements...\n";
    }

    int main {
      // calls Vector::Vector(int n) to construct a Vector
      // then calls print with that Vector
      print(3);
    };
    \end{cppcode}
  \end{exampleblockGB}
\end{frame}

\begin{frame}[fragile]
  \frametitlecpp[98]{Explicit unary constructor}
  \begin{block}{Concept}
    \begin{itemize}
      \item The keyword \cppinline{explicit} forbids such implicit conversions.
      \item It is recommended to use it systematically, except in special cases.
    \end{itemize}
  \end{block}
  \begin{cppcode}
    class Vector {
    public:
      explicit Vector(int n);
      Vector(const Vector &other);
      ~Vector();
      ...
    };
  \end{cppcode}
\end{frame}

\begin{frame}[fragile]
  \frametitlecpp[11]{Defaulted Constructor}
  \begin{block}{Idea}
    \begin{itemize}
    \item avoid empty default constructors like \cppinline{ClassName() {}}
    \item declare them as \cppinline{= default}
    \end{itemize}
  \end{block}
  \begin{block}{Details}
    \begin{itemize}
    \item without a user-defined constructor, a default one is provided
    \item any user-defined constructor disables the default one
    \item but the default one can be requested explicitly
    \item rule can be more subtle depending on data members
    \end{itemize}
  \end{block}
  \begin{exampleblock}{Practically}
    \begin{cppcode}
      Class() = default; // provide default if possible
      Class() = delete;  // disable default constructor
    \end{cppcode}
  \end{exampleblock}
\end{frame}

\begin{frame}[fragile]
  \frametitlecpp[11]{Delegating constructor}
  \begin{block}{Idea}
    \begin{itemize}
    \item avoid replication of code in several constructors
    \item by delegating to another constructor, in the initialization list
    \end{itemize}
  \end{block}
  \begin{exampleblock}{Practically}
    \begin{cppcode}
      struct Delegate {
        int m_i;
        Delegate(int i) : m_i(i) {
          ... complex initialization ...
        }
        Delegate() : Delegate(42) {}
      };
    \end{cppcode}
  \end{exampleblock}
\end{frame}

\begin{frame}[fragile]
  \frametitlecpp[11]{Constructor inheritance}
  \begin{block}{Idea}
    \begin{itemize}
    \item avoid having to re-declare parent's constructors
    \item by stating that we inherit all parent constructors
    \item derived class can add more constructors
    \end{itemize}
  \end{block}
  \begin{exampleblock}{Practically}
    \begin{cppcode}
      struct BaseClass {
        BaseClass(int a);
      };
      struct DerivedClass : BaseClass {
        using BaseClass::BaseClass;
        DerivedClass(int a, int b);
      };
      DerivedClass a{5};
    \end{cppcode}
  \end{exampleblock}
\end{frame}

\begin{frame}[fragile]
  \frametitlecpp[11]{Member initialization}
  \begin{block}{Idea}
    \begin{itemize}
    \item avoid redefining same default value for members n times
    \item by defining it once at member declaration time
    \end{itemize}
  \end{block}
  \begin{exampleblock}{Practically}
    \begin{cppcode}
      struct BaseClass {
        int a{5}; // also possible: int a = 5;
        BaseClass() = default;
        BaseClass(int _a) : a(_a) {}
      };
      struct DerivedClass : BaseClass {
        int b{6};
        using BaseClass::BaseClass;
      };
      DerivedClass d{7}; // a = 7, b = 6
    \end{cppcode}
  \end{exampleblock}
\end{frame}

\begin{frame}[fragile]
  \frametitlecpp[11]{Calling constructors}
  \begin{block}{After object declaration, arguments within \{\}}
    \begin{cppcode*}{gobble=2}
      struct A {
        int a;
        float b;
        A();
        A(int);
        A(int, int);
      };

      A a{1,2};    // A::A(int, int)
      A a{1};      // A::A(int)
      A a{};       // A::A()
      A a;         // A::A()
      A a = {1,2}; // A::A(int, int)
    \end{cppcode*}
  \end{block}
\end{frame}

\begin{frame}[fragile]
  \frametitlecpp[98]{Calling constructors the old way}
  \begin{block}{Arguments are given within (), aka \cpp98 nightmare}
    \begin{cppcode*}{gobble=2}
      struct A {
        int a;
        float b;
        A();
        A(int);
        A(int, int);
      };

      A a(1,2);    // A::A(int, int)
      A a(1);      // A::A(int)
      A a();       // declaration of a function !
      A a;         // A::A()
      A a = (1,2); // A::A(int), comma operator !
      A a = {1,2}; // not allowed
    \end{cppcode*}
  \end{block}
\end{frame}

\begin{frame}[fragile]
  \frametitlecpp[11]{Calling constructors for arrays and vectors}
  \begin{exampleblock}{list of items given within \{\}}
    \begin{cppcode*}{firstnumber=10}
     int ip[3]{1,2,3};
     int* ip = new int[3]{1,2,3};
     std::vector<int> v{1,2,3};
    \end{cppcode*}
  \end{exampleblock}
  \pause
  \begin{block}{\cpp98 nightmare}
    \begin{cppcode*}{firstnumber=10}
     int ip[3]{1,2,3};            // OK
     int* ip = new int[3]{1,2,3}; // not allowed
     std::vector<int> v{1,2,3};   // not allowed
    \end{cppcode*}
  \end{block}
\end{frame}
