\subsection[inherit]{Inheritance}

\begin{frame}[fragile]
  \frametitlecpp[98]{First inheritance}
  \begin{multicols}{2}
    \begin{cppcode*}{gobble=2}
      struct MyFirstClass {
        int a;
        void squareA() { a *= a; }
      };
      struct MySecondClass :
        MyFirstClass {
        int b;
        int sum() { return a + b; }
      };

      MySecondClass myObj2;
      myObj2.a = 2;
      myObj2.b = 5;
      myObj2.squareA();
      int i = myObj2.sum(); // i = 9
    \end{cppcode*}
    \columnbreak
    \raggedleft
    \null \vfill
    \begin{overprint}[0.8\columnwidth]
      \onslide<1>
      \begin{tikzpicture}[node distance=2.5cm]
        \classbox{MyFirstClass}{
          int a; \\
          void squareA();
        }
        \classbox[below of=MyFirstClass]{MySecondClass}{
          int b; \\
          int sum();
        }
        \draw[very thick,-Triangle] (MySecondClass)--(MyFirstClass);
      \end{tikzpicture}
      \onslide<2>
      \begin{tikzpicture}
        \memorystack[size x=3cm,word size=1,block size=4,nb blocks=4]
        \memorypush{a = 2}
        \memorypush{b = 5}
        \memorystruct{1}{2}{\tiny myobj2}
      \end{tikzpicture}
    \end{overprint}
    \vfill \null
  \end{multicols}
\end{frame}

\begin{frame}[fragile]
  \frametitlecpp[98]{Managing access to class members}
  \begin{block}{{\it public} \color{white} / {\it private} keywords}
    \begin{description}
      \item[private] allows access only within the class
      \item[public] allows access from anywhere
    \end{description}
    \begin{itemize}
       \item The default for \texttt{class} is {\it private}
       \item A \texttt{struct} is just a \texttt{class} that defaults to {\it public} access
    \end{itemize}
  \end{block}
  \pause
  \begin{multicols}{2}
    \begin{cppcode*}{gobble=2}
      class MyFirstClass {
      public:
        void setA(int x);
        int getA();
        void squareA();
      private:
        int a;
      };
    \end{cppcode*}
    \columnbreak
    \begin{cppcode*}{gobble=2,firstnumber=9}
      MyFirstClass obj;
      obj.a = 5;   // error !
      obj.setA(5); // ok
      obj.squareA();
      int b = obj.getA();
    \end{cppcode*}
    \pause
    \begin{tcolorbox}[left=0mm,right=0mm,top=0mm,bottom=0mm,colback=red!5!white,colframe=red!75!black]
      This breaks MySecondClass !
    \end{tcolorbox}
  \end{multicols}
\end{frame}

\begin{frame}[fragile]
  \frametitlecpp[98]{Managing access to class members(2)}
  \begin{block}{Solution is {\it protected} keyword}
    Gives access to classes inheriting from base class
  \end{block}
  \begin{multicols}{2}
    \begin{cppcode*}{gobble=2}
      class MyFirstClass {
      public:
        void setA(int a);
        int getA();
        void squareA();
      protected:
        int a;
      };
    \end{cppcode*}
    \columnbreak
    \begin{cppcode*}{gobble=2,firstnumber=13}
      class MySecondClass :
        public MyFirstClass {
      public:
        int sum() {
          return a + b;
        }
      private:
        int b;
      };
    \end{cppcode*}
  \end{multicols}
\end{frame}

\begin{frame}[fragile]
  \frametitlecpp[98]{Managing inheritance privacy}
  \begin{block}{Inheritance can be public, protected or private}
    It influences the privacy of inherited members for external code.\\
    The code of the class itself is not affected
    \begin{description}
    \item[public] privacy of inherited members remains unchanged
    \item[protected] inherited public members are seen as protected
    \item[private] all inherited members are seen as private \\
      this is the default for \texttt{class} if nothing is specified
    \end{description}
  \end{block}
  \pause
  \begin{block}{Net result for external code}
    \begin{itemize}
    \item only public members of public inheritance are accessible
    \end{itemize}
  \end{block}
  \begin{block}{Net result for grand child code}
    \begin{itemize}
    \item only public and protected members of public and protected parents are accessible
    \end{itemize}
  \end{block}
\end{frame}

\begin{frame}[fragile]
  \frametitlecpp[98]{Managing inheritance privacy - public}
  \begin{multicols}{2}
    \begin{tikzpicture}[node distance=3cm]
      \classbox{MyFirstClass}{
      private: \\
        \hspace{0.4cm}int priv; \\
      protected: \\
        \hspace{0.4cm}int prot; \\
      public: \\
        \hspace{0.4cm}int pub;
      }
      \classbox[below of=MyFirstClass]{MySecondClass}{
        void funcSecond();
      }
      \classbox[below of=MySecondClass,node distance=1.75cm]{MyThirdClass}{
        void funcThird();
      }
      \draw[very thick,-Triangle] (MySecondClass)--(MyFirstClass) node[midway,right] {public};
      \draw[very thick,-Triangle] (MyThirdClass)--(MySecondClass) node[midway,right] {public};
    \end{tikzpicture}
    \columnbreak
    \begin{cppcode*}{gobble=2}
      void funcSecond() {
        int a = priv;   // Error
        int b = prot;   // OK
        int c = pub;    // OK
      }
      void funcThird() {
        int a = priv;   // Error
        int b = prot;   // OK
        int c = pub;    // OK
      }
      void extFunc(MyThirdClass t) {
        int a = t.priv; // Error
        int b = t.prot; // Error
        int c = t.pub;  // OK
      }
    \end{cppcode*}
  \end{multicols}
\end{frame}

\begin{frame}[fragile]
  \frametitlecpp[98]{Managing inheritance privacy - protected}
  \begin{multicols}{2}
    \begin{tikzpicture}[node distance=3cm]
      \classbox{MyFirstClass}{
      private: \\
        \hspace{0.4cm}int priv; \\
      protected: \\
        \hspace{0.4cm}int prot; \\
      public: \\
        \hspace{0.4cm}int pub;
      }
      \classbox[below of=MyFirstClass]{MySecondClass}{
        void funcSecond();
      }
      \classbox[below of=MySecondClass,node distance=1.75cm]{MyThirdClass}{
        void funcThird();
      }
      \draw[very thick,-Triangle] (MySecondClass)--(MyFirstClass) node[midway,right] {protected};
      \draw[very thick,-Triangle] (MyThirdClass)--(MySecondClass) node[midway,right] {public};
    \end{tikzpicture}
    \columnbreak
    \begin{cppcode*}{gobble=2}
      void funcSecond() {
        int a = priv;   // Error
        int b = prot;   // OK
        int c = pub;    // OK
      }
      void funcThird() {
        int a = priv;   // Error
        int b = prot;   // OK
        int c = pub;    // OK
      }
      void extFunc(MyThirdClass t) {
        int a = t.priv; // Error
        int b = t.prot; // Error
        int c = t.pub;  // Error
      }
    \end{cppcode*}
  \end{multicols}
\end{frame}

\begin{frame}[fragile]
  \frametitlecpp[98]{Managing inheritance privacy - private}
  \begin{multicols}{2}
    \begin{tikzpicture}[node distance=3cm]
      \classbox{MyFirstClass}{
      private: \\
        \hspace{0.4cm}int priv; \\
      protected: \\
        \hspace{0.4cm}int prot; \\
      public: \\
        \hspace{0.4cm}int pub;
      }
      \classbox[below of=MyFirstClass]{MySecondClass}{
        void funcSecond();
      }
      \classbox[below of=MySecondClass,node distance=1.75cm]{MyThirdClass}{
        void funcThird();
      }
      \draw[very thick,-Triangle] (MySecondClass)--(MyFirstClass) node[midway,right] {private};
      \draw[very thick,-Triangle] (MyThirdClass)--(MySecondClass) node[midway,right] {public};
    \end{tikzpicture}
    \columnbreak
    \begin{cppcode*}{gobble=2}
      void funcSecond() {
        int a = priv;   // Error
        int b = prot;   // OK
        int c = pub;    // OK
      }
      void funcThird() {
        int a = priv;   // Error
        int b = prot;   // Error
        int c = pub;    // Error
      }
      void extFunc(MyThirdClass t) {
        int a = t.priv; // Error
        int b = t.prot; // Error
        int c = t.pub;  // Error
      }
    \end{cppcode*}
  \end{multicols}
\end{frame}

\begin{frame}[fragile]
  \frametitlecpp[11]{Final class}
  \begin{block}{Idea}
    \begin{itemize}
    \item make sure you cannot inherit from a given class
    \item by declaring it final
    \end{itemize}
  \end{block}
  \begin{exampleblock}{Practically}
    \begin{cppcode}
      struct Base final {
        ...
      };
      struct Derived : Base { // compiler error
        ...
      };
    \end{cppcode}
  \end{exampleblock}
\end{frame}
