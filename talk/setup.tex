%%%%%%%%%%%%%%%%%%%%%%%%%%%%%%%%%%%%%%%%%%%%%%%%%%%%%%%%%%%%%%%
% Improvement on the default split theme : added line numbers %
%%%%%%%%%%%%%%%%%%%%%%%%%%%%%%%%%%%%%%%%%%%%%%%%%%%%%%%%%%%%%%%

\setbeamercolor{frametitle}{fg=white}
\setbeamercolor{frametitle right}{fg=white}

\defbeamertemplate*{footline}{mysplit theme}
{%
  \leavevmode%
  \hbox{\begin{beamercolorbox}[wd=.5\paperwidth,ht=2.5ex,dp=1.125ex,leftskip=.3cm plus1fill,rightskip=.3cm]{author in head/foot}%
    \usebeamerfont{author in head/foot}\insertshortauthor
  \end{beamercolorbox}%
  \begin{beamercolorbox}[wd=.4\paperwidth,ht=2.5ex,dp=1.125ex,leftskip=.3cm,rightskip=.3cm plus1fil]{title in head/foot}%
    \usebeamerfont{title in head/foot}\insertshorttitle
  \end{beamercolorbox}}%
    \begin{beamercolorbox}[wd=.1\paperwidth,ht=2.5ex,dp=1.125ex,leftskip=.1cm plus1fill,rightskip=.1cm]{date in head/foot}
      \usebeamerfont{date in head/foot} \insertframenumber{} / \inserttotalframenumber
    \end{beamercolorbox}
  \vskip0pt%
}

\defbeamertemplate*{headline}{mysplit theme}
{%
  \leavevmode%
  \begin{beamercolorbox}[wd=.4\paperwidth,ht=2.5ex,dp=1.125ex]{section in head/foot}%
    \insertsectionnavigationhorizontal{.4\paperwidth}{\hskip0pt plus1filll}{}%
  \end{beamercolorbox}%
  \begin{beamercolorbox}[wd=.6\paperwidth,ht=2.5ex,dp=1.125ex]{subsection in head/foot}%
    \insertsubsectionnavigationhorizontal{.6\paperwidth}{}{\hskip0pt plus1filll}%
  \end{beamercolorbox}%
}

\setbeamercolor{goodpractice head}{fg=white,bg=orange!85!black}
\setbeamercolor{goodpractice body}{fg=black,bg=orange!25}

%%%%%%%%%%%%
% packages %
%%%%%%%%%%%%

\usepackage[utf8]{inputenc}
\usepackage{newunicodechar}
\usepackage{scontents}
\makeatletter
\let\verbatimsc\@undefined
\let\endverbatimsc\@undefined
\makeatother
\usepackage{minted}
\newminted{tex}{linenos}
\newenvironment{verbatimsc}
               {\VerbatimEnvironment
                 \begin{minted}[linenos,escapeinside=||]{cpp}}
               {\end{minted}}
\newcommand\highlightCppCode[2]{
  \renewenvironment{verbatimsc}
                   {\VerbatimEnvironment
                     \begin{minted}[linenos,highlightlines={#1},escapeinside=||]{cpp}}
                   {\end{minted}}
  \typestored{#2}
}
\newminted{cpp}{gobble=4,linenos}
\newminted{shell-session}{gobble=4}
\newminted[makefile]{shell-session}{gobble=4}
\newminted{python}{linenos=true,gobble=4}

\usepackage{fancyvrb}
\newcommand*{\fvtextcolor}[2]{\textcolor{#1}{#2}}

\usepackage{pgf}
\usepackage{pgffor}
\usepackage{tikz}
\usetikzlibrary{arrows,arrows.meta,automata,positioning,snakes,shapes}

\usepackage{tcolorbox}

\usepackage[framemethod=TikZ]{mdframed}
\mdfdefinestyle{simplebox}{roundcorner=4pt,linewidth=0,backgroundcolor=blue!50!black,fontcolor=white}

\usepackage{multicol}
\usepackage{tikz-uml}

\usepackage{booktabs}
\usepackage{ifthen}

%%%%%%%%%%%%%%%%%%%
% useful commands %
%%%%%%%%%%%%%%%%%%%
\newcommand{\cpp}{C$^{++}$}
\newcommand{\deprecated}{\textcolor{red}{\bf Deprecated}}
\newcommand{\removed}{\textcolor{red}{\bf Removed}}


%%%%%%%%%%%%%%%%%%%%%%%%%%%%%%%%%%%%%%%%%%%%%%%%%
% boxes with good practices                     %
% Use \begin{goodpractice}{Title} to create one %
%%%%%%%%%%%%%%%%%%%%%%%%%%%%%%%%%%%%%%%%%%%%%%%%%
\makeatletter
\newcommand\listofgoodpractices{%
  \small
  \begin{multicols}{2}
    \begin{enumerate}
      \@starttoc{lgp}%
    \end{enumerate}
  \end{multicols}
}

\newcounter{gp@counter}
\def\gp@lasttitle{None}

\newenvironment{goodpractice}[1]
{%
  % Generate an entry in the index of good practices. In case of slide overlays, we only generate
  % an entry for the first slide of the overlay set by saving the last title of the
  % good practice box.
  \ifthenelse{\equal{\gp@lasttitle}{#1}}{}{%
    \hypertarget<\insertslidenumber>{goodpractice.\thegp@counter}{}
    \addtocontents{lgp}{\protect\item \protect\hyperlink{goodpractice.\thegp@counter}{#1\protect\hfill\insertpagenumber}}
    \stepcounter{gp@counter}
    \gdef\gp@lasttitle{#1}
  }
  % The box that shows up on the slide
  \begin{beamerboxesrounded}[upper=goodpractice head,lower=goodpractice body,shadow=true]{Good practice: #1}
}%
{%
  \end{beamerboxesrounded}
}
\makeatother

%%%%%%%%%%%%%%%%%%%%%%%%%%%%%%%
% frametitle with C++ version %
%%%%%%%%%%%%%%%%%%%%%%%%%%%%%%%
% Use as \frametitlecpp[14]{Title}
\newcommand\frametitlecpp[2][98]{
  \frametitle{#2 \hfill \cpp#1}
}

%%%%%%%%%%%%%%%%%%%%%%%%%%%%%%%
% easy class diagrams in tikz %
%%%%%%%%%%%%%%%%%%%%%%%%%%%%%%%

\newcommand\classbox[3][]{
  \def\temp{#3}
  \ifx\temp\empty
    \draw[thick] node (#2) [#1]
         [rectangle,rounded corners,draw] {#2};
  \else
    \draw[thick] node (#2) [#1]
         [rectangle,rounded corners,draw] {
      \begin{tabular}{l}
        \multicolumn{1}{c}{#2} \\
        \hline
        #3
      \end{tabular}
    };
  \fi
}

%%%%%%%%%%%%%%%%%%%%%%%%%%%%%%%%%%%%%%
% easy memory stack diagrams in tikz %
%%%%%%%%%%%%%%%%%%%%%%%%%%%%%%%%%%%%%%

\newcounter{memorystackindex}

\pgfkeys{
  memorystack/.is family,
  memorystack,
  size x/.initial=4cm,
  size y/.initial=.5cm,
  word size/.initial=4,
  block size/.initial=1,
  nb blocks/.initial=8,
  base address/.initial=12288,
  color/.initial=black,
  addresses/.initial=1
}

\makeatletter

\newcommand\memorystackset[1]{\pgfkeys{memorystack,#1}}
\newcommand\memorystack[1][]{
  \memorystackset{#1,
    size x/.get=\stacksizex,
    size y/.get=\stacksizey,
    word size/.get=\stackwordsize,
    block size/.get=\blocksize,
    nb blocks/.get=\stacknbblocks,
    base address/.get=\stackbaseaddr,
    color/.get=\stackcolor,
    addresses/.get=\displayaddrs
  }
  \draw[thick,\stackcolor,text=white] node (title)
        at (\stacksizex/2, \stacknbblocks*\stacksizey+.5cm)
        [rectangle,rounded corners,fill=blue!50!black] {Memory layout};
  \setcounter{memorystackindex}{1}
  \draw[thick,\stackcolor] (0,0) rectangle (\stacksizex,\stacknbblocks*\stacksizey);
  \pgfmathsetmacro{\nbbars}{\stacknbblocks-1}
  \pgfmathtruncatemacro\nbbarstrunc{\nbbars}
  \ifnum\nbbarstrunc>0
    \foreach \n in {1,...,\nbbars} {
      \draw[\stackcolor!70] (0,\n*\stacksizey) -- +(\stacksizex,0);
    }
  \fi
  \foreach \n in {1,...,\stacknbblocks} {
    \foreach \p in {1,...,\stackwordsize} {
      \draw node (stack\n-\p)
            at (\stacksizex/\stackwordsize*\p-\stacksizex/\stackwordsize/2,\n*\stacksizey-\stacksizey/2)
            [rectangle,minimum width=\stacksizex,minimum height=\stacksizey] {};
    }
    \ifnum1=\displayaddrs\relax
      \pgfmathparse{(\n-1)*\blocksize*\stackwordsize+\stackbaseaddr}
      \pgfmathdectoBase\hexversion{\pgfmathresult}{16}
      \draw node at (\stacksizex,\n*\stacksizey-\stacksizey/2) [right=2pt]
            {0x\hexversion};
    \fi
  }
  \pgfmathsetmacro{\nbseps}{\stackwordsize-1}
  \pgfmathtruncatemacro\nbsepstrunc{\nbseps}
  \ifnum\nbsepstrunc>0
    \foreach \n in {1,...,\nbseps} {
      \draw[\stackcolor!10] (\stacksizex/\stackwordsize*\n,0) -- +(0,\stacknbblocks*\stacksizey);
    }
  \fi
}

\newcommand\memorypushvalue[3]{
  \draw node at (stack#1-#2) {#3};
}

\newcommand\memorypushwidevalue[1]{
  \memorystackset{
    size x/.get=\stacksizex,
    size y/.get=\stacksizey,
  }
  \draw node (content) at (\stacksizex/2,\value{memorystackindex}*\stacksizey-\stacksizey/2) {#1};
  \draw[\stackcolor!80,->] (content) -- (.2cm,\value{memorystackindex}*\stacksizey-\stacksizey/2);
  \draw[\stackcolor!80,->] (content) -- (\stacksizex-.2cm,\value{memorystackindex}*\stacksizey-\stacksizey/2);
  \addtocounter{memorystackindex}{1}
}

\newcommand\memorypushhalfvalue[1]{
  \memorystackset{
    size x/.get=\stacksizex,
    size y/.get=\stacksizey,
  }
  \draw node (content) at (\stacksizex/4,\value{memorystackindex}*\stacksizey-\stacksizey/2) {#1};
  \draw[\stackcolor!80,->] (content) -- (.2cm,\value{memorystackindex}*\stacksizey-\stacksizey/2);
  \draw[\stackcolor!80,->] (content) -- (\stacksizex/2-.2cm,\value{memorystackindex}*\stacksizey-\stacksizey/2);
  \addtocounter{memorystackindex}{1}
}

\newcounter{localcount}
\newcommand\memorypush[1]{
  \memorystackset{
    word size/.get=\stackwordsize,
    nb blocks/.get=\stacknbblocks
  }
  \count@=0
  \setcounter{localcount}{1}
  \@for\v:=#1\do{
    \ifnum\count@<\stackwordsize
      \advance\count@ 1
      \memorypushvalue{\arabic{memorystackindex}}{\arabic{localcount}}{\v}
    \fi
    \addtocounter{localcount}{1}
  }
  \addtocounter{memorystackindex}{1}
}

\newcommand\memorypushpointer[2][]{
  \memorystackset{
    word size/.get=\stackwordsize,
    base address/.get=\stackbaseaddr,
    block size/.get=\blocksize,
  }
  \pgfmathparse{(#2-1)*\stackwordsize*\blocksize+\stackbaseaddr}
  \pgfmathdectoBase\hexaddress{\pgfmathresult}{16}
  \memorypushvalue{\arabic{memorystackindex}}{1}{#1 0x\hexaddress}
  \draw[\stackcolor!80,->] (stack\arabic{memorystackindex}-1.west) .. controls +(left:1) and +(left:1) .. (stack#2-1.west);
  \addtocounter{memorystackindex}{1}
}

\newcommand\memorystruct[3]{
  \memorystackset{
    size y/.get=\stacksizey
  }
  \draw[snake=brace,thick] (-2pt,#1*\stacksizey-\stacksizey) -- (-2pt,#2*\stacksizey)
    node [midway, above, sloped] {#3};
}

\newcommand\memorygoto[1]{
  \setcounter{memorystackindex}{#1}
}
\makeatother

%%%%%%%%%%%%%%%%%%
% Document setup %
%%%%%%%%%%%%%%%%%%

\title{HEP \cpp course}
\author[B. Gruber, S. Hageboeck, S. Ponce]{Based on the work of \\ S\'ebastien Ponce \\ \texttt{sebastien.ponce@cern.ch}}
\institute{CERN}
\date{March 2022}
\pgfdeclareimage[height=0.5cm]{cernlogo}{CERN-logo.jpg}
\logo{\pgfuseimage{cernlogo}}

\AtBeginSection[] {
  \begin{frame}<beamer>
    \frametitle{\insertsection}
    \begin{multicols}{2}
      \tableofcontents[sectionstyle=show/shaded,subsectionstyle=show/show/hide]
    \end{multicols}
  \end{frame}
}

\AtBeginSubsection[] {
  \begin{frame}<beamer>
    \frametitle{\insertsubsection}
    \tableofcontents[sectionstyle=show/hide,subsectionstyle=show/shaded/hide]
  \end{frame}
}

\hypersetup{
  colorlinks=true,
  allcolors=.,
  urlcolor={blue!80!white}
}
